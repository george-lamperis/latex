\RequirePackage[l2tabu, orthodox]{nag}

\documentclass[letterpaper, 12pt, oneside]{memoir}
\usepackage{amsmath,amsthm,amssymb,mathtools}
\usepackage{float} % H option to place tables exactly

% Memoir configuration
\pagestyle{plain}
\setlength{\parindent}{0cm}
\setlength{\parskip}{1ex}

% Omit chapter numbering
\counterwithout{section}{chapter}

% ------------------------------------------------------------------------------
% Theorem Commands
% ------------------------------------------------------------------------------
\newtheoremstyle{mystyle}% name
    {2ex}       % Space above
    {4ex}       % Space below
    {}          % Body font
    {}          % Indent amount (empty = no indent, \parindent = para indent)
    {\bfseries} % Thm head font
    {.}         % Punctuation after thm head
    {\newline}  % Space after thm head: \newline = linebreak
    {}          % Thm head spec

\theoremstyle{mystyle}

\newtheorem{thm}{Theorem}[section]
\newtheorem{prop}[thm]{Proposition}
\newtheorem{lemma}[thm]{Lemma}
\newtheorem{claim}[thm]{Claim}
\newtheorem{example}[thm]{Example}

% ------------------------------------------------------------------------------
% Begin Document
% ------------------------------------------------------------------------------
\title{Math 435 - Midterm 2}
\author{George Lamperis}
\date{}

\begin{document}
\maketitle


\section{Problem 6.4.}
% ------------------------------------------------------------------------------
% Problem 1
% ------------------------------------------------------------------------------
\section{}

For parts (a-i) express your answer as congruence conditions.
\begin{enumerate}[(a)]
\item Find all primes $p$ such that $(-1/p) = +1$. 

lets see. what happens
\item Find all primes $p$ such that $(-1/p) = -1$. 
\item Find all primes $p$ such that $(2/p) = +1$.
\item Find all primes $p$ such that $(2/p) = -1$.
\item Find all primes $p$ such that $(3/p) = +1$.
\item Find all primes $p$ such that $(3/p) = -1$.
\item Find all primes $p$ such that $(-6/p) = +1$. 
\item Find all primes $p$ such that $(-6/p) = -1$. 
\item Find all primes $p$ such that the equation
$3x^2 + 6x + 5 \equiv 0 \pmod{p}$ is solvable.
\end{enumerate}

Hints.  First assume your prime p is larger that 5; check primes 2, 3, and 5 
separately.  For (a) and (b) use Euler's formula for the quadratic residue 
symbol. There is a formula for (2/p). The p's you find will be given by
congruence conditions. For (e) and (f) use Quadratic Reciprocity. For (g) and 
(h) use multiplicativity of the quadratic symbol, and the Chinese Remainder 
Theorem.


\subsection{1(a)}
Theorem 22.1 states that if $p \equiv 1 \pmod 4$, then
$\left( \dfrac{-1}{p} \right)  = 1$ 

There is one remaining case, $p=2$. Notice $-1 \equiv 1 \pmod 2$. Also notice
that $1^2 \equiv 1 \pmod 2$, and so $1$ is a quadratic residue modulo 2. Hence
$\left( \dfrac{-1}{2} \right) = 1$.

Hence, $\left( \dfrac{-1}{p} \right)  = 1$ when $p = 2$ or $p \equiv 1 \pmod 4$.

List: $2$, $4k+1$

\subsection{1(b)}
Theorem 22.1 states that if $p \equiv 3 \pmod 4$, then
$\left( \dfrac{-1}{p} \right)  = -1$. Nothing further is required. 

\subsection{1(c,d)}
This also follows from Theorem 22.1. The only remaining case is $p=2$. But 
$2 \equiv 0 \pmod{2}$, and 0 is not a quadratic residue.

Hence, $\left( \dfrac{-1}{p} \right)  = 1$ if $p \equiv \pm 1 \pmod{8}$
and $\left( \dfrac{-1}{p} \right)  = -1$ if $p \equiv \pm 3 \pmod{8}$

\subsection{1(f)}

\begin{lemma}
Let $p$ be an odd prime, $p \neq 3$. Then
\[
\left( \frac{3}{p} \right) =
\begin{cases}
	1  & \text{if } p \equiv \pm 1 \pmod{12} \\
	-1 & \text{if } p \equiv \pm 5 \pmod{12}
\end{cases}
\]
\end{lemma}
\begin{proof}
First note that since $p$ is odd, $p \equiv 1 \pmod 4$ or $p \equiv 3 \pmod 4$.
Also, since $p$ is prime and $p > 3$, either $p \equiv 1 \pmod 3$ or 
$p \equiv 2 \pmod 3$.

% Case 1
\textbf{Case 1.}
Suppose $p \equiv 1 \pmod 4$ and $p \equiv 1 \pmod 3$. This implies that
$p \equiv 1 \pmod{12}$.

Then using the Law of Quadratic Reciprocity
\[ 
\left( \frac{3}{p} \right) = 
\left( \frac{p}{3} \right) = 
\left( \frac{1}{3} \right) = 1 
\]

Since $1^2 = 1$.


\end{proof}

\section{} 
% ------------------------------------------------------------------------------
% Problem 2
% ------------------------------------------------------------------------------
\section{}

Find all the solutions to $x^2 \equiv 1 \pmod{264}$.

Using a python script, I found that the solutions are 
\[ \{1, 23, 43, 65, 67, 89, 109, 131, 133, 155, 175, 197, 199, 221, 241, 263\}
\]

There are 16 solutions and also 16 divisors of 264. I think this is not a
coincidence. Also notice
\begin{align*}
	x &\equiv \pm 1 \\
	x &\equiv \pm 23 \\
	x &\equiv \pm 43 \\
	x &\equiv \pm 65 \\
	x &\equiv \pm 67 \\
	x &\equiv \pm 89 \\
	x &\equiv \pm 109 \\
	x &\equiv \pm 131 \\
\end{align*}

Unfortunately thats all I've got.

\section{Problem 11.12} 
% ------------------------------------------------------------------------------
% Problem 11.12
% ------------------------------------------------------------------------------

\begin{lemma}
    Let $n$ be an integers. Also let $p_1, p_2, \ldots, p_r$ be the prime
    factors of $n$ listed in ascending order, i.e. $p_1 < p_2 < \ldots < p_r$.
    
    If $\phi(n) = n/k$, for some positive integer $k$, then
    \[p_1 p_2 \ldots p_n = k(p_1 - 1)(p_2 - 1)(\ldots)(p_r - 1) \]
\end{lemma}
\begin{proof}
    Suppose $\phi(n) = n/k$. Then

    \begin{align*}
        \phi(n) &= n \left( 1 - \dfrac{1}{p_1} \right) \left( 1 - \dfrac{1}{p_2} \right) \dots \left( 1 - \dfrac{1}{p_r} \right) \\
        \dfrac{n}{k} &= n \left( 1 - \dfrac{1}{p_1} \right) \left( 1 - \dfrac{1}{p_2} \right) \dots \left( 1 - \dfrac{1}{p_r} \right) \\
        \dfrac{1}{k} &= \left( 1 - \dfrac{1}{p_1} \right) \left( 1 - \dfrac{1}{p_2} \right) \dots \left( 1 - \dfrac{1}{p_r} \right) \\
        \dfrac{1}{k} &= \dfrac{(p_1 - 1) \dots (p_r - 1)}{p_1 \dots p_r} \\
        p_1 p_2 \dots p_r &= k \cdot (p_1 - 1) \dots (p_r - 1) \\
    \end{align*}
\end{proof}

\begin{prop}[11.12a]
    An integer $n$ has the form $n=2^k$, $k \geq 1$ if and only if 
    $\phi(n) = n/2$.
\end{prop}
\begin{proof}
    Suppose $n=2^k$ with $k \geq 0$. Then $\phi(n) = 2^{k-1}$ and so we are
    done.
    
    Conversely, suppose $\phi(n)=n/2$. Let $p_1, p_2, \dots p_n$ be the distinct
    prime factors of $n$, listed in order i.e. $p_1 < p_2 < \cdots < p_r$.
    
    Then Lemma 3.1 says
    \[ p_1 p_2 \ldots p_r = 2 (p_1 - 1)(\ldots)(p_r - 1) \]
    Since 2 occurs on the right hand side, it must also be on the left hand
    side. Therefore, $p_1 = 2$. Now we have
    \begin{align*}
    2 \cdot p_2 \dots p_r &= 2 \cdot (2 - 1) \cdot (p_2 - 1) \dots (p_r - 1) \\
    2 \cdot p_2 \dots p_r &= 2 \cdot (p_2 - 1) \dots (p_r - 1) \\
    \end{align*}
    
    Notice the remaining unknowns on the right-hand side are all even, while
    the unknowns on the left-hand side are all odd. For this equality to hold,
    it must be the case that $r=1$ and $p_1 = 2$, i.e. $n=2^k$ with $k \geq 1$.
\end{proof}


\begin{prop}[11.12b]
    An integer $n$ has the form $n=2^{k_1} 3^{k_2}$, $k_i \geq 1$ if and only if 
    $\phi(n) = n/3$.
\end{prop}
\begin{proof}
    Suppose $n = 2^{k_1} 3^{k_2}$, $k_i \geq 1$. Then
    \begin{align*}
        \phi(n) &= \phi(2^{k_1} 3^{k_2}) \\
                &= \phi(2^{k_1}) \phi(3^{k_2}) \\
                &= (2^{k_1 - 1})(3^{k_2 - 1} \cdot 2) \\
                &= 2^{k_1} 3^{k_2 - 1} \\
                &= n/3
    \end{align*}
    
    Conversely, suppose $\phi(n) = n/3$. Then Lemma 3.1 says
    \[ p_1 p_2 \ldots p_r = 3 (p_1 - 1)(\ldots)(p_r - 1) \]
    
    Since 3 appears on the right-hand side, a 3 must appear on the left-hand 
    side also. This means for some $1 \leq i \leq r$, we have $p_i = 3$. Then
    the term $(p_i - 1) = 2$ occurs on the right-hand side, which means that
    a 2 must occur on the left-hand side also. Therefore, we have $p_1 = 2$ and
    $p_2 = 3$.
    
    Now we have
    \begin{align*}
    2 \cdot 3 \ldots p_r &= 3 (2 - 1)(3-1)(\ldots)(p_r - 1) \\
    2 \cdot 3 \ldots p_r &= 3 \cdot 2 \cdot (\ldots)(p_r - 1)
    \end{align*}
    Notice the remaining unknowns on the right-hand side are all even, while
    the unknowns on the left-hand side are all odd. For this equality to hold, 
    it must be the case that $r = 2$.
    
    Therefore, 2 and 3 are the only prime factors of $n$, i.e. 
    $n=2^{k_1} 3^{k_2}$.
\end{proof}


\begin{prop}[11.12c]
    There does not exist an $n$ such that $\phi(n) = n/6$.
\end{prop}
\begin{proof}
    Suppose $\phi(n) = n/6$. Then Lemma 3.1 says
    \begin{align*}
    p_1 p_2 \ldots p_r &= 6 (p_1 - 1)(\ldots)(p_r - 1) \\
                       &= 2 \cdot 3 (p_1 - 1)(\ldots)(p_r - 1) \\
    \end{align*}
    
    This implies that $p_1 = 2$ and $p_2 = 3$. Substituting these values, we
    have
    \begin{align*}
    2 \cdot 3 \ldots p_r &= 2 \cdot 3 (2 - 1)(3-1)(\ldots)(p_r - 1)\\
                         &= 2 \cdot 3 \cdot 2 \cdot (p_3 - 1)(\ldots)(p_r - 1)\\
    \end{align*} 
    
    As you can see, there are two 2's on the right hand side of the equation, 
    while there can only be one on the left (since every other prime is odd).
    And so we have a contradiction.
\end{proof}



% ------------------------------------------------------------------------------
% ------------------------------------------------------------------------------
\section{} 
4. Find all n such that  3|n-2, 5|n-3, 7|n-1. This should certainly remind you of the Chinese Remainder Theorem. 
\section{} 
5. Read Chapter 12 and do Problem 12.2(a). 
\section{} 
6. A function f: N to C is called multiplicative if f(mn) = f(m)f(n) whenever gcd(m,n)=1. The most famous example of a multiplicative function is Euler's phi-function. It should be clear to you that if f is multiplicative, the values of f on numbers of the form pk for various primes p identifies f. 
(a) Show that if f is multiplicative, then the function g(n) = sumd|n f(d) is multiplicative. 
(b) Use (a) to show that sumd|n phi(d) = n, where phi is the Euler phi-function. 

\end{document}
