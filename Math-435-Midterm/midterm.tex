\RequirePackage[l2tabu, orthodox]{nag}

\documentclass[letterpaper, 12pt, oneside]{memoir}
\usepackage{amsmath,amsthm,amssymb,mathtools}

% Memoir configuration
\pagestyle{plain}
\setlength{\parindent}{0cm}
\setlength{\parskip}{1ex}

% Omit chapter numbering
\counterwithout{section}{chapter}

% ------------------------------------------------------------------------------
% Theorem Commands
% ------------------------------------------------------------------------------
\newtheoremstyle{mystyle}% name
    {2ex}       % Space above
    {4ex}       % Space below
    {}          % Body font
    {}          % Indent amount (empty = no indent, \parindent = para indent)
    {\bfseries} % Thm head font
    {.}         % Punctuation after thm head
    {\newline}  % Space after thm head: \newline = linebreak
    {}          % Thm head spec

\theoremstyle{mystyle}

\newtheorem{thm}{Theorem}[section]
\newtheorem{prop}[thm]{Proposition}
\newtheorem{lemma}[thm]{Lemma}

% ------------------------------------------------------------------------------
% Begin Document
% ------------------------------------------------------------------------------
\title{Math 435 - Midterm 2}
\author{George Lamperis}
\date{}

\begin{document}
\maketitle

% ------------------------------------------------------------------------------
% ------------------------------------------------------------------------------
\section{Problem 6.4.}

\begin{lemma}
    Something profound here.
\end{lemma}

\begin{lemma}
    Something profound here.
\end{lemma}

\begin{lemma}
    Something profound here.
\end{lemma}

% ------------------------------------------------------------------------------
% ------------------------------------------------------------------------------
\section{} 
2. Show that for any integer n, n5/5 + n3/3 + 7n/15 is always an integer. Can you generalize?

\section{} 
3. Problem 11.12
\section{} 
4. Find all n such that  3|n-2, 5|n-3, 7|n-1. This should certainly remind you of the Chinese Remainder Theorem. 
\section{} 
5. Read Chapter 12 and do Problem 12.2(a). 
\section{} 
6. A function f: N to C is called multiplicative if f(mn) = f(m)f(n) whenever gcd(m,n)=1. The most famous example of a multiplicative function is Euler's phi-function. It should be clear to you that if f is multiplicative, the values of f on numbers of the form pk for various primes p identifies f. 
(a) Show that if f is multiplicative, then the function g(n) = sumd|n f(d) is multiplicative. 
(b) Use (a) to show that sumd|n phi(d) = n, where phi is the Euler phi-function. 

\end{document}
