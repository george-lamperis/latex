\RequirePackage[l2tabu, orthodox]{nag}

\documentclass[letterpaper, 12pt, oneside]{memoir}
\usepackage{amsmath,amsthm,amssymb,mathtools}
\usepackage{float} % H option to place tables exactly

% Memoir configuration
\pagestyle{plain}
\setlength{\parindent}{0cm}
\setlength{\parskip}{1ex}

% Omit chapter numbering
\counterwithout{section}{chapter}

% ------------------------------------------------------------------------------
% Theorem Commands
% ------------------------------------------------------------------------------
\newtheoremstyle{mystyle}% name
    {2ex}       % Space above
    {4ex}       % Space below
    {}          % Body font
    {}          % Indent amount (empty = no indent, \parindent = para indent)
    {\bfseries} % Thm head font
    {.}         % Punctuation after thm head
    {\newline}  % Space after thm head: \newline = linebreak
    {}          % Thm head spec

\theoremstyle{mystyle}

\newtheorem{thm}{Theorem}[section]
\newtheorem{prop}[thm]{Proposition}
\newtheorem{lemma}[thm]{Lemma}
\newtheorem{claim}[thm]{Claim}
\newtheorem{example}[thm]{Example}

% ------------------------------------------------------------------------------
% Begin Document
% ------------------------------------------------------------------------------
\title{Math 435 - Midterm 2}
\author{George Lamperis}
\date{}

\begin{document}
\maketitle

\section{Problem 6.4.}
% ------------------------------------------------------------------------------
% Problem 1
% ------------------------------------------------------------------------------
\section{}

For parts (a-i) express your answer as congruence conditions.
\begin{enumerate}[(a)]
\item Find all primes $p$ such that $(-1/p) = +1$. 
\item Find all primes $p$ such that $(-1/p) = -1$. 
\item Find all primes $p$ such that $(2/p) = +1$.
\item Find all primes $p$ such that $(2/p) = -1$.
\item Find all primes $p$ such that $(3/p) = +1$.
\item Find all primes $p$ such that $(3/p) = -1$.
\item Find all primes $p$ such that $(-6/p) = +1$. 
\item Find all primes $p$ such that $(-6/p) = -1$. 
\item Find all primes $p$ such that the equation
$3x^2 + 6x + 5 \equiv 0 \pmod{p}$ is solvable.
\end{enumerate}

Hints.  First assume your prime p is larger that 5; check primes 2, 3, and 5 
separately.  For (a) and (b) use Euler's formula for the quadratic residue 
symbol. There is a formula for (2/p). The p's you find will be given by
congruence conditions. For (e) and (f) use Quadratic Reciprocity. For (g) and 
(h) use multiplicativity of the quadratic symbol, and the Chinese Remainder 
Theorem.


\subsection{2(a)}
Theorem 22.1 states that if $p \equiv 1 \pmod 4$, then
$\left( \dfrac{-1}{p} \right)  = 1$ 

There is one remaining case, $p=2$. Notice $-1 \equiv 1 \pmod 2$. Also notice
that $1^2 \equiv 1 \pmod 2$, and so $1$ is a quadratic residue modulo 2. Hence
$\left( \dfrac{-1}{2} \right) = 1$.

Hence, $\left( \dfrac{-1}{p} \right)  = 1$ when $p = 2$ or $p \equiv 1 \pmod 4$.


\subsection{2(b)}
Theorem 22.1 states that if $p \equiv 3 \pmod 4$, then
$\left( \dfrac{-1}{p} \right)  = -1$. Nothing further is required. 

\section{} 
% ------------------------------------------------------------------------------
% Problem 2
% ------------------------------------------------------------------------------
\section{}

Find all the solutions to $x^2 \equiv 1 \pmod{264}$.

Using a python script, I found that the solutions are 
\[ \{1, 23, 43, 65, 67, 89, 109, 131, 133, 155, 175, 197, 199, 221, 241, 263\}
\]

There are 16 solutions and also 16 divisors of 264. I think this is not a
coincidence. Also notice
\begin{align*}
	x &\equiv \pm 1 \\
	x &\equiv \pm 23 \\
	x &\equiv \pm 43 \\
	x &\equiv \pm 65 \\
	x &\equiv \pm 67 \\
	x &\equiv \pm 89 \\
	x &\equiv \pm 109 \\
	x &\equiv \pm 131 \\
\end{align*}

Unfortunately thats all I've got.

\section{Problem 11.12} 
% ------------------------------------------------------------------------------
% Problem 3
% ------------------------------------------------------------------------------
\section{}

If $p > 5$ is a prime expressible as $a^2 + 5 b^2$, then $p \equiv 1 \pmod{20}$,
or $p \equiv 9 \pmod{20}$.

For example $29 = 3^2 + 5 \cdot 2^2$, and clearly $29 \equiv 9 \pmod{20}$. 

\begin{proof}
First some data:
\begin{table}[H]
\centering
\begin{subtable}[b]{.3\linewidth}
	\centering
	\begin{tabular}{c|c}
		$a$ & $a^2$ \\ \hline
		0 & 0 \\
		1 & 1 \\
		2 & 4 \\
		3 & 4 \\
		4 & 1 \\
	\end{tabular}
	\caption*{modulo 5}
\end{subtable}
\begin{subtable}[b]{.3\linewidth}
	\centering
	\begin{tabular}{c|c}
		$a$ & $a^2$ \\ \hline
		0 & 0 \\
		1 & 1 \\
		2 & 4 \\
		3 & 9 \\
		4 & 16 \\
		5 & 5 \\
		6 & 16 \\
		7 & 9 \\
		8 & 4 \\
		9 & 1 \\
	\end{tabular}
	\caption*{modulo 20}
\end{subtable}
\begin{subtable}[b]{.3\linewidth}
	\centering
	\begin{tabular}{c|c}
		$a$ & $a^2$ \\ \hline
		10 & 0 \\
		11 & 1 \\
		12 & 4 \\
		13 & 9 \\
		14 & 16 \\
		15 & 5 \\
		16 & 16 \\
		17 & 9 \\
		18 & 4 \\
		19 & 1
	\end{tabular}
	\caption*{}
\end{subtable}
\end{table}


We know every odd integer has the form $4k+1$ or $4k+3$. Hence, either
$p \equiv 1 \pmod 4$ or $p \equiv 3 \pmod 4$ .

Also note that $p = a^2 + 5b^2 \equiv a^2 \pmod 5$ i.e. $p$ is a quadratic
residue modulo 5. However, 1 and 4 are the only quadratic residues modulo 5.
Therefore, $p \equiv 1 \pmod 5$ or $p \equiv 4 \pmod 5$.

\begin{case}
$p \equiv 1 \pmod 4$ and $a^2 \equiv 1 \pmod 5$

Then
\begin{align*}
		  p &\equiv a^2 \pmod 5 \\
	 4k + 1 &\equiv 1 \pmod 5 \\
	     4k &\equiv 0 \pmod 5 \\
	      k &\equiv 0 \pmod 5
\end{align*}
So $k$ has the form $k=4m$. Substitute this back into our equation.
\begin{align*}
	p &= 4k + 1 \\
	  &= 4(5m) + 1 \\
	  &= 20m + 1
\end{align*}
Clearly $p \equiv 1 \pmod{20}$.
\end{case}

\begin{case}
$p \equiv 1 \pmod 4$ and $a^2 \equiv 1 \pmod 5$

Then
\begin{align*}
		  p &\equiv a^2 \pmod 5 \\
	 4k + 1 &\equiv 1 \pmod 5 \\
	     4k &\equiv 0 \pmod 5 \\
	      k &\equiv 0 \pmod 5
\end{align*}
So $k$ has the form $k=4m$. Substitute this back into our equation.
\begin{align*}
	p &= 4k + 1 \\
	  &= 4(5m) + 1 \\
	  &= 20m + 1
\end{align*}
Clearly $p \equiv 1 \pmod{20}$.
\end{case}

\begin{case}
$p \equiv 1 \pmod 4$ and $a^2 \equiv 1 \pmod 5$

Then
\begin{align*}
		  p &\equiv a^2 \pmod 5 \\
	 4k + 1 &\equiv 1 \pmod 5 \\
	     4k &\equiv 0 \pmod 5 \\
	      k &\equiv 0 \pmod 5
\end{align*}
So $k$ has the form $k=4m$. Substitute this back into our equation.
\begin{align*}
	p &= 4k + 1 \\
	  &= 4(5m) + 1 \\
	  &= 20m + 1
\end{align*}
Clearly $p \equiv 1 \pmod{20}$.
\end{case}

\begin{case}
$p \equiv 1 \pmod 4$ and $a^2 \equiv 1 \pmod 5$

Then
\begin{align*}
		  p &\equiv a^2 \pmod 5 \\
	 4k + 1 &\equiv 1 \pmod 5 \\
	     4k &\equiv 0 \pmod 5 \\
	      k &\equiv 0 \pmod 5
\end{align*}
So $k$ has the form $k=4m$. Substitute this back into our equation.
\begin{align*}
	p &= 4k + 1 \\
	  &= 4(5m) + 1 \\
	  &= 20m + 1
\end{align*}
Clearly $p \equiv 1 \pmod{20}$.
\end{case}

\end{proof}



\section{} 
Find all n such that  3|n-2, 5|n-3, 7|n-1. This should certainly remind you of the Chinese Remainder Theorem. 

\section{Problem 12.2(a)} 
% ------------------------------------------------------------------------------
% Problem 5
% ------------------------------------------------------------------------------
\section{}

Show that there are infinitely many primes of the form $4k+1$. 

Hint: By the formula used for part (a) of Problem 1 we know that if an odd prime
$p$ divides a number of the form $x^2 + 1$, then $p$ must be of the form $4k+1$.
Now using Euclid's idea for the infinitude of prime numbers, consider numbers of
the form $(N!)^2 + 1$ for appropriate numbers $N$.


\section{} 
% ------------------------------------------------------------------------------
% Problem 6
% ------------------------------------------------------------------------------

A function f: N to C is called multiplicative if f(mn) = f(m)f(n) whenever gcd(m,n)=1. The most famous example of a multiplicative function is Euler's phi-function. It should be clear to you that if f is multiplicative, the values of f on numbers of the form pk for various primes p identifies f. 
(a) Show that if f is multiplicative, then the function g(n) = sumd|n f(d) is multiplicative. 
(b) Use (a) to show that sumd|n phi(d) = n, where phi is the Euler phi-function. 

\begin{lemma}
    If $f$ is multiplicative, then $f(1) = 1$
\end{lemma}
\begin{proof}
    Suppose for contradiction that $f(1) \neq 1$. Additionally, let $p$ be
    prime.
    
    Since $\gcd(1, p) = 1$, we have $f(p) = f(1)f(p)$. But $f(1) \neq 1$ and so 
    we have a contradiction. Hence, $f(1) = 1$. 
\end{proof}

\begin{prop}[part a]
    If $f$ is a multiplicative function, then the function
    \[ g(n) = \sum_{d \mid n} f(d) \]
    is multiplicative.
\end{prop}
\begin{proof}
\end{proof}
    Let $p, q$ be primes. 
    \[ g(p^\alpha) = f(1) + f(p) + \ldots + f(p^\alpha) \]
    and
    \[ g(q^\beta) = f(1) + f(q) + \ldots + f(q^\beta) \]
    
    Then
    \begin{align*}
    g(p^\alpha q^\beta) 
        &= f(p^0 q^0) + f(p^1 q^0) + \dots + f(p^\alpha q^0) \\
        & \;\;\;\; + f(p^0 q^1) + f(p^1 q^1) + \dots + f(p^\alpha q^1) \\
        & \;\;\;\; + \dots \\
        & \;\;\;\; + f(p^0 q^\beta) + f(p^1 q^0) + \ldots + f(p^\alpha q^\beta) \\
        \\    
        %-----------------------------------------------------------------------
        &= f(p^0)f(q^0) + f(p^1)f(q^0) + \dots + f(p^\alpha)f(q^0) \\
        & \;\;\;\; + f(p^0)f(q^1) + f(p^1)f(q^1) + \dots + f(p^\alpha)f(q^1) \\
        & \;\;\;\; + \dots \\
        & \;\;\;\; + f(p^0)f(q^\beta) + f(p^1)f(q^\beta) + \dots + f(p^\alpha)f(q^\beta) \\
        \\
        %-----------------------------------------------------------------------     
        &= f(q^0)[f(p^0) + f(p^1) + \ldots + f(p^\alpha)] \\
        & \;\;\;\; + f(q^1)[f(p^0) + f(p^1) + \ldots + f(p^\alpha)] \\
        & \;\;\;\; + \ldots \\
        & \;\;\;\; + f(q^\beta)[f(p^0) + f(p^1) + \ldots + f(p^\alpha)] \\
        \\
        %-----------------------------------------------------------------------     
        &= [f(p^0) + f(p^1) + \ldots + f(p^\alpha)] \cdot [f(q^0) + f(q^1) + \ldots + f(q^\beta)] \\
        %-----------------------------------------------------------------------     
        &= f(p^\alpha)f(q^\beta)
    \end{align*}
    
    Since each number has a prime factorization, this applies for all integers.
\begin{prop}[part b]
    The function
    \[ F(n) = \sum_{d \mid n} \phi(d) = n \]
\end{prop}
\begin{proof}
    Let $p$ be prime. Then
    \begin{align*}
    F(p^k) &= \sum_{d \mid n} \phi(d) \\
           &= \phi(1) + \phi(p) + \phi(p^2) + \ldots + \phi(p^k) \\
           &= 1 + (p-1) + (p^2 - p) + \ldots + (p^{k-1} - p^{k-2}) + (p^k - p^{k-1}) \\
           &= (1-1) + (p-p) + (p^2 - p^2) \ldots + (p^{k-1} - p^{k-1}) + p^k \\
           &= p^k
    \end{align*}    

    Since $\phi$ is multiplicative, Proposition 6.2 says that $F$ is
    multiplicative. Therefore, $F(n) = n$ for all $n$.
\end{proof}


\end{document}
