\RequirePackage[l2tabu, orthodox]{nag}

\documentclass[letterpaper, 12pt, oneside]{memoir}
\usepackage{amsmath,amsthm,amssymb,mathtools}
\usepackage{float} % H option to place tables exactly

% Memoir configuration
\pagestyle{plain}
\setlength{\parindent}{0cm}
\setlength{\parskip}{1ex}

% Omit chapter numbering
\counterwithout{section}{chapter}

% ------------------------------------------------------------------------------
% Theorem Commands
% ------------------------------------------------------------------------------
\newtheoremstyle{mystyle}% name
    {2ex}       % Space above
    {4ex}       % Space below
    {}          % Body font
    {}          % Indent amount (empty = no indent, \parindent = para indent)
    {\bfseries} % Thm head font
    {.}         % Punctuation after thm head
    {\newline}  % Space after thm head: \newline = linebreak
    {}          % Thm head spec

\theoremstyle{mystyle}

\newtheorem{thm}{Theorem}[section]
\newtheorem{prop}[thm]{Proposition}
\newtheorem{lemma}[thm]{Lemma}
\newtheorem{claim}[thm]{Claim}
\newtheorem{example}[thm]{Example}

% ------------------------------------------------------------------------------
% Begin Document
% ------------------------------------------------------------------------------
\title{Math 435 - Midterm 2}
\author{George Lamperis}
\date{}

\begin{document}
\maketitle

% ------------------------------------------------------------------------------
% Problem 6.4
% ------------------------------------------------------------------------------
\section{Problem 6.4.}

\begin{lemma}
    Given integers $a$ and $b$, $\gcd(a,b) = 1$ if and only if there are
    integers $x, y$ such that $ax + by = 1$.
\end{lemma}
\begin{proof}
    Suppose $\gcd(a,b) = 1$. Then by Theorem 6.1, there are integers $x,y$ such
    that $ax+by=1$ and so we are done.
    
    Conversely, suppose that there are integers $x,y$ such that $ax+by=1$. 
    
    Let the $\gcd(a,b)=g$. Since $g \mid a$ and $g \mid b$, it follows that
    \begin{align*}
        g & \mid ax+by \\
        g & \mid 1
    \end{align*}
    But the only divisor of 1 is 1. Hence, $g=1$.
\end{proof}
    
\begin{lemma}
    If $\gcd(a,b,c) = g$, then there are integers $x,y,z$ such that 
    $ax + by + cz = g$
\end{lemma}
\begin{proof}
    Let $d=\gcd(a,b)$. Then there are integers $x,y$ such that $ax+by=d$.
    
    Now let $g=gcd(d,c)$. There are integers $w,z$ such that $dw+cz=g$. Then:
    \begin{align*}
        g &= dw + cz \\
          &= (ax+by)w + cz \\
          &= axw + byw + cz
    \end{align*}
\end{proof}

\begin{prop}
    Let $a,b,c$ be integers. Then $gcd(a,b,c) = 1$ if and only if there are
    integers $x,y,z$ such that $ax+by+cz=1$.
\end{prop}
\begin{proof}
    Suppose $gcd(a,b,c) = 1$. Then the previous lemma says we are done.
    
    Conversely, suppose there are $x,y,z$ such that $ax+by+cz=1$. Since 
    $g \mid a$, $g \mid b$, and $g \mid c$ it follows that
    \begin{align*}
        g & \mid ax+by+cz \\
        g & \mid 1
    \end{align*}
    But the only divisor of 1 is 1. Hence $g=1$.
\end{proof}

Let $a, b$ and $c$ be integers with $\gcd(a,b,c) = 1$. In general, to find $x,
y$ and $z$ such that $ax + by + cz = 1$:

First calculate $d = \gcd(a,b)$ using the Euclidian Algorithm. Then, using your
computations, find $r$ and $s$ such that 
\[ ar + bs = d \]

Next calcuate $1 = \gcd(d, c)$ using the Euclidian Algorithm. Find $w$ and
$z$ such that
\[dw + cz = 1\]

Now subsitute $d = ar + bs$:
\begin{align*}
    (ar + bs)w + cz &= 1  \\
    arw + bsw + cz  &= 1
\end{align*}

Now we have $x = rw$, $y = sw$ and $z$ such that $ax + by + cz = 1$.

\begin{example}
    Find integers $x, y$ and $z$ that satisfy the equation
    \[ 6x + 15y + 20z = 1\]
    
    First calculate $gcd(6,15) = 3$ using the Euclidian Algorithm.
    \begin{table}[H]
    \centering
    \begin{tabular}{c|c|c|c}
            & 2  & 2  &    \\ \hline
        15  & 6  & 3  & 0  \\ \hline
        12  & 6  &    &
    \end{tabular}
    \end{table}

    Using the above table, we back substitute to write $3 = 15(1) + 2(-6)$.
    
    Now calculate $gcd(3,20) = 1$
    \begin{table}[H]
    \centering 
    \begin{tabular}{c|c|c|c|c}
            & 6  & 1  & 2  &    \\ \hline
        20  & 3  & 2  & 1  & 0  \\ \hline
        18  & 2  & 2  &    &
    \end{tabular}
    \end{table}
    
    Back substitute to write 
    \begin{align*}
        1 &= 3 \cdot 7 + 20 \cdot (-1) \\
          &= (6 \cdot (-2) + 15 \cdot 1)7 + 20 \cdot (-1) \\
          &= 6 \cdot (-14) + 15 \cdot 7 + 20 \cdot (-1)
    \end{align*}
    
    And so we have $x = -14$, $y = 7$, and $z=-1$.
\end{example}


\begin{example}
    Find integers $x, y$ and $z$ that satisfy the equation
    \[155x + 341y + 385z = 1 \]
    
    First calculate $gcd(155, 341) = 31$
    \begin{table}[H]
    \centering
    \begin{tabular}{c|c|c|c}
            & 2   & 5  &   \\ \hline
        341 & 155 & 31 & 0 \\ \hline
        310 & 155 &    & 
    \end{tabular}
    \end{table}
    
    Back substitute to write $155(-2) + 341(1)= 31$.
    
    Now calculate $gcd(31, 385) = 1$.
     \begin{table}[H]
    \centering
    \begin{tabular}{c|c|c|c|c|c|c}
            & 12  & 2   & 2  & 1  & 1  &   \\ \hline
        385 & 31  & 13  & 5  & 3  & 2  & 1 \\ \hline
        372 & 25  & 10  & 3  & 2  &    & 
    \end{tabular}
    \end{table}
    
    Then write $1 =  31 \cdot (-149) + 385 \cdot 12 $.
    
    Now substitute $31 = 155 \cdot (-2) + 341 \cdot 1$:
    \begin{align*}
        1 &= 31 \cdot (-149) + 385 \cdot 12 \\
          &= (155 \cdot (-2) + 341 \cdot 1) (-149) + 385 \cdot 12 \\
          &= 155 \cdot 298 + 341 \cdot (-149) + 385 \cdot 12
    \end{align*} 
    
    And so we have $x = 298$, $y = -149$ and $z = 12$.
    
\end{example}

% ------------------------------------------------------------------------------
% ------------------------------------------------------------------------------
\section{} 
2. Show that for any integer n, n5/5 + n3/3 + 7n/15 is always an integer. Can you generalize?

% ------------------------------------------------------------------------------
% Problem 11.12
% ------------------------------------------------------------------------------
\section{Problem 11.12} 

\begin{lemma}
    Let $n$ be an integers. Also let $p_1, p_2, \ldots, p_r$ be the prime
    factors of $n$ listed in ascending order, i.e. $p_1 < p_2 < \ldots < p_r$.
    
    If $\phi(n) = n/k$, for some positive integer $k$, then
    \[p_1 p_2 \ldots p_n = k(p_1 - 1)(p_2 - 1)(\ldots)(p_r - 1) \]
\end{lemma}
\begin{proof}
    Suppose $\phi(n) = n/k$. Then

    \begin{align*}
        \phi(n) &= n \left( 1 - \dfrac{1}{p_1} \right) \left( 1 - \dfrac{1}{p_2} \right) \dots \left( 1 - \dfrac{1}{p_r} \right) \\
        \dfrac{n}{k} &= n \left( 1 - \dfrac{1}{p_1} \right) \left( 1 - \dfrac{1}{p_2} \right) \dots \left( 1 - \dfrac{1}{p_r} \right) \\
        \dfrac{1}{k} &= \left( 1 - \dfrac{1}{p_1} \right) \left( 1 - \dfrac{1}{p_2} \right) \dots \left( 1 - \dfrac{1}{p_r} \right) \\
        \dfrac{1}{k} &= \dfrac{(p_1 - 1) \dots (p_r - 1)}{p_1 \dots p_r} \\
        p_1 p_2 \dots p_r &= k \cdot (p_1 - 1) \dots (p_r - 1) \\
    \end{align*}
\end{proof}

\begin{prop}[11.12a]
    An integer $n$ has the form $n=2^k$, $k \geq 1$ if and only if 
    $\phi(n) = n/2$.
\end{prop}
\begin{proof}
    Suppose $n=2^k$ with $k \geq 0$. Then $\phi(n) = 2^{k-1}$ and so we are
    done.
    
    Conversely, suppose $\phi(n)=n/2$. Let $p_1, p_2, \dots p_n$ be the distinct
    prime factors of $n$, listed in order i.e. $p_1 < p_2 < \cdots < p_r$.
    
    Then Lemma 3.1 says
    \[ p_1 p_2 \ldots p_r = 2 (p_1 - 1)(\ldots)(p_r - 1) \]
    Since 2 occurs on the right hand side, it must also be on the left hand
    side. Therefore, $p_1 = 2$. Now we have
    \begin{align*}
    2 \cdot p_2 \dots p_r &= 2 \cdot (2 - 1) \cdot (p_2 - 1) \dots (p_r - 1) \\
    2 \cdot p_2 \dots p_r &= 2 \cdot (p_2 - 1) \dots (p_r - 1) \\
    \end{align*}
    
    Notice the remaining unknowns on the right-hand side are all even, while
    the unknowns on the left-hand side are all odd. For this equality to hold,
    it must be the case that $r=1$ and $p_1 = 2$, i.e. $n=2^k$ with $k \geq 1$.
\end{proof}


\begin{prop}[11.12b]
    An integer $n$ has the form $n=2^{k_1} 3^{k_2}$, $k_i \geq 1$ if and only if 
    $\phi(n) = n/3$.
\end{prop}
\begin{proof}
    Suppose $n = 2^{k_1} 3^{k_2}$, $k_i \geq 1$. Then
    \begin{align*}
        \phi(n) &= \phi(2^{k_1} 3^{k_2}) \\
                &= \phi(2^{k_1}) \phi(3^{k_2}) \\
                &= (2^{k_1 - 1})(3^{k_2 - 1} \cdot 2) \\
                &= 2^{k_1} 3^{k_2 - 1} \\
                &= n/3
    \end{align*}
    
    Conversely, suppose $\phi(n) = n/3$. Then Lemma 3.1 says
    \[ p_1 p_2 \ldots p_r = 3 (p_1 - 1)(\ldots)(p_r - 1) \]
    
    Since 3 appears on the right-hand side, a 3 must appear on the left-hand 
    side also. This means for some $1 \leq i \leq r$, we have $p_i = 3$. Then
    the term $(p_i - 1) = 2$ occurs on the right-hand side, which means that
    a 2 must occur on the left-hand side also. Therefore, we have $p_1 = 2$ and
    $p_2 = 3$.
    
    Now we have
    \begin{align*}
    2 \cdot 3 \ldots p_r &= 3 (2 - 1)(3-1)(\ldots)(p_r - 1) \\
    2 \cdot 3 \ldots p_r &= 3 \cdot 2 \cdot (\ldots)(p_r - 1)
    \end{align*}
    Notice the remaining unknowns on the right-hand side are all even, while
    the unknowns on the left-hand side are all odd. For this equality to hold, 
    it must be the case that $r = 2$.
    
    Therefore, 2 and 3 are the only prime factors of $n$, i.e. 
    $n=2^{k_1} 3^{k_2}$.
\end{proof}


\begin{prop}[11.12c]
    $\phi(n) = n/6$.
\end{prop}

% ------------------------------------------------------------------------------
% ------------------------------------------------------------------------------
\section{} 
4. Find all n such that  3|n-2, 5|n-3, 7|n-1. This should certainly remind you of the Chinese Remainder Theorem. 
\section{} 
5. Read Chapter 12 and do Problem 12.2(a). 
\section{} 
6. A function f: N to C is called multiplicative if f(mn) = f(m)f(n) whenever gcd(m,n)=1. The most famous example of a multiplicative function is Euler's phi-function. It should be clear to you that if f is multiplicative, the values of f on numbers of the form pk for various primes p identifies f. 
(a) Show that if f is multiplicative, then the function g(n) = sumd|n f(d) is multiplicative. 
(b) Use (a) to show that sumd|n phi(d) = n, where phi is the Euler phi-function. 

\end{document}
