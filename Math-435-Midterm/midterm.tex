\RequirePackage[l2tabu, orthodox]{nag}

\documentclass[letterpaper, 12pt, oneside]{memoir}
\usepackage{amsmath,amsthm,amssymb,mathtools}
\usepackage{float} % H option to place tables exactly

% Memoir configuration
\pagestyle{plain}
\setlength{\parindent}{0cm}
\setlength{\parskip}{1ex}

% Omit chapter numbering
\counterwithout{section}{chapter}

% ------------------------------------------------------------------------------
% Theorem Commands
% ------------------------------------------------------------------------------
\newtheoremstyle{mystyle}% name
    {2ex}       % Space above
    {4ex}       % Space below
    {}          % Body font
    {}          % Indent amount (empty = no indent, \parindent = para indent)
    {\bfseries} % Thm head font
    {.}         % Punctuation after thm head
    {\newline}  % Space after thm head: \newline = linebreak
    {}          % Thm head spec

\theoremstyle{mystyle}

\newtheorem{thm}{Theorem}[section]
\newtheorem{prop}[thm]{Proposition}
\newtheorem{lemma}[thm]{Lemma}
\newtheorem{claim}[thm]{Claim}
\newtheorem{example}[thm]{Example}

% ------------------------------------------------------------------------------
% Begin Document
% ------------------------------------------------------------------------------
\title{Math 435 - Midterm 2}
\author{George Lamperis}
\date{}

\begin{document}
\maketitle

% ------------------------------------------------------------------------------
% Problem 6.4
% ------------------------------------------------------------------------------
\section{Problem 6.4.}

\begin{lemma}
    Given integers $a$ and $b$, $\gcd(a,b) = 1$ if and only if there are
    integers $x, y$ such that $ax + by = 1$.
\end{lemma}
\begin{proof}
    Suppose $\gcd(a,b) = 1$. Then by Theorem 6.1, there are integers $x,y$ such
    that $ax+by=1$ and so we are done.
    
    Conversely, suppose that there are integers $x,y$ such that $ax+by=1$. 
    
    Let the $\gcd(a,b)=g$. Since $g \mid a$ and $g \mid b$, it follows that
    \begin{align*}
        g & \mid ax+by \\
        g & \mid 1
    \end{align*}
    But the only divisor of 1 is 1. Hence, $g=1$.
\end{proof}
    
\begin{lemma}
    If $\gcd(a,b,c) = g$, then there are integers $x,y,z$ such that 
    $ax + by + cz = g$
\end{lemma}
\begin{proof}
    Let $d=\gcd(a,b)$. Then there are integers $x,y$ such that $ax+by=d$.
    
    Now let $g=gcd(d,c)$. There are integers $w,z$ such that $dw+cz=g$. Then:
    \begin{align*}
        g &= dw + cz \\
          &= (ax+by)w + cz \\
          &= axw + byw + cz
    \end{align*}
\end{proof}

\begin{prop}
    Let $a,b,c$ be integers. Then $gcd(a,b,c) = 1$ if and only if there are
    integers $x,y,z$ such that $ax+by+cz=1$.
\end{prop}
\begin{proof}
    Suppose $gcd(a,b,c) = 1$. Then the previous lemma says we are done.
    
    Conversely, suppose there are $x,y,z$ such that $ax+by+cz=1$. Since 
    $g \mid a$, $g \mid b$, and $g \mid c$ it follows that
    \begin{align*}
        g & \mid ax+by+cz \\
        g & \mid 1
    \end{align*}
    But the only divisor of 1 is 1. Hence $g=1$.
\end{proof}

\begin{example}
    Find integers $x, y$ and $z$ that satisfy the equation
    \[ 6x + 15y + 20z = 1\]
    
    First calculate $gcd(6,15) = 3$ using the Euclidian Algorithm.
    \begin{table}[H]
    \centering
    \begin{tabular}{c|c|c|c}
            & 2  & 2  &    \\ \hline
        15  & 6  & 3  & 0  \\ \hline
        12  & 6  &    &
    \end{tabular}
    \end{table}

    Using the above table, we back substitute to write $3 = 15(1) + 2(-6)$.
    
    Now calculate $gcd(3,20) = 1$
    \begin{table}[H]
    \centering 
    \begin{tabular}{c|c|c|c|c}
            & 6  & 1  & 2  &    \\ \hline
        20  & 3  & 2  & 1  & 0  \\ \hline
        18  & 2  & 2  &    &
    \end{tabular}
    \end{table}
    
    We back substitute to write 
    \begin{align*}
        1 &= 3 \cdot 7 + 20 \cdot 1 \\
          &= (6 \cdot (-2) + 15 \cdot 1)7 + 20(1) \\
          &= 6 \cdot (-14) + 15 \cdot 7 + 20 \cdot (-1)
    \end{align*}
    
    And so we have $x = -14$, $y = 7$, and $z=-1$.
    
\end{example}

% ------------------------------------------------------------------------------
% ------------------------------------------------------------------------------
\section{} 
2. Show that for any integer n, n5/5 + n3/3 + 7n/15 is always an integer. Can you generalize?

\section{} 
3. Problem 11.12
\section{} 
4. Find all n such that  3|n-2, 5|n-3, 7|n-1. This should certainly remind you of the Chinese Remainder Theorem. 
\section{} 
5. Read Chapter 12 and do Problem 12.2(a). 
\section{} 
6. A function f: N to C is called multiplicative if f(mn) = f(m)f(n) whenever gcd(m,n)=1. The most famous example of a multiplicative function is Euler's phi-function. It should be clear to you that if f is multiplicative, the values of f on numbers of the form pk for various primes p identifies f. 
(a) Show that if f is multiplicative, then the function g(n) = sumd|n f(d) is multiplicative. 
(b) Use (a) to show that sumd|n phi(d) = n, where phi is the Euler phi-function. 

\end{document}
