% ------------------------------------------------------------------------------
% Problem 6
% ------------------------------------------------------------------------------

A function f: N to C is called multiplicative if f(mn) = f(m)f(n) whenever gcd(m,n)=1. The most famous example of a multiplicative function is Euler's phi-function. It should be clear to you that if f is multiplicative, the values of f on numbers of the form pk for various primes p identifies f. 
(a) Show that if f is multiplicative, then the function g(n) = sumd|n f(d) is multiplicative. 
(b) Use (a) to show that sumd|n phi(d) = n, where phi is the Euler phi-function. 

\begin{lemma}
    If $f$ is multiplicative, then $f(1) = 1$
\end{lemma}
\begin{proof}
    Suppose for contradiction that $f(1) \neq 1$. Additionally, let $p$ be
    prime.
    
    Since $\gcd(1, p) = 1$, we have $f(p) = f(1)f(p)$. But $f(1) \neq 1$ and so 
    we have a contradiction. Hence, $f(1) = 1$. 
\end{proof}

\begin{prop}[part a]
    If $f$ is a multiplicative function, then the function
    \[ g(n) = \sum_{d \mid n} f(d) \]
    is multiplicative.
\end{prop}
\begin{proof}
\end{proof}
    Let $p, q$ be primes. 
    \[ g(p^\alpha) = f(1) + f(p) + \ldots + f(p^\alpha) \]
    and
    \[ g(q^\beta) = f(1) + f(q) + \ldots + f(q^\beta) \]
    
    Then
    \begin{align*}
    g(p^\alpha q^\beta) 
        &= f(p^0 q^0) + f(p^1 q^0) + \dots + f(p^\alpha q^0) \\
        & \;\;\;\; + f(p^0 q^1) + f(p^1 q^1) + \dots + f(p^\alpha q^1) \\
        & \;\;\;\; + \dots \\
        & \;\;\;\; + f(p^0 q^\beta) + f(p^1 q^0) + \ldots + f(p^\alpha q^\beta) \\
        \\    
        %-----------------------------------------------------------------------
        &= f(p^0)f(q^0) + f(p^1)f(q^0) + \dots + f(p^\alpha)f(q^0) \\
        & \;\;\;\; + f(p^0)f(q^1) + f(p^1)f(q^1) + \dots + f(p^\alpha)f(q^1) \\
        & \;\;\;\; + \dots \\
        & \;\;\;\; + f(p^0)f(q^\beta) + f(p^1)f(q^\beta) + \dots + f(p^\alpha)f(q^\beta) \\
        \\
        %-----------------------------------------------------------------------     
        &= f(q^0)[f(p^0) + f(p^1) + \ldots + f(p^\alpha)] \\
        & \;\;\;\; + f(q^1)[f(p^0) + f(p^1) + \ldots + f(p^\alpha)] \\
        & \;\;\;\; + \ldots \\
        & \;\;\;\; + f(q^\beta)[f(p^0) + f(p^1) + \ldots + f(p^\alpha)] \\
        \\
        %-----------------------------------------------------------------------     
        &= [f(p^0) + f(p^1) + \ldots + f(p^\alpha)] \cdot [f(q^0) + f(q^1) + \ldots + f(q^\beta)] \\
        %-----------------------------------------------------------------------     
        &= f(p^\alpha)f(q^\beta)
    \end{align*}
    
    Since each number has a prime factorization, this applies for all integers.
\begin{prop}[part b]
    The function
    \[ F(n) = \sum_{d \mid n} \phi(d) = n \]
\end{prop}
\begin{proof}
    Let $p$ be prime. Then
    \begin{align*}
    F(p^k) &= \sum_{d \mid n} \phi(d) \\
           &= \phi(1) + \phi(p) + \phi(p^2) + \ldots + \phi(p^k) \\
           &= 1 + (p-1) + (p^2 - p) + \ldots + (p^{k-1} - p^{k-2}) + (p^k - p^{k-1}) \\
           &= (1-1) + (p-p) + (p^2 - p^2) \ldots + (p^{k-1} - p^{k-1}) + p^k \\
           &= p^k
    \end{align*}    

    Since $\phi$ is multiplicative, Proposition 6.2 says that $F$ is
    multiplicative. Therefore, $F(n) = n$ for all $n$.
\end{proof}
