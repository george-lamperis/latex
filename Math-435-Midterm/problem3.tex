% ------------------------------------------------------------------------------
% Problem 11.12
% ------------------------------------------------------------------------------

\begin{lemma}
    Let $n$ be an integers. Also let $p_1, p_2, \ldots, p_r$ be the prime
    factors of $n$ listed in ascending order, i.e. $p_1 < p_2 < \ldots < p_r$.
    
    If $\phi(n) = n/k$, for some positive integer $k$, then
    \[p_1 p_2 \ldots p_n = k(p_1 - 1)(p_2 - 1)(\ldots)(p_r - 1) \]
\end{lemma}
\begin{proof}
    Suppose $\phi(n) = n/k$. Then

    \begin{align*}
        \phi(n) &= n \left( 1 - \dfrac{1}{p_1} \right) \left( 1 - \dfrac{1}{p_2} \right) \dots \left( 1 - \dfrac{1}{p_r} \right) \\
        \dfrac{n}{k} &= n \left( 1 - \dfrac{1}{p_1} \right) \left( 1 - \dfrac{1}{p_2} \right) \dots \left( 1 - \dfrac{1}{p_r} \right) \\
        \dfrac{1}{k} &= \left( 1 - \dfrac{1}{p_1} \right) \left( 1 - \dfrac{1}{p_2} \right) \dots \left( 1 - \dfrac{1}{p_r} \right) \\
        \dfrac{1}{k} &= \dfrac{(p_1 - 1) \dots (p_r - 1)}{p_1 \dots p_r} \\
        p_1 p_2 \dots p_r &= k \cdot (p_1 - 1) \dots (p_r - 1) \\
    \end{align*}
\end{proof}

\begin{prop}[11.12a]
    An integer $n$ has the form $n=2^k$, $k \geq 1$ if and only if 
    $\phi(n) = n/2$.
\end{prop}
\begin{proof}
    Suppose $n=2^k$ with $k \geq 0$. Then $\phi(n) = 2^{k-1}$ and so we are
    done.
    
    Conversely, suppose $\phi(n)=n/2$. Let $p_1, p_2, \dots p_n$ be the distinct
    prime factors of $n$, listed in order i.e. $p_1 < p_2 < \cdots < p_r$.
    
    Then Lemma 3.1 says
    \[ p_1 p_2 \ldots p_r = 2 (p_1 - 1)(\ldots)(p_r - 1) \]
    Since 2 occurs on the right hand side, it must also be on the left hand
    side. Therefore, $p_1 = 2$. Now we have
    \begin{align*}
    2 \cdot p_2 \dots p_r &= 2 \cdot (2 - 1) \cdot (p_2 - 1) \dots (p_r - 1) \\
    2 \cdot p_2 \dots p_r &= 2 \cdot (p_2 - 1) \dots (p_r - 1) \\
    \end{align*}
    
    Notice the remaining unknowns on the right-hand side are all even, while
    the unknowns on the left-hand side are all odd. For this equality to hold,
    it must be the case that $r=1$ and $p_1 = 2$, i.e. $n=2^k$ with $k \geq 1$.
\end{proof}


\begin{prop}[11.12b]
    An integer $n$ has the form $n=2^{k_1} 3^{k_2}$, $k_i \geq 1$ if and only if 
    $\phi(n) = n/3$.
\end{prop}
\begin{proof}
    Suppose $n = 2^{k_1} 3^{k_2}$, $k_i \geq 1$. Then
    \begin{align*}
        \phi(n) &= \phi(2^{k_1} 3^{k_2}) \\
                &= \phi(2^{k_1}) \phi(3^{k_2}) \\
                &= (2^{k_1 - 1})(3^{k_2 - 1} \cdot 2) \\
                &= 2^{k_1} 3^{k_2 - 1} \\
                &= n/3
    \end{align*}
    
    Conversely, suppose $\phi(n) = n/3$. Then Lemma 3.1 says
    \[ p_1 p_2 \ldots p_r = 3 (p_1 - 1)(\ldots)(p_r - 1) \]
    
    Since 3 appears on the right-hand side, a 3 must appear on the left-hand 
    side also. This means for some $1 \leq i \leq r$, we have $p_i = 3$. Then
    the term $(p_i - 1) = 2$ occurs on the right-hand side, which means that
    a 2 must occur on the left-hand side also. Therefore, we have $p_1 = 2$ and
    $p_2 = 3$.
    
    Now we have
    \begin{align*}
    2 \cdot 3 \ldots p_r &= 3 (2 - 1)(3-1)(\ldots)(p_r - 1) \\
    2 \cdot 3 \ldots p_r &= 3 \cdot 2 \cdot (\ldots)(p_r - 1)
    \end{align*}
    Notice the remaining unknowns on the right-hand side are all even, while
    the unknowns on the left-hand side are all odd. For this equality to hold, 
    it must be the case that $r = 2$.
    
    Therefore, 2 and 3 are the only prime factors of $n$, i.e. 
    $n=2^{k_1} 3^{k_2}$.
\end{proof}


\begin{prop}[11.12c]
    There does not exist an $n$ such that $\phi(n) = n/6$.
\end{prop}
\begin{proof}
    Suppose $\phi(n) = n/6$. Then Lemma 3.1 says
    \begin{align*}
    p_1 p_2 \ldots p_r &= 6 (p_1 - 1)(\ldots)(p_r - 1) \\
                       &= 2 \cdot 3 (p_1 - 1)(\ldots)(p_r - 1) \\
    \end{align*}
    
    This implies that $p_1 = 2$ and $p_2 = 3$. Substituting these values, we
    have
    \begin{align*}
    2 \cdot 3 \ldots p_r &= 2 \cdot 3 (2 - 1)(3-1)(\ldots)(p_r - 1)\\
                         &= 2 \cdot 3 \cdot 2 \cdot (p_3 - 1)(\ldots)(p_r - 1)\\
    \end{align*} 
    
    As you can see, there are two 2's on the right hand side of the equation, 
    while there can only be one on the left (since every other prime is odd).
    And so we have a contradiction.
\end{proof}

