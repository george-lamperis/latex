% ------------------------------------------------------------------------------
% Problem 6.4
% ------------------------------------------------------------------------------

\begin{lemma}
    Given integers $a$ and $b$, $\gcd(a,b) = 1$ if and only if there are
    integers $x, y$ such that $ax + by = 1$.
\end{lemma}
\begin{proof}
    Suppose $\gcd(a,b) = 1$. Then by Theorem 6.1, there are integers $x,y$ such
    that $ax+by=1$ and so we are done.
    
    Conversely, suppose that there are integers $x,y$ such that $ax+by=1$. 
    
    Let the $\gcd(a,b)=g$. Since $g \mid a$ and $g \mid b$, it follows that
    \begin{align*}
        g & \mid ax+by \\
        g & \mid 1
    \end{align*}
    But the only divisor of 1 is 1. Hence, $g=1$.
\end{proof}
    
\begin{lemma}
    If $\gcd(a,b,c) = g$, then there are integers $x,y,z$ such that 
    $ax + by + cz = g$
\end{lemma}
\begin{proof}
    Let $d=\gcd(a,b)$. Then there are integers $x,y$ such that $ax+by=d$.
    
    Now let $g=gcd(d,c)$. There are integers $w,z$ such that $dw+cz=g$. Then:
    \begin{align*}
        g &= dw + cz \\
          &= (ax+by)w + cz \\
          &= axw + byw + cz
    \end{align*}
\end{proof}

\begin{prop}
    Let $a,b,c$ be integers. Then $gcd(a,b,c) = 1$ if and only if there are
    integers $x,y,z$ such that $ax+by+cz=1$.
\end{prop}
\begin{proof}
    Suppose $gcd(a,b,c) = 1$. Then the previous lemma says we are done.
    
    Conversely, suppose there are $x,y,z$ such that $ax+by+cz=1$. Since 
    $g \mid a$, $g \mid b$, and $g \mid c$ it follows that
    \begin{align*}
        g & \mid ax+by+cz \\
        g & \mid 1
    \end{align*}
    But the only divisor of 1 is 1. Hence $g=1$.
\end{proof}

Let $a, b$ and $c$ be integers with $\gcd(a,b,c) = 1$. In general, to find $x,
y$ and $z$ such that $ax + by + cz = 1$:

First calculate $d = \gcd(a,b)$ using the Euclidian Algorithm. Then, using your
computations, find $r$ and $s$ such that 
\[ ar + bs = d \]

Next calcuate $1 = \gcd(d, c)$ using the Euclidian Algorithm. Find $w$ and
$z$ such that
\[dw + cz = 1\]

Now subsitute $d = ar + bs$:
\begin{align*}
    (ar + bs)w + cz &= 1  \\
    arw + bsw + cz  &= 1
\end{align*}

Now we have $x = rw$, $y = sw$ and $z$ such that $ax + by + cz = 1$.

\begin{example}
    Find integers $x, y$ and $z$ that satisfy the equation
    \[ 6x + 15y + 20z = 1\]
    
    First calculate $gcd(6,15) = 3$ using the Euclidian Algorithm.
    \begin{table}[H]
    \centering
    \begin{tabular}{c|c|c|c}
            & 2  & 2  &    \\ \hline
        15  & 6  & 3  & 0  \\ \hline
        12  & 6  &    &
    \end{tabular}
    \end{table}

    Using the above table, we back substitute to write $3 = 15(1) + 2(-6)$.
    
    Now calculate $gcd(3,20) = 1$
    \begin{table}[H]
    \centering 
    \begin{tabular}{c|c|c|c|c}
            & 6  & 1  & 2  &    \\ \hline
        20  & 3  & 2  & 1  & 0  \\ \hline
        18  & 2  & 2  &    &
    \end{tabular}
    \end{table}
    
    Back substitute to write 
    \begin{align*}
        1 &= 3 \cdot 7 + 20 \cdot (-1) \\
          &= (6 \cdot (-2) + 15 \cdot 1)7 + 20 \cdot (-1) \\
          &= 6 \cdot (-14) + 15 \cdot 7 + 20 \cdot (-1)
    \end{align*}
    
    And so we have $x = -14$, $y = 7$, and $z=-1$.
\end{example}


\begin{example}
    Find integers $x, y$ and $z$ that satisfy the equation
    \[155x + 341y + 385z = 1 \]
    
    First calculate $gcd(155, 341) = 31$
    \begin{table}[H]
    \centering
    \begin{tabular}{c|c|c|c}
            & 2   & 5  &   \\ \hline
        341 & 155 & 31 & 0 \\ \hline
        310 & 155 &    & 
    \end{tabular}
    \end{table}
    
    Back substitute to write $155(-2) + 341(1)= 31$.
    
    Now calculate $gcd(31, 385) = 1$.
     \begin{table}[H]
    \centering
    \begin{tabular}{c|c|c|c|c|c|c}
            & 12  & 2   & 2  & 1  & 1  &   \\ \hline
        385 & 31  & 13  & 5  & 3  & 2  & 1 \\ \hline
        372 & 25  & 10  & 3  & 2  &    & 
    \end{tabular}
    \end{table}
    
    Then write $1 =  31 \cdot (-149) + 385 \cdot 12 $.
    
    Now substitute $31 = 155 \cdot (-2) + 341 \cdot 1$:
    \begin{align*}
        1 &= 31 \cdot (-149) + 385 \cdot 12 \\
          &= (155 \cdot (-2) + 341 \cdot 1) (-149) + 385 \cdot 12 \\
          &= 155 \cdot 298 + 341 \cdot (-149) + 385 \cdot 12
    \end{align*} 
    
    And so we have $x = 298$, $y = -149$ and $z = 12$.
    
\end{example}