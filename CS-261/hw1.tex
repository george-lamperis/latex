\RequirePackage[l2tabu, orthodox]{nag}

\documentclass[letterpaper, 11pt]{article}

\usepackage[margin=1in]{geometry}
\usepackage{amsmath,amsthm,amssymb,mathtools}
\usepackage[parfill]{parskip}   % paragraph formatting
\usepackage{paralist}           % compactitem
\usepackage{array}              % required for tabular environment

% ------------------------------------------------------------------------------
% Theorem Commands
% ------------------------------------------------------------------------------
\newtheoremstyle{mystyle}% name
    {2ex}       % Space above
    {4ex}       % Space below
    {}          % Body font
    {}          % Indent amount (empty = no indent, \parindent = para indent)
    {\bfseries} % Thm head font
    {.}         % Punctuation after thm head
    {\newline}  % Space after thm head: \newline = linebreak
    {}          % Thm head spec

\theoremstyle{mystyle}
\newtheorem{theorem}{Theorem}[section] % master theorem
\newtheorem{example}[theorem]{Example}

% Formatting options
% \setlength{\tabcolsep}{20pt}
\newcolumntype{M} {
    >{\ttfamily\arraybackslash} 
    l
    <{}
}



% ------------------------------------------------------------------------------
% Begin Document
% ------------------------------------------------------------------------------
\title{CS 261 - Homework 1}
\author{George Lamperis}
\date{UIN:664714097}

\begin{document}
\maketitle

% ------------------------------------------------------------------------------
% Problem 1
% ------------------------------------------------------------------------------
\section{}
Q1. Consider the following 8 bit floating point representation based on the IEEE floating point
format.

Format A:
\begin{compactitem}
    \item There is one sign bit.
    \item There are $k=3$ exponent bits.
    \item There are $n=4$ fraction bits.
    \item bias $= 2^{3-1}-1 = 3 $.
\end{compactitem}

Format B:
\begin{compactitem}
    \item There is one sign bit.
    \item There are $k=4$ exponent bits.
    \item There are $n=3$ fraction bits.
    \item bias $= 2^{4-1}-1 = 7 $.

\end{compactitem}
Below, you are given some bit patterns of pattern A. Your task is to find out the values of numbers given

Check these well, especially denormalized/normalized.


\begin{tabular}{M|M|M}
    Format A    & Decimal Value     & Format B    \\ \hline
    1 010 1000  &  -3/4             & 1 0110 100  \\ 
    1 110 0000  &  -8               & 1 1010 000  \\ 
    0 101 1010  &  6.5              & 1 1001 101  \\
    0 000 1001  &  9/64             & 0 0100 001  \\
\end{tabular}


% ------------------------------------------------------------------------------
% Problem 2
% ------------------------------------------------------------------------------
\section{}
Format:
\begin{compactitem}
    \item non-negative (no sign bit)
    \item There are $k=4$ exponent bits.
    \item There are $n=4$ fraction bits.
    \item bias $= 2^{4-1}-1 = 7 $.
\end{compactitem}

\begin{tabular}{M|M|M|M}
    description                     & exponent  & fraction  & Single precision value \\ \hline
    Smallest denormalized value     & 0000      & 0001      &                        \\
    Largest denormalized value      & 0000      & 1111      &                        \\
    Smallest normalized value       & 0001      & 0000      &                        \\
    One                             & 0111      & 0000      & $1*2^0$                \\
    Largest normalized value        & 111       & 1111      &                        \\
\end{tabular}

\end{document}
