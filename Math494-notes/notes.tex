\documentclass[12pt]{article}

\usepackage{amsmath, amsthm, amssymb,amscd}
\usepackage{enumerate}

\raggedright

% ------------------------------------------------------------------------------
% Theorem Commands
% ------------------------------------------------------------------------------
\newtheoremstyle{mystyle}% name
    {2ex}       % Space above
    {2ex}       % Space below
    {}          % Body font
    {}          % Indent amount (empty = no indent, \parindent = para indent)
    {\bfseries} % Thm head font
    {.}         % Punctuation after thm head
    {\newline}  % Space after thm head: \newline = linebreak
    {}          % Thm head spec

\theoremstyle{mystyle}

\newtheorem{thm}{Theorem}
\newtheorem{prop}[thm]{Proposition}
\newtheorem{defn}[thm]{Definition}
\newtheorem{lemma}[thm]{Lemma}
\newtheorem{claim}[thm]{Claim}
\newtheorem{example}[thm]{Example}

% Number sets
\newcommand{\N}{\mathbb{N}}
\newcommand{\Z}{\mathbb{Z}}
\newcommand{\Q}{\mathbb{Q}}
\newcommand{\R}{\mathbb{R}}
\newcommand{\C}{\mathbb{C}}


\newcommand{\Kn}{k[x_1, \ldots, x_n]}

\newcommand{\A}[2]{\mathbb{A}^{#1}_{#2}}
\newcommand{\Ank}{\A{n}{k}}


% ------------------------------------------------------------------------------
% Begin Document
% ------------------------------------------------------------------------------
\title{Math 494 - Notes}
\author{George Lamperis}
\date{}

\begin{document}
\maketitle

\section{Background}


\section{}

\begin{defn}
  A polynomial over a field $k$ has the form
  \[ a_1 x_1 + a_2 x_2 + \ldots + a_n x_n^{i_n}\]
  where $a_i \in k$.
\end{defn}

\begin{defn}
  The r
\end{defn}

\begin{defn}
  Affine $n$-space over a field $k$, denoted $\A{n}{k}$, is the space of
  $n$-tuples
  \[ \Ank = \{(a_1, \ldots, a_n) \mid a_i \in k \}\]
\end{defn}

\begin{defn}[Zero set]
  Let $f \in \Kn$. The zero set of $f$ is the set
  \[ V(f) = \{ (a_1, \ldots, a_n) \in \Ank \mid f(a_1, \ldots, a_n) = 0 \}\]
  
  Note $V(f) \subset \Ank$.
\end{defn}

\begin{defn}[Zarski Topology]
Let $X \subset \Ank$. Then let
\[ I(X) = \{ f \in k[x_1, \ldots, x_n] \mid f(x) = 0, \forall x \in X \} \]

Note $I(X) \subset \Kn$.
\end{defn}

\begin{prop}
$I(X)$ is an ideal.
\end{prop}

\end{document}
