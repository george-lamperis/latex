\section{Rings}

\begin{defn}[Ring axioms]
  A ring $(R, +, \cdot)$ is a set $R$ together with two binary operations, $+$
  and $\cdot$, which we call addition and multiplication, defined on $R$ such
  that the following is satisfied.
  
  \begin{enumerate}[(i)]
    \item $(R,+)$ is an abelian group, meaning:
    \begin{enumerate}[(a)]
      \item Addition is associative.
      \item There is an additive identity $0 \in R$.
      \item For each $a \in R$, there is an additive inverse $-a \in R$.
      \item Addition is commutative.
    \end{enumerate}
    
    \item Multiplication is associative. 
    \item There is a multiplicative identity $1 \in R$.
    \item Multiplication distributes over addition.
  \end{enumerate}
  
  If multiplication is also commutative, then we say $R$ is a commutative ring.
\end{defn}


\begin{defn}[Division Ring]
  A ring $R$ with identity $1$, where $1 \neq 0$ is called a division ring
  if every nonzero element $a \in R$ has a multiplicative inverse, i.e., there
  exists $b \in R$ such that $ab=ba=1$.
  
  A commutative division ring is called a field. 
\end{defn}


\begin{prop}
  Some properties of rings here?
\end{prop}


\begin{defn}[Zero divisor]
  A nonzero element $a \in R$ is called a zero divisor if there is a nonzero
  element $b \in R$ such that either $ab=0$ or $ba=0$.
\end{defn}


\begin{defn}[Unit]
  Let $R$ be a ring with $1 \neq 0$. An element $u \in R$ is called a unit if
  there is some $v \in R$ such that $uv=vu=1$. The set of units in $R$ is
  denoted by $R^\times$.
\end{defn}


\begin{defn}[Integral Domain]
  A commutative ring with identity $1 \neq 0$ is called an integral domain if it
  has no zero divisors, i.e if $ab=0$, then $a=0$, or $b=0$.
\end{defn}


\begin{defn}[Subring]
  A subring of the ring $R$ is a subgroup of $R$ that is closed under
  multiplication.
\end{defn}


\begin{defn}[Ring Homomorphism]
  Let $R$ and $R$ be rings. 
  \begin{enumerate}
    \item A ring homomorphism is a map $\phi: R \rightarrow S$ that satisfies
    the following:
    \begin{enumerate}[(i)]
      \item $\phi(a+b) = \phi(a) + \phi(b)$
      \item $\phi(ab) = \phi(a) \phi(b)$
    \end{enumerate}
    
    \item The kernel of the ring homomorphism $\phi$, denoted $ker\phi$, is the
    set of elements of $R$ that map to $0 \in S$.
    \item A bijective ring homomorphism is called an isomorphism.
  \end{enumerate}  
\end{defn}


\begin{prop}
  Let $R$ and $S$ be rings and let $\phi: R \rightarrow S$ be a homomorphism.
  \begin{enumerate}
    \item The image of $\phi$ is a subring of $S$.
    \item The kernel of $\phi$ is a subring of $R$. Furthermore, if 
    $a \in ker\phi$, then $ra \in ker\phi$ and $ar \in ker\phi$. In other words,
    $ker\phi$ is closed under multiplication by elements from $R$.
  \end{enumerate}
\end{prop}


\begin{defn}[Ideal]
  Let $R$ be a ring, let $I$ be a subset of $R$, and let $r \in R$.
  \begin{enumerate}
    \item $rI = \{ra \mid a \in I \}$ and $Ir = \{ar \mid a \in I \}$.
    \item A subset $I$ or $R$ is a left ideal of $R$ if
      \begin{enumerate}[(i)]
        \item $I$ is a subring of $R$.
        \item $I$ is closed under left multiplication by elements from $R$,
        i.e., $rI \subseteq I$ for all $r \in R$.
      \end{enumerate}
    We call $I$ a right ideal if (ii) holds for right multiplication.  
    \item A subset $I$ that is both a left ideal and a right ideal is called an
    ideal, or for added emphasis, a two sided ideal.
  \end{enumerate}
\end{defn}


\begin{remark}
  The kernel of any ring homomorphism is an ideal.
\end{remark}


\begin{remark}
  In this course, we adopt the convention that ``ring'' means ``commutative
  ring''.
\end{remark}


\begin{defn}[Ideal in a commutative ring]
  Let $(R,+,\cdot)$ be a commutative ring and let $(R, +)$ be its additive
  group. A subset $I$ is called an ideal if it satisfies:
  \begin{itemize}
    \item $I$ is nonempty and $\forall x,y \in I : x-y \in I$, i.e. $(I,+)$ is a
    subgroup of $(I,+)$.
    \item $\forall r \in R, i \in I : ri \in I$.
  \end{itemize}
\end{defn}


\begin{defn}
  A polynomial over a field $k$ has the form
  \[ a_1 x_1 + a_2 x_2 + \ldots + a_n x_n^{i_n}\]
  where $a_i \in k$.
\end{defn}

\begin{defn}[prime ideal]
  The r
\end{defn}