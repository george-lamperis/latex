\documentclass[12pt]{article}

\usepackage{amsmath, amsthm, amssymb,amscd}
\usepackage{setspace}
\usepackage{enumerate}
\usepackage[margin=1in]{geometry}


\newtheoremstyle{mystyle}% name
    {2ex}       % Space above
    {2ex}       % Space below
    {}          % Body font
    {}          % Indent amount (empty = no indent, \parindent = para indent)
    {\bfseries} % Thm head font
    {.}         % Punctuation after thm head
    {\newline}  % Space after thm head: \newline = linebreak
    {}          % Thm head spec

\theoremstyle{mystyle}

\newtheorem{thm}{Theorem}
\newtheorem{defn}[thm]{Definition}
\newtheorem{prop}[thm]{Proposition}

\pagestyle{empty}
\doublespace



% ------------------------------------------------------------------------------
% Begin Document
% ------------------------------------------------------------------------------
\title{Math 300 - Proof 3}
\author{George Lamperis}
\date{}

\begin{document}
\maketitle

\begin{prop}
  Let $S$ be a partially ordered set, with the additional property that every
  chain $s_0 \leq s_1 \leq s_2 \leq \ldots$ has an upper bound in $S$. Suppose
  $C$ is a countably infinite subset of $S$ such that for every $u,v \in C$
  there is a $w \in C$ such that $u \leq w$ and $v \leq w$. Prove that $C$ has 
  an upper bound in $S$.
\end{prop}
\begin{proof}
  We will construct an ascending chain of elements in $C$. Let
  \begin{align*}
    c_0 &= min(u,v) \\
    c_1 &= max(u,v)
  \end{align*}
  It follows that there is a $c_2 \in C$ such that $c_0 \leq c_2$ and 
  $c_1 \leq c_2$. This process may be repeated, so that for each $c_{i-1}$
  and $c_{i-2}$, we have a $c_i \in C$ such that $c_{i-1} \leq c_i$ and
  $c_{i-2} \leq c_i$. Thus, we have the ascending chain
  $$ c_0 \leq c_1 \leq c_2 \leq \ldots$$
  
  We know that every ascending chain in $S$ has an upper bound in $S$, and so we
  are done.
\end{proof}

\end{document}
