\documentclass[12pt]{article}

\usepackage{amsmath, amsthm, amssymb,amscd}
\usepackage{setspace}
\usepackage{enumerate}
\usepackage[margin=1in]{geometry}


\newtheoremstyle{mystyle}% name
    {2ex}       % Space above
    {2ex}       % Space below
    {}          % Body font
    {}          % Indent amount (empty = no indent, \parindent = para indent)
    {\bfseries} % Thm head font
    {.}         % Punctuation after thm head
    {\newline}  % Space after thm head: \newline = linebreak
    {}          % Thm head spec

\theoremstyle{mystyle}

\newtheorem{thm}{Theorem}
\newtheorem{defn}[thm]{Definition}
\newtheorem{prop}[thm]{Proposition}
\newtheorem{remark}[thm]{Remark}
\newtheorem{example}[thm]{Example}
\doublespace

% ------------------------------------------------------------------------------
% Begin Document
% ------------------------------------------------------------------------------
\title{Math 300 - Proof 6}
\author{George Lamperis}

\date{}

\begin{document}
\maketitle

\begin{prop}
  There are infinitely many lines that pass through the origin and are
  also tangent to the curve $y = \sin x$.
\end{prop}
\begin{proof}

(x,sin(x)) is such a tangency point iff its slope sin(x)/x = the derivative of sin at x, namely cos(x), so the condition is x = sin(x)/cos(x)  which means: x is a fixed-point of tan
<Z-module> if you look at the graph of tan, you can see it intersects the line y = x infinitely often

\end{proof}

\end{document}