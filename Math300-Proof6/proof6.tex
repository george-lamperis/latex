\documentclass[12pt]{article}

\usepackage{amsmath, amsthm, amssymb,amscd}
\usepackage{setspace}
\usepackage{enumerate}
\usepackage[margin=1in]{geometry}


\newtheoremstyle{mystyle}% name
    {2ex}       % Space above
    {2ex}       % Space below
    {}          % Body font
    {}          % Indent amount (empty = no indent, \parindent = para indent)
    {\bfseries} % Thm head font
    {.}         % Punctuation after thm head
    {\newline}  % Space after thm head: \newline = linebreak
    {}          % Thm head spec

\theoremstyle{mystyle}

\newtheorem{thm}{Theorem}
\newtheorem{defn}[thm]{Definition}
\newtheorem{prop}[thm]{Proposition}
\newtheorem{remark}[thm]{Remark}
\newtheorem{example}[thm]{Example}
\doublespace

% ------------------------------------------------------------------------------
% Begin Document
% ------------------------------------------------------------------------------
\title{Math 300 - Proof 6}
\author{George Lamperis}

\date{}

\begin{document}
\maketitle

\begin{prop}
  There are infinitely many lines that pass through the origin and are
  also tangent to the curve $y = \sin x$.
\end{prop}
\begin{proof}
  Recall that a line passing through the points $(0,0)$ and $(x,y)$ is
  described by the equation $y=mx$ and has slope $m$.
  
  We are looking for points $(x, \sin(x))$ such that:
  $$
    \frac{\sin x}{x} = \sin(x)^\prime 
  $$
  Given such a point, we have:
  \begin{align*}
    \frac{\sin x}{x} = \sin(x)^\prime 
    &\implies \frac{\sin x}{x} = \cos x \\
    &\implies x = \tan x
  \end{align*}

  Notice that $\tan x$ has infinitely many fixed points, i.e. points such that 
  $x = \tan x$. You can confirm this by graphing $f(x) = x - \tan x$ and
  noticing that this function has infinitely many zeros.
  
  Then it follows that there are infinitely many lines that pass
  through the origin and are also tangent to the curve $y = \sin x$.
\end{proof}

\end{document}