\documentclass[12pt, letterpaper]{article}
 
\usepackage[margin=1in]{geometry} 
\usepackage{amsmath,amsthm,amssymb}
\usepackage{enumitem}

% ------------------------------------------------------------------------------
% Math commands
% ------------------------------------------------------------------------------
\newcommand{\N}{\mathbb{N}}
\newcommand{\Z}{\mathbb{Z}}
\newcommand{\Q}{\mathbb{Q}}
\newcommand{\R}{\mathbb{R}}

\renewcommand{\epsilon}{\varepsilon}


\newcommand{\seq}[2]{(#1)_{#2 \geq 1}}

\newcommand{\seqxn}{(x_n)_{n\geq1}}
\newcommand{\seqyn}{(y_n)_{n\geq1}}
% ------------------------------------------------------------------------------
% Document formatting
% ------------------------------------------------------------------------------
\setlength{\parskip}{2ex} %--skip lines between paragraphs
\linespread{1.2}

% ------------------------------------------------------------------------------
% Document Begins Here!
%-------------------------------------------------------------------------------
\begin{document}

\title{Math 313 - Problem Set 5\vspace{-2ex}}%replace X with the appropriate number
\author{George Lamperis - UIN:664714097} %if necessary, replace with your course title
\date{\vspace{-2ex}}
\maketitle


\begin{enumerate}[label=\bfseries1\alph*)]

% ------------------------------------------------------------------------------
% 1a)
% ------------------------------------------------------------------------------
\item Let $(n_k)_{k \geq 1}$ be a strictly monotone increasing sequence of 
natural numbers. Then $n_k \geq k$ for every $k$.
\begin{proof} 
	We will induct on $k$:

	For every $m \in \N$, we have $m \geq 1$. Hence, $n_1 \geq 1$.

	Suppose for inductive hypothesis that $n_k \geq k$ for some natural 
	number $k$. Because $(n_k)_{k \geq 1}$ is strictly monotone increasing, 
	we have $n_k < n_{k+1}$. This implies $n_k + 1 \leq n_{k+1}$.

	\begin{align*}
		k \leq n_k 	&\implies k+1 \leq n_k + 1 \leq n_{k+1} \\
					&\implies k+1 \leq n_{k+1}
	\end{align*}

	Hence, by induction, $n_k \geq k$ for every $k$.
\end{proof}

% ------------------------------------------------------------------------------
% 1b)
% ------------------------------------------------------------------------------
\item Let $(x_n)_{n \geq 1}$ be a sequence of real numbers and suppose that 
$(x_n)_{n \geq 1}$ is a null sequence. Let $(n_k)_{k \geq 1}$ be a strictly 
monotone increasing sequence of natural numbers. Then every subsequence 
$(x_{n_k})_{k \geq 1}$ is a null sequence.

\begin{proof}
	Let $\epsilon > 0$ be given. 

	Since $(x_n)_{n \geq 1}$ is a null sequence, there is an $N$ such that for all
	$n \geq N$, we have $|x_n| < \epsilon$. 

	Because $\seq{n}{k}$ is strictly monotone increasing, by (1a) we have that 
	$k \leq n_k$ for every $k$. Then for every $k \geq N$, we have $N \leq k \leq n_k$.

	Then for every $k \geq N$, we have $|x_{n_k}| < \epsilon$, i.e. 
	$(x_{n_k})_{k \geq 1}$ is a null sequence.

	Hence, every subsequence of a null sequence is a null sequence.
\end{proof}

% ------------------------------------------------------------------------------
% 1c)
% ------------------------------------------------------------------------------
\item Let $\seq{x}{n}$ be a sequence of real numbers converging to a limit
$L$. Then every subsequence of $\seq{x}{n}$ converges to $L$

\begin{proof}
	The sequence $\seq{x}{n}$ converges to $L$, and so by definition, $\seq{x_n -L}{n}$
	is a null sequence. 

	Let $(n_k)_{k \geq 1}$ be a strictly monotone increasing sequence of natural 
	numbers.

	By (1b), we have that $(x_{n_k}-L)_{k \geq 1}$ is a null sequence. By definition, 
	$(x_{n_k}-L)_{k \geq 1}$ is a null sequence if and only if $(x_{n_k})_{k \geq 1}$
	converges to $L$.

	Hence, every subsequence of $\seq{x}{n}$ converges to $L$.
\end{proof}

% ------------------------------------------------------------------------------
% 1d)
% ------------------------------------------------------------------------------
\item Let $\seq{x}{n}$ be a sequence of real numbers. If $(x_{2k})_{k\geq1}$ and
$(x_{2k-1})_{k\geq1}$ are null sequences, then $\seq{x}{n}$ is a null sequence.

\begin{proof}
	The sequences $\seq{2k}{k}$ and $\seq{2k-1}{k}$ are strictly monotone increasing, 
	since $2k-1 < 2(k+1) -1$ and $2k < 2(k+1)$.

	Suppose that $(x_{2k})_{k\geq1}$ and $(x_{2k-1})_{k\geq1}$ are null sequences.
	
	Then by definition, for every $\epsilon > 0$, there is a $K_1$ such that
	for all $k \geq K_1$, we have $|x_{2k}| < \epsilon$.

	Similarly, for every $\epsilon > 0$, there is a $K_2$ such that
	for all $k \geq K_2$, we have $|x_{2k-1}| < \epsilon$.

	Set $K = max(K_1, K_2)$ and let $\epsilon > 0$ be given. Then for every 
	$k \geq K$, we have $|x_{2k}| < \epsilon$ and $|x_{2k-1}| < \epsilon$.

	Now set $N=2K$. Then for every $n \geq N$, we have $|x_n| < \epsilon$. 

	Hence, if $(x_{2k})_{k\geq1}$ and $(x_{2k-1})_{k\geq1}$ are null sequences,
	$\seq{x}{n}$ is a null sequence. 
\end{proof}

% ------------------------------------------------------------------------------
% 1e) 
% ------------------------------------------------------------------------------
\item Let $\seq{x}{n}$ be a sequence of real numbers. If $(x_{2k})_{k\geq1}$ and
$(x_{2k-1})_{k\geq1}$ converge to $L$, then $\seq{x}{n}$ also converges to $L$.

\begin{proof}
	Suppose $(x_{2k})_{k\geq1}$ and $(x_{2k-1})_{k\geq1}$ converge to $L$. 

	Then by definition, $(x_{2k} - L)_{k\geq1}$ and $(x_{2k-1} - L)_{k\geq1}$ 
	are null sequences.

	Because these are null sequences, by (1d), we have that $(x_n -L)_{k\geq }$
	is also a null sequence. Then by definition, we have that $\seq{x}{n}$ converges to $L$.

	Hence, if $(x_{2k})_{k\geq1}$ and $(x_{2k-1})_{k\geq1}$ converge to $L$, then 
	$\seq{x}{n}$ also converges to $L$.
\end{proof}
\end{enumerate}


% Problem 2 --------------------------------------------------------------------
\begin{enumerate}[label=\bfseries2\alph*)]

% ------------------------------------------------------------------------------
% 2a) 
% ------------------------------------------------------------------------------
\item Let $\sum_{i=1}^{\infty} u_i$ be a series of real numbers. Assume 
$\seq{u}{i}$ is a null sequence, and that the series $\sum_{i=1}^{\infty} (u_{2j-1} + u_{2j})$
converges to a sum $S$.

Then $\sum_{i=1}^{\infty} u_i$ converges to $S$.

\begin{proof}
	% Let \[T_n = \sum_{i=1}^{n} u_i = u_1 + u_2 + ... + u_n \].
	
	% Then $(T_n)_{n\geq1}$ is the sequence of partial sums for $\sum_{i=1}^{\infty} u_i$
	
	% Then the subsequence $(T_{2n})_{n\geq1}$

\end{proof}

% ------------------------------------------------------------------------------
% 2b) 
% ------------------------------------------------------------------------------
\item The series \[\sum_{i=1}^{\infty} \frac{(-1)^{i-1}}{i} = 1 - \frac{1}{2} 
+ \frac{1}{3} - \frac{1}{4} + ...\] converges. Also, the series does not converge
absolutely.

\begin{proof}
	The sequence \[ (x_i)_{i\geq1} = \left(\frac{(-1)^{i-1}}{i}\right)_{i\geq1}\] is a null sequence.

	Define two subsequences \[x_{2j-1} = \left(\frac{1}{2j-1}\right)_{j\geq1}\]
	and \[x_{2j} = \left(-\frac{1}{2j}\right)_{j\geq1}\]

	The sequence $\sum_{j=1}^{\infty} (\frac{1}{2})^j$ converges, and for all $j$,
	\[x_{2j-1} - x_{2j} = \frac{1}{(2j)(2j-1)} \leq \left(\frac{1}{2}\right)^j\].

	Then by the Comparison Test, $\sum_{j=1}^{\infty}(x_{2j-1} - x_{2j})$
	converges, say, to some number $S$. 

	Because $(x_i)_{i\geq1}$ is a null sequence and $\sum_{j=1}^{\infty}(x_{2j-1} - x_{2j})$ 
	converges to $S$, then by (4a), $\sum_{i=1}^{\infty} x_i$ also converges
	to $S$.

	Hence, the series \[\sum_{i=1}^{\infty} \frac{(-1)^{i-1}}{i} \] converges.

	However, \[\sum_{i=1}^{\infty} \frac{(-1)^{i-1}}{i} = \sum_{i=1}^{\infty} \frac{1}{i}\]
	is the Harmonic series, which diverges. Hence, the series
	\[\sum_{i=1}^{\infty} \frac{(-1)^{i-1}}{i} \] is not absolutely convergent.
\end{proof}
\end{enumerate}


% Problem 3 --------------------------------------------------------------------
\begin{enumerate}[label=\bfseries3\alph*)]

% ------------------------------------------------------------------------------
% 3a) 
% ------------------------------------------------------------------------------
\item placeholder

% ------------------------------------------------------------------------------
% 3b) 
% ------------------------------------------------------------------------------
\item placeholder

% ------------------------------------------------------------------------------
% 3c) 
% ------------------------------------------------------------------------------
\item placeholder

\end{enumerate}


% % Problem 4 --------------------------------------------------------------------
\begin{enumerate}[label=\bfseries4\alph*)]

% ------------------------------------------------------------------------------
% 4a) 
% ------------------------------------------------------------------------------
\item Let $\seq{x}{n}$ be a monotone increasing sequence of positive real numbers.
Assume that some subsequence $\aseq{x}{n_k}{k}$ is a null sequence.

Then $\seq{x}{n}$ is a null sequence.

\begin{proof}
	$\aseq{x}{n_k}{k}$ is a null sequence, and so for every $\epsilon > 0$, there
	is a $K$ such that for all $k \geq K$, we have ${|x_{n_k}| < \epsilon}$.

	But for every $k$, there is an $N$ such that $N = n_k$. 

	Hence, for every $\epsilon > 0$, there is an $N = n_k$ such that for all 
	$n \geq N$, we have ${|x_n| < \epsilon}$.
\end{proof}

% ------------------------------------------------------------------------------
% 4b) 
% ------------------------------------------------------------------------------
\item Show that $({1/\sqrt{n}})_{n\geq1}$ is a null sequence.

\begin{proof}
	
	The sequence $({1/\sqrt{n}})_{n\geq1}$ is strictly monotone decreasing because
	$1/\sqrt{n} > 1/\sqrt{n+1}$. 

	The sequence $(k^2)_{k\geq1}$ is strictly monotone increasing because 
	$n^2 < (n+1)^2$. Also, all of its terms are natural numbers.

	Then $(1/\sqrt{k^2})_{k\geq1} = (1/k)_{k\geq1}$ is a subsequence of 
	$({1/\sqrt{n}})_{n\geq1}$. It was shown in class that $(1/k)_{k\geq1}$ is a 
	null sequence.

	Now we can apply (4a), and we have that $({1/\sqrt{n}})_{n\geq1}$ is a null sequence.
\end{proof}

% ------------------------------------------------------------------------------
% 4c) 
% ------------------------------------------------------------------------------
\item Let $\seq{x}{n}$ and $\seq{y}{n}$ be sequences of real numbers. Assume that
$\seq{x}{n}$ is a null sequence and that $|y_n| \leq |x_n|$ for every $n$. 

Then $\seq{y}{n}$ is a null sequence.

\begin{proof}
	Since $\seq{x}{n}$ is a null sequence, for all $\epsilon > 0$, there is an
	$N$ such that for all $n \geq N$, $|x_n| < \epsilon$. 

	For all $n$, $|y_n| \leq |x_n|$. Then for all $\epsilon > 0$, there is 
	an $N$ such that for all $n \geq N$, $|y_n| \leq |x_n| < \epsilon$.

	Hence, $\seq{y}{n}$ is a null sequence.
\end{proof}

% ------------------------------------------------------------------------------
% 4d) 
% ------------------------------------------------------------------------------
\item Prove that $\nseq{\sqrt{n+1} - \sqrt{n}}{n}$ is a null sequence.

\begin{proof}
	As shown in (4b), $({1/\sqrt{n}})_{n\geq1}$ is a null sequence. 

	For all $n$, we have \[\frac{1}{\sqrt{n+1}} < \frac{1}{\sqrt{n}}\]
	Then by (4c), we have that $\left(1/\sqrt{n+1}\right)_{n\geq1}$ is a null sequence. 

	The sum of two null sequences is also a null sequence. Hence:
	\[ \left(\frac{1}{\sqrt{n+1} + \sqrt{n}}\right)_{n \geq 1}\]
	is a null sequence. 

	Notice that $(\sqrt{n+1} - \sqrt{n}) (\sqrt{n+1} + \sqrt{n}) = 1$. 
	And so, \[ \frac{1}{\sqrt{n+1} + \sqrt{n}} = \sqrt{n+1} - \sqrt{n} \]

	Therefore, $\nseq{\sqrt{n+1} - \sqrt{n}}{n}$ is a null sequence.
\end{proof}

% ------------------------------------------------------------------------------
% 4e) 
% ------------------------------------------------------------------------------
\item The series \[\sum_{i=1}^{\infty} \sqrt{i+1} - \sqrt{i} \]
is divergent.

\begin{proof}
	For every $n$, $1 \leq \sqrt{n}$ and $1 < \sqrt{2} \leq \sqrt{n+1}$. And so, 
	$2 < \sqrt{n} + \sqrt{n+1}$ for every $n$.

	The sequence $(2)_{n \geq 1}$ is not a null sequence (Take $\epsilon < 2$).
	As shown in class, this is sufficent to prove that the series $\sum_{i=1}^{\infty} 2$
	does not converge.

	Since $2 < \sqrt{n} + \sqrt{n+1}$ for every $n$ and the series 
	$\sum_{i=1}^{\infty} 2$ does not converge, then by the Comparison Test,
	\[\sum_{i=1}^{\infty} \sqrt{i+1} - \sqrt{i} \] does not converge.


\end{proof}
\end{enumerate}
% ------------------------------------------------------------------------------
% Don't edit below this!
% ------------------------------------------------------------------------------
\end{document}
