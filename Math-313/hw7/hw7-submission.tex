\documentclass[12pt, letterpaper]{article}
 
\usepackage[margin=1in]{geometry} 
\usepackage{amsmath,amsthm,amssymb}
\usepackage{enumitem}

% ------------------------------------------------------------------------------
% Math commands
% ------------------------------------------------------------------------------
\newcommand{\N}{\mathbb{N}}
\newcommand{\Z}{\mathbb{Z}}
\newcommand{\Q}{\mathbb{Q}}
\newcommand{\R}{\mathbb{R}}

\renewcommand{\epsilon}{\varepsilon}


\newcommand{\seq}[2]{(#1)_{#2 \geq 1}}

\newcommand{\seqxn}{(x_n)_{n\geq1}}
\newcommand{\seqyn}{(y_n)_{n\geq1}}

\setlength{\parskip}{2ex} %--skip lines between paragraphs
\linespread{1.2}

% ------------------------------------------------------------------------------
% Document Begins Here!
%-------------------------------------------------------------------------------
\begin{document}

\title{Math 313 - Problem Set 7\vspace{-2ex}}   % replace X with appropriate number
\author{George Lamperis - UIN:664714097}        
\date{\vspace{-2ex}}
\maketitle


% ------------------------------------------------------------------------------
% Problem 1 
% ------------------------------------------------------------------------------
\begin{enumerate}[label=\bfseries1\alph*)]

% 1a) --------------------------------------------------------------------------
\item Let $I$ be an arbitrary interval in $\R$. Let $f$ be a function which is 
differentiable on the interior of $I$ and continuous at each endpoint of $I$. 
Assume that $f^\prime(x) \geq 0$ for every $x$ in the interior of $I$. 

Then $f$ is monotone increasing on $I$.

\begin{proof}
    Since $f$ is differentiable on the interior of $I$, this implies that $f$ is 
    continuous on the interior as well. Also, $f$ is continuous at the end points
    of $I$, so $f$ is continuous at $c$ for all $c \in I$.

    Let $a,b \in I, a<b$. We have that $f$ is continuous on $[a,b]$ and 
    differentiable on $(a,b)$. 

    Apply the Mean Value Theorem. There exists a $c \in (a,b)$ such that:
    \[f^\prime(c) = \frac{f(b) - f(a)}{b-a} \]

    By hypothesis $f^\prime (c) \geq 0$ and we also have $a < b$.

    \begin{align*}
        \frac{f(b) - f(a)}{b-a} \geq 0 &\implies f(b)-f(a) \geq 0 \\
                                       &\implies f(b) \geq f(a)
    \end{align*}

    Hence, $f$ is monotone increasing on $I$.
\end{proof}


% 1b) --------------------------------------------------------------------------
\item Let $I$ be an arbitrary interval in $\R$. Let $f$ be a function which is 
differentiable on the interior of $I$ and continuous at each endpoint of $I$. 
Assume that $f^\prime(x) > 0$ for every $x$ in the interior of $I$. 

Then $f$ is strictly monotone increasing on $I$.

\begin{proof}
    We have that $f$ is continuous at $c$ for all $c \in I$.

    Let $a,b \in I, a<b$. We have that $f$ is continuous on $[a,b]$ and 
    differentiable on $(a,b)$. 

    Apply the Mean Value Theorem. There exists a $c \in (a,b)$ such that:
    \[f^\prime(c) = \frac{f(b) - f(a)}{b-a} \]

    By hypothesis $f^\prime (c) > 0$ and we also have $a < b$.

    \begin{align*}
        \frac{f(b) - f(a)}{b-a} > 0 &\implies f(b)-f(a) > 0 \\
                                       &\implies f(b) > f(a)
    \end{align*}

    Hence, $f$ is strictly monotone increasing on $I$.
\end{proof}
\end{enumerate}


% ------------------------------------------------------------------------------
% Problem 2 
% ------------------------------------------------------------------------------
\begin{enumerate}[label=\bfseries2\alph*)]

% 2a) --------------------------------------------------------------------------
\item Let $I$ be an arbitrary interval in $\R$. Let $f$ be a function which is 
differentiable on the interior of $I$ and continuous at each endpoint of $I$. 
Assume that $f^\prime(x) \leq 0$ for every $x$ in the interior of $I$. 

Then $f$ is monotone decreasing on $I$.

\begin{proof}
    We have that $f$ is continuous at $c$ for all $c \in I$.

    Let $a,b \in I, a<b$. We have that $f$ is continuous on $[a,b]$ and 
    differentiable on $(a,b)$. 

    Apply the Mean Value Theorem. There exists a $c \in (a,b)$ such that:
    \[f^\prime(c) = \frac{f(b) - f(a)}{b-a} \]

    By hypothesis $f^\prime (c) \leq 0$ and we also have $a < b$.

    \begin{align*}
        \frac{f(b) - f(a)}{b-a} \leq 0 &\implies f(b)-f(a) \leq 0 \\
                                       &\implies f(b) \leq f(a)
    \end{align*}

    Hence, $f$ is monotone decreasing on $I$.
\end{proof}


% 2b) --------------------------------------------------------------------------
\item Let $I$ be an arbitrary interval in $\R$. Let $f$ be a function which is 
differentiable on the interior of $I$ and continuous at each endpoint of $I$. 
Assume that $f^\prime(x) < 0$ for every $x$ in the interior of $I$. 

Then $f$ is strictly monotone decreasing on $I$.

\begin{proof}
    We have that $f$ is continuous at $c$ for all $c \in I$.

    Let $a,b \in I, a<b$. We have that $f$ is continuous on $[a,b]$ and 
    differentiable on $(a,b)$. 

    Apply the Mean Value Theorem. There exists a $c \in (a,b)$ such that:
    \[f^\prime(c) = \frac{f(b) - f(a)}{b-a} \]

    By hypothesis $f^\prime (c) < 0$ and we also have $a < b$.

    \begin{align*}
        \frac{f(b) - f(a)}{b-a} < 0 &\implies f(b)-f(a) < 0 \\
                                       &\implies f(b) < f(a)
    \end{align*}

    Hence, $f$ is strictly monotone decreasing on $I$.
\end{proof}
\end{enumerate}


% ------------------------------------------------------------------------------
% Problem 3
% ------------------------------------------------------------------------------
\begin{enumerate}[label=\bfseries3\alph*)]

% 3a) --------------------------------------------------------------------------
\item Let $f$ be a twice differentiable function defined on $\R$. Assume that 
$f^{\prime\prime}(x) + f(x) = 0$ for every $x$. 

Then $f(x)^2 + f^\prime(x)^2$ is constant.

\begin{proof}
    Since $f$ is differentiable and $(f(x))^\prime = f^\prime(x)$, then by
    the product rule:
    \begin{align*} 
        (f(x)^2)^\prime &= (f(x)f(x))^\prime \\
                        &= f^\prime(x)f(x) + f(x)f^\prime(x) \\
                        &= 2f(x)f^\prime(x)
    \end{align*}

    Since $f^\prime$ is differentiable and 
    $(f^\prime(x))^\prime = f^{\prime\prime}(x)$, then by the product rule:
    \begin{align*} 
        (f^\prime(x)^2)^\prime &= (f^\prime(x)f^\prime(x))^\prime \\
                               &= f^{\prime\prime}(x)f^\prime(x) + f^\prime(x)f^{\prime\prime}(x) \\
                               &= 2f^\prime(x)f^{\prime\prime}(x)
    \end{align*}

    Since $f(x)^2$ and $f^\prime(x)^2$ are differentiable, then by the sum rule:
    \begin{align*}
        (f(x)^2 + f^\prime(x)^2)^\prime &= 2f(x)f^\prime(x) + 2f^\prime(x)f^{\prime\prime}(x) \\
                                        &= 2f^\prime(x)(f(x) + f^{\prime\prime}(x)) \\
                                        &= 2f^\prime(x)(0)  &\text{By hypothesis}  \\
                                        &= 0 
    \end{align*}

    Since $(f(x)^2 + f^\prime(x)^2)^\prime = 0$ for every $x \in \R$, then by
    Corollary 1 to the mean value theorem, $f(x)^2 + f^\prime(x)^2$ is constant.
\end{proof}


% 3b) --------------------------------------------------------------------------
\item Let $f$ be a twice differentiable function defined on $\R$. Assume that 
$f^{\prime\prime}(x) + f(x) = 0$ for every $x$ and that $f(0) = f^\prime(0) = 0$. 

Then $f(x) = 0$ for every $x$.

\begin{proof}
    We showed in (3a) that $f(x)^2 + f^\prime(x)^2$ is constant, i.e. there 
    is a $C$ such that for all $x$ we have $f(x)^2 + f^\prime(x)^2=C$ 

    By hypothesis, $f(0) = f^\prime(0) = 0$, and so:
    \begin{align*}
    C   &= f(0)^2 + f^\prime(0)^2 \\
        &= (0)^2 + (0)^2 \\
        &= 0   
    \end{align*}

    Hence, $f(x)^2 + f^\prime(x)^2 = 0$ for every $x$.

    For any $a,b \in \R$, we have $a^2 \geq 0$ and $b^2 \geq 0$. If $a^2 + b^2 = 0$, then
    $a=b=0$. Therefore, $f(x)=0$ for all $x$.
\end{proof}
\end{enumerate}


% ------------------------------------------------------------------------------
% Problem 4
% ------------------------------------------------------------------------------
\begin{enumerate}[label=\bfseries4\alph*)]

% 4a) --------------------------------------------------------------------------
\item Let $f$ be a continuous function defined on $\R$. Let $c \in \R$ be given,
and define a sequence $\seqxn$ recursively by setting $x_1 = c$ and $x_{n+1} = f(x_n)$
for every $n \geq 1$. Assume $\seqxn$ converges to a limit $L$.

Then $f(L)=L$.

\begin{proof}
    Let $\epsilon > 0$ and choose $\delta = \epsilon$.

    Since $\seqxn$ converges to $L$, there is an $N$ such that for all 
    $n \geq N$, we have $|x_n - L| < \delta$. 

    Then we have 
    \[0 < |x_{n+1} - L| < \delta \implies |f(x_n) - L| < \epsilon\] 
    i.e. $\lim\limits_{x \to L} f(x) = L$.

    Since $f$ is continuous, $\lim\limits_{x\to L} f(x)= f(L)$, and so by transitivity, $f(L) = L$.
\end{proof}


% 4b) --------------------------------------------------------------------------
\item A $k$-contraction is continuous everywhere.

\begin{proof}
    Let $c \in \R$ be given. 

    Let $\epsilon > 0$ and choose $\delta = \epsilon / k$. 

    Then for every $x\in \R$ we have:
    \[ 0 < |x-c| < \delta \implies k|x-c| < \delta k = \epsilon \]

    Since $f$ is a $k$-contraction, we have: 
    \[ |f(x)-f(c)| \leq k|x-c| < \epsilon \]
    i.e, $\lim\limits_{x\to c} f(x)=f(c)$.

    Hence, a $k$-contraction is continuous everywhere.
\end{proof}


% 4c) --------------------------------------------------------------------------
\item Let $k \in \R$ with $0 < k < 1$ and let $f$ be a $k$-contraction. Let $c \in R$
be given, and define a sequence $\seqxn$ recursively by $x_1 = c$ and 
$x_{n+1} = f(x_n)$. Set $C = |x_1 - x_2|$.

Then $|x_n - x_{n+1}| \leq k^{n-1}C$ for every $n$ and the series 
$\sum_{i=1}^\infty (x_{i+1}-x_i)$ converges absolutely.

\begin{proof}
    We will induct on $n$

    We have $C = |x_1 - x_2|$ by definition. 

    Now suppose that $|x_n - x_{n+1}| \leq k^{n-1}C$. Then we have that 
    $|x_{n+1} - x_{n+2}| = |f(x_n) - f(x_{n+1})|$. Since $f$ is a $k$-contraction, 
    $|f(x_n) - f(x_{n+1})| \leq k|x_n-x_{n+1}| \leq k^n C$.

    Hence, $|x_n - x_{n+1}| \leq k^{n-1}C$ for every $n$.

    Since $|k| < 1$, the series $\sum_{i=1}^\infty k^{n-1}C$ is a convergent
    geometric series. 

    Since $0 \leq |x_n - x_{n+1}| \leq k^{n-1}C$ for all $n$, then by the 
    comparison test, $\sum_{i=1}^\infty (x_{i+1}-x_i)$ converges absolutely.
\end{proof}

% 4d) --------------------------------------------------------------------------
\item Under the hypotheses of (4c), the sequence $\seqxn$ converges to a limit
$L$ and that $f(L)=L$.

\begin{proof}
    Notice that
    \begin{align*}
    \sum_{i=1}^n (x_{i+1}-x_i) &= x_n - x_1 \\
                               &= x_n - c
    \end{align*}

    The series $\sum_{i=1}^n (x_{i+1}-x_i)$ converges absolutely, and therefore
    it converges. This series converges if and only $\seq{x_n-c}{n}$ is a 
    null sequence. Hence, the sequence $\seqxn$ converges to $L$.

    Then by (4a), we have that $f(L)=L$.
\end{proof}
\end{enumerate}
% ------------------------------------------------------------------------------
% Don't edit below this!
% ------------------------------------------------------------------------------
\end{document}
