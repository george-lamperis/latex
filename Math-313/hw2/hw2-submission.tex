\documentclass[12pt, letterpaper]{article}
 
\usepackage[margin=1in]{geometry} 
\usepackage{amsmath,amsthm,amssymb}
\usepackage{enumitem}


% ------------------------------------------------------------------------------
% 
% ------------------------------------------------------------------------------
\linespread{1.3}

\newcommand{\N}{\mathbb{N}}
\newcommand{\Z}{\mathbb{Z}}

\renewcommand\thesection{\arabic{section}.}
\renewcommand\thesubsection{\thesection\alph{subsection})}

% ------------------------------------------------------------------------------
% theorem templates
% ------------------------------------------------------------------------------
\theoremstyle{definition} % disables italics in theorems

\newtheorem*{prop}{Proposition}
\newtheorem*{definition}{Definition}


\newtheorem{lemma}{Lemma}[subsection]
\renewcommand{\thelemma}{\arabic{lemma}}

\newtheorem*{lemma*}{Lemma}

\newtheorem{case}{Case}[subsection]
\renewcommand{\thecase}{\arabic{case}}

% ------------------------------------------------------------------------------
% Document Begins Here!
%-------------------------------------------------------------------------------
\begin{document}

% ------------------------------------------------------------------------------
% Title
% ------------------------------------------------------------------------------
\title{Math 313 - Problem Set 2}%replace X with the appropriate number
\author{George Lamperis - UIN:664714097} %if necessary, replace with your course title
\date{}
\maketitle

% ------------------------------------------------------------------------------
% Problem 1 - Done
% ------------------------------------------------------------------------------
\section{}

\begin{prop}
    If $m$ and $n$ are natural numbers, then the product $mn$ is also a natural
    number.

    \begin{proof}
        We will induct on $n$.

        Suppose $n=1$ and $m$ is some natural number. Then $m \cdot 1 = m$, which
        is a natural number. 

        Now suppose for inductive hypothesis that $mn$ is a natural number. Then:
        \begin{align*}
            m(n+1) &= mn + m \cdot 1 \\
                   &= mn + m 
        \end{align*}

        By inductive hypothesis we have that $mn$ is a natural number. Additionally, 
        we have that the sum of two natural numbers is a natural number (proved 
        in class). Hence, $mn + m$ is a natural number.
    \end{proof}
\end{prop}


% ------------------------------------------------------------------------------
% Problem 2 - Done
% ------------------------------------------------------------------------------
\section{}

\begin{definition}
    Given $m,n \in N$ such that $n > m$. Then $n = m + 1 + ... + 1$ for some 
    finite amount of 1's, say $k$ of them. 

    We define $n - m = k$ and $m - n = -k$. for some $k \in N$.
\end{definition}

\begin{prop}
    If $m$ and $n$ are natural numbers, then one of the following is true:
    \begin{enumerate}[label=(\roman{*})]
        \item $n-m$ is a natural number
        \item $m-n$ is a natural number
        \item $m=n$
    \end{enumerate}

    \begin{proof}
        Notice that exactly one of these three properties can hold. We will fix 
        an arbitary natural number $m$ and induct on a natural number $n$: 
        
        Since $m \in \N$, either $m=1$ or $m=k+1$ for some $k \in \N$. If $m=1$,
        then we have $1=1$. If $m=k+1$, then $m-1 = (k+1)-1 = k$, which is a 
        natural number.

        Suppose, for inductive hypothesis, that either (i), (ii), or (iii) hold 
        for $m$ and $n$. 

        \begin{case}
            If $n-m=k$, for some $k \in \N$. Then:

            \begin{align*}
                (n+1)-m &= (n-m) + 1    \\
                        &= k + 1
            \end{align*}

            which is a natural number.
        \end{case}

        \begin{case}
            If $m-n=k$ for some $k \in \N$, there are two possibilities: 
            Either $k=1$ or $k=j+1$ for some $j\in\N$

            If $k=1$, then we have $m=n+1$, and so the hypothesis holds.

            If $k=j+1$, then:
            \begin{align*}
                m -(n+1) &= (m-n) - 1   \\
                         &= (j+1) - 1   \\
                         &= j
            \end{align*}

            which is a natural number. 
        \end{case}

        \begin{case}
            If $m=n$, then $(n+1)-m = 1$ which is a natural number.
        \end{case}

        Therefore, by induction, the proposition holds.
    \end{proof}
\end{prop} 


% ------------------------------------------------------------------------------
% Problem 3 - Done
% ------------------------------------------------------------------------------
\section{}

\begin{definition}
    Define an integer to be an element of $F$ which is either a natural number,
    0, or of the form $-n$ for some natural number $n$.
\end{definition}

% ------------------------------------------------------------------------------
% 3a) - Done
% ------------------------------------------------------------------------------
\subsection{}

\begin{prop}
    If $x$ and $y$ are integers, then $x+y$ is an integer.

    \begin{proof} 
        Let $x$ and $y$ be integers:

        \begin{case} 
            If both $x$ and $y$ are natural numbers, then this follows by a 
            proposition proved in class.
        \end{case}

        \begin{case} 
            If $x= -j$ and $y= -k$ for natural numbers $j,k$, then:

            \begin{align*}
                x+y &= -j - k  \\
                    &= -(j+k)  &\text{(HW1)}
            \end{align*}

            The sum of two natural numbers is a natural number,
            and so $j+k \in \N$. We have $x+y=-(j+k)$, and so by definition, $x+y$ 
            is an integer.
        \end{case}

        \begin{case} 
            If $x$ is a natural number and $y= -k$ for a natural number $k$, 
            then $x+y = x-k$. By Problem 2, we have that either $x-k=j$, $k-x=j$
            or $x=k$, for some natural number $j$.

            If $x-k=j$ then we are done, since $j \in \N$.

            If $k-x=j$, then by definition of subtraction, $x-k=-j$, which is an
            integer.

            If $x=k$, then by (F4), $x-k=0$ which is an integer.
        \end{case}

        \begin{case} 
            If $x$ is an integer and $y=0$ then $x+0=x$, which is an integer.
        \end{case}

    \end{proof}
\end{prop}

% ------------------------------------------------------------------------------
% 3b) Done
% ------------------------------------------------------------------------------
\subsection{}

\begin{lemma}
    Given an element of an ordered field $x$, $-(-x) = x$

    \begin{proof}
        By definition, $-(-x)$ is an element such that $(-x)+(-(-x)) = 0$

        Given $x\in F$, we have that $x+(-x) = 0$. By 
        applying the Commutative Property, we have $(-x)+x=0$

        \begin{align*}
            (-x) + x &= (-x) + (-(-x)) \\
                   x &= -(-x)
        \end{align*}

        by the Cancellation Law for addition. Hence, $-(-x)=x$.
    \end{proof}
\end{lemma}


\begin{prop}
    If $x$ and $y$ are integers, then $xy$ is an integer.
    \begin{proof}
        \begin{case}
            If both $x$ and $y$ are natural numbers, then this follows by 
            Problem 1.
        \end{case}

        \begin{case}
            If $x= -j$ and $y= -k$ for natural numbers $j,k$, then:

            \begin{align*}
                xy &= (-j)(-k) \\
                   &= (-1)(-1)(jk) \\  
                   &= -(-jk)    \\
                   &= jk        & \text{(by previous lemma)}
            \end{align*}

            The product of two natural numbers is a natural number, and so $xy$ 
            is an integer.
        \end{case}

        \begin{case}
            If $x$ is a natural number and $y=-k$ for a natural number $k$, then:

            \begin{align*}
                xy &= x(-k) \\
                   &= -(xk)    & \text{(HW1 - 1d)}
            \end{align*}

            The product of two natural numbers is a natural number, and so $xk$ 
            is an integer. We have $xy=-(xk)$ and so $xy$ is an integer.
        \end{case}

        \begin{case}
            If $x$ is an integer and $y=0$, then $xy=0$ which is an integer.
        \end{case}
    \end{proof}
\end{prop}

% ------------------------------------------------------------------------------
% Problem 4
% ------------------------------------------------------------------------------
\section{}
\begin{definition}
    A rational number is defined to be an element of $F$ that has the form
    $\dfrac{a}{b}$, where $a$ and $b$ are integers and $b \neq 0$.
\end{definition}
% ------------------------------------------------------------------------------
% 4a)
% ------------------------------------------------------------------------------
\subsection{}

\begin{lemma}
    Given a rational number $x= \dfrac{a}{b}$, and an integer $c \neq 0$, then
    $x=\dfrac{ac}{bc}$.

    \begin{proof}
        \begin{align*}
            x   &= \dfrac{a}{b} \\
                &= \dfrac{a}{b} \cdot 1 \\
                &= \dfrac{a}{b} \cdot (cc^{-1}) \\
                &= \dfrac{a}{b} \cdot \dfrac{c}{c} \\
                &= \dfrac{ac}{bc}      &\text{(HW1 - 3d)}
        \end{align*}
    \end{proof} 
\end{lemma}

\begin{prop}
    If $x$ and $y$ are rational numbers, then $x+y$ and $xy$ are rational 
    numbers.
    \begin{proof}
        By definition of rational numbers, we have $x = \dfrac{a}{b}$
        and $y = \dfrac{c}{d}$ for integers $a,b,c,d$ with $b \neq 0$ and 
        $d \neq 0$. 

        For addition, we have:
        \begin{align*}
            x+y &= \dfrac{a}{b} + \dfrac{c}{d} \\
                &= \dfrac{ad}{bd} + \dfrac{cb}{db}  &\text{(previous lemma)} \\
                &= \dfrac{ad}{bd} + \dfrac{bc}{bd} \\ 
                &= (ad)(bd)^{-1} + (bc)(bd)^{-1}   \\
                &= (bd)^{-1} (ad + bc)  \\
                &= \dfrac{ad+bc}{bd}
        \end{align*}

        The product of two integers is an integer, and so $ad$, $bc$ and $bd$ are
        integers. The sum of two integers is an integer, and so $ad+bc$ is an 
        integer. Additionally, $b\neq 0$ and $d\neq 0$. Then by HW1 - (3c), 
        $bd \neq 0$. Hence, $x+y$ is a rational number.

        For multiplication, we have:
        \begin{align*}
            x+y &= \dfrac{a}{b} \cdot \dfrac{c}{d} \\
                &= \dfrac{ac}{bd}   &\text{(HW1 - 3d)}
        \end{align*}

        The product of two integers is an integer, and so both $ac$ and $bd$ are
        integers. Additionally, $b\neq 0$ and $d\neq 0$. Then by HW1 - (3c), 
        $bd \neq 0$. Hence, $xy$ is a rational number.
    \end{proof}

\end{prop}
% ------------------------------------------------------------------------------
% 4b)
% ------------------------------------------------------------------------------
\subsection{}
\begin{prop}
    If $x$ is a rational number then $-x$ is a rational number, and if
    $x \neq 0$ then $x^{-1}$ is rational.
    \
\end{prop}
% ------------------------------------------------------------------------------
% Document Ends
% ------------------------------------------------------------------------------
\end{document}
