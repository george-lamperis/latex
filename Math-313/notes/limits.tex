% ------------------------------------------------------------------------------
% Limits
% ------------------------------------------------------------------------------

\begin{defn}[Interval]
    An interval is a set of real numbers having a similar form to: 
    \[[a,b] = \{x \in \R : a \leq x \leq b, a<b \}\].

    Intervals can be open, e.g. $(a,b)$ and intervals can be infinite e.g. 
    $(-\infty, \infty ) = \R$.
\end{defn}


\begin{defn}[Limit]
    Let $I$ be an interval and let $c \in I$ be given. Suppose $f$ is a real valued
    function defined for every point in $I$, except for perhaps $c$.

    We say that a number $L$ is a limit of $f$ as $x \rightarrow c$ if for every 
    sequence $\seq{x_n}{n}$ such that $x_n \in I$, $x_n \neq c$ and $\seq{x_n}{n}$
    converges to $c$, then we have that $(f(x_n))_{n \geq 1}$ converges to $L$.

    If we write $\lim\limits_{x \to c} f(x) = L $, it means that $f$ has a limit
    when $x \rightarrow c$ and that limit equals $L$.
\end{defn}


\begin{prop}
    A function $f$ defined on an interval $I$, except perhaps at $c \in I$, has
    at most one limit.
\end{prop}
\begin{proof}
    Suppose $L$ and $L'$ are limits of $f$ as $x \rightarrow c$. Choose a 
    sequence $\seq{x}{n}$ of points in $I$ such that $x_n \neq c$ and $\seq{x}{n}$ 
    converges to $c$.

    Then $\seq{f(x_n)}{n}$ converges to $L$ and $L'$. Limits are unique for 
    sequences, and so $L = L'$.
\end{proof}


\begin{prop}
    Let $f(x)$ and $g(x)$ be real valued functions. If $\lim\limits_{x \to c} f(x) = L $
    and  $\lim\limits_{x \to c} g(x) = L' $, then:

    \begin{enumerate}
        \item $\lim\limits_{x \to c} f(x)+g(x) = L+L'$,
        \item $\lim\limits_{x \to c} f(x)g(x) = LL'$,
    \end{enumerate}
\end{prop}
\begin{proof}
    Reduce this problem to sums and products of sequences.
\end{proof}


\begin{corollary}
    If $\lim\limits_{x \to c} f(x) = L $ and $\lim\limits_{x \to c} g(x) = L' $, then
    $\lim\limits_{x \to c} f(x)-g(x) = L-L'$,
\end{corollary}


\begin{remark}
    If $f$ is a constant function with $f(x)=k$ for all $x \in I$, then 
    $\lim\limits_{x \to c} f(x) = k$.
\end{remark}


\begin{prop}
    If $g(x)$ is defined and nonzero on $I$ and if $\lim\limits_{x \to c} g(x) = L \neq 0 $,
    then \[\lim_{x \to c} \frac{1}{g(x)} = \frac{1}{L} \].
\end{prop}


\begin{corollary}
    Let $f(x)$ and $g(x)$ are defined on $I$ (except perhaps at $c$). Suppose 
    $g(x)$ is nonzero for all $x\in\R$. 

    If $\lim\limits_{x \to c} f(x) = L $ and $\lim\limits_{x \to c} g(x) = L' \neq 0 $, 
    then \[\lim_{x \to c} \frac{f(x)}{g(x)} = \frac{L}{L'}\].
\end{corollary}


\begin{remark}
    Saying $\lim\limits_{x \to c} f(x)=L$ means that for every sequence $\seqxn$
    with $x_n \in I$, $x_n \neq c$, and $x_n \to c$ we have $f(x) \to L$.

    Roughly, this means that when $x$ gets really close to $c$ but $x \neq c$, 
    $f(x)$ is getting very close to $L$.
\end{remark}


\begin{prop}
    Let $f(x)$ be defined on $I$ except possibly at $c \in I$. Let $L$ be given.

    Then $\lim\limits_{x \to c} f(x)=L$ if and only if the following holds:

    For every $\epsilon >  0$ there is a $\delta > 0$ such that for every
    $x \in I$ with $0 < |x-c| < \delta$ we have $|f(x) - L| < \epsilon$.
\end{prop}

% ------------------------------------------------------------------------------
% Continuity
% ------------------------------------------------------------------------------

\begin{defn}[Continuous Function]
    Let $f$ be defined on some interval $I$. We say that $f$ is continuous at 
    $c \in I$ if $\lim\limits_{x \to c} f(x) = f(c)$.

    We say $f$ is continuous if $f$ is continuous at $c$ for all $c \in I$.
\end{defn}


\begin{lemma}
    The function $f(x)=x$ and any constant functions are continuous on $\R$.
\end{lemma}


\begin{prop}
    If $f(x)$ and $g(x)$ are continuous at a point $c$, then $f(x)+g(x)$ and
    $f(x)g(x)$ are also continuous at $c$.
\end{prop}


\begin{prop}
    Let $f$ be defined on $I$. Let $c \in I$ be given.

    Then $f$ is continuous if and only if for every sequence $\seqxn$ of points
    in $I$ converging to $c$, the sequence $\seq{f(x_n)}{n}$ converges to $f(c)$.
\end{prop}


\begin{theorem}[Intermediate Value Theorem]
    Suppose a function $f$ is continuous on an interval $[a,b]$ and that 
    $f(a) \leq 0 \leq f(b)$.

    Then for some $c \in [a,b]$ we have $f(c)=0$.
\end{theorem}
    

\begin{defn}[Polynomial]
    A degree-n polynomial is a function $f:\R \to \R$ of the form:
    \[f(x) = a_n x^n + a_{n-a} x^{n-1} + ... + a_0\]
    where $x \in \R$ and $n \in \Z, n \geq 0$.
\end{defn}

\begin{theorem}
    If $n$ is odd, then for every degree-n polynomial $f$, there is a $c \in \R$ 
    such that $f(c)=0$.

    Note: This is not true for even values of $n$. For example, $f(x) = x^2 + 1$.
\end{theorem}
