\documentclass[12pt, letterpaper]{article}
 
\usepackage[margin=1in]{geometry} 
\usepackage{amsmath,amsthm,amssymb}
\usepackage{enumitem}

% ------------------------------------------------------------------------------
% Document formatting
% ------------------------------------------------------------------------------
\setlength{\parskip}{2ex} %--skip lines between paragraphs
\setlength{\parindent}{0pt} %--don't indent paragraphs

% \linespread{1.3}

\makeatletter
\renewcommand\section{\@startsection{section}{1}{\z@}%
                                  {-3.5ex \@plus -1ex \@minus -.2ex}%
                                  {2.3ex \@plus.2ex}%
                                  {\normalfont\large\bfseries}}
\makeatother


% ------------------------------------------------------------------------------
% Math commands
% ------------------------------------------------------------------------------
\newcommand{\N}{\mathbb{N}}
\newcommand{\Z}{\mathbb{Z}}
\newcommand{\R}{\mathbb{R}}

\renewcommand{\epsilon}{\varepsilon}


% ------------------------------------------------------------------------------
% theorem templates and formatting
% ------------------------------------------------------------------------------
\renewcommand\thesection{\arabic{section}.}
\renewcommand\thesubsection{\thesection\alph{subsection})}

\theoremstyle{definition} % disables italics in theorems

\newtheorem*{prop}{Proposition}
\newtheorem*{definition}{Definition}


\newtheorem{lemma}{Lemma}[subsection]
\renewcommand{\thelemma}{\arabic{lemma}}

\newtheorem*{lemma*}{Lemma}

\newtheorem{case}{Case}[subsection]
\renewcommand{\thecase}{\arabic{case}}

% ------------------------------------------------------------------------------
% Document Begins Here!
%-------------------------------------------------------------------------------
\begin{document}

\title{Math 313 - Problem Set 4}%replace X with the appropriate number
\author{George Lamperis - UIN:664714097} %if necessary, replace with your course title
\date{}
\maketitle

\begin{enumerate}[label=\bfseries\arabic*.]

% ------------------------------------------------------------------------------
% 1
% ------------------------------------------------------------------------------
\item 
\begin{enumerate}[label=\bfseries(\alph*)]

    % 1a) ----------------------------------------------------------------------
    \item The constant sequence $(0)_{n \geq 1}$ is a null sequence.
    \begin{proof} ~\\
        Let $\epsilon > 0$ be given.

        For every $n$, $(0)_{n \geq 1} = 0$. To see that this sequence is a null 
        sequence, take $N=1$. Then we have $(0)_{n \geq 1} < \epsilon$ for all
        $n \geq N$. Hence, $(0)_{n \geq 1}$ is a null sequence.
    \end{proof}

    % 1b) ----------------------------------------------------------------------
    \item A constant sequence $(c)_{n \geq 1}$ converges to $c$.
    \begin{proof} ~\\
        We need to show that $(c-c)_{n \geq 1}$ is a null sequence. 

        But $(c-c)_{n \geq 1} = (0)_{n \geq 1}$. By (1a), we have that  
        $(0)_{n \geq 1}$ is a null sequence. Hence, $(c)_{n \geq 1}$ converges 
        to $c$.
    \end{proof}

    % 1c) ----------------------------------------------------------------------
    \item If $(x_n)_{n \geq 1}$ converges to $L$, then $(cx_n)_{n \geq 1}$ 
    converges to $cL$ for any constant $c$.

    \begin{proof} ~\\
        Suppose that $(x_n)_{n \geq 1}$ converges to $L$.  

        By (1b), the constant sequence $(c)_{n \geq 1}$ converges to $c$.

        As proved in lecture, the product of two convergent sequences converges 
        to the product of their limits. Hence, $(cx_n)_{n \geq 1}$ converges to 
        $cL$ for any constant $c$.
    \end{proof}

    % 1d) ----------------------------------------------------------------------
    \item If $(x_n)_{n \geq 1}$ converges to $L$ and $(y_n)_{n \geq 1}$ converges
    to $L'$, then $(x_n - y_n)_{n \geq 1}$ converges to $L-L'$.

    \begin{proof} ~\\
        By (1c), we have that $(-y_n)_{n \geq 1}$ converges to $-L'$.

        We showed in class that the sum of two convergent sequences converges to 
        the sum of their limits. Hence, $(x_n - y_n)_{n \geq 1}$ converges to
        $L-L'$.
    \end{proof}
\end{enumerate}

% ------------------------------------------------------------------------------
% 2 
% ------------------------------------------------------------------------------
\item 
\begin{enumerate}[label=\bfseries(\alph*)]
        
    % 2a) ----------------------------------------------------------------------
    \item Let $x,L,\epsilon$ be real numbers with $\epsilon > 0$. 
    
    Then $|x-L| < \epsilon$ if and only if $L-\epsilon < x < L+\epsilon$.
    \begin{proof} ~\\
        Since $\epsilon > 0$, we have $-\epsilon < 0$.

        By (HW3-2b) we have that $|x-L| = \pm(x-L)$, and by (HW3-2a) we have that
        $|x-L| \geq 0$. 


        Suppose that $|x-L| < \epsilon$. Then:
        \begin{equation*}
            -\epsilon < 0 \leq |x-L| < \epsilon
        \end{equation*}

        If $|x-L| = x-L$, then:
        \begin{eqnarray*}
            -\epsilon < x-L < \epsilon \\
            L - \epsilon < x < L + \epsilon 
        \end{eqnarray*}

        If $|x-L| = -(x-L)$, then:
        \begin{eqnarray*}
            -\epsilon < -(x-L) < \epsilon \\
            \epsilon > x-L > -\epsilon \\
            L + \epsilon > x > L - \epsilon 
        \end{eqnarray*}

        Conversely, suppose that $L-\epsilon < x < L+\epsilon$. Then we have:
        \begin{equation*}
            -\epsilon < x-L < \epsilon
        \end{equation*}

        We know that $|x-L| = \pm(x-L)$. 

        If $|x-L| = x-L$, we already have $x-L < \epsilon$.

        Now suppose $|x-L| = -(x-L)$. By assumption, we have $-\epsilon < x-L$.
        \begin{equation*}
            -\epsilon < x-L \implies -(x-L) < \epsilon
        \end{equation*}

        Hence, $|x-L| < \epsilon$ if and only if $L-\epsilon < x < L + \epsilon$.
    \end{proof}

    % 2b) ----------------------------------------------------------------------
    \item A sequence $(x_n)_{n \geq 1}$ converges to a number $L$ if and only if
    for every $\epsilon > 0$, there is an $N$ such that:

    \begin{equation*} 
        L-\epsilon < x_n < L+\epsilon
    \end{equation*} 

    for all $n \geq N$.

    \begin{proof} ~\\
        Suppose $(x_n)_{n \geq 1}$ converges to $L$.

        Then for every $\epsilon > 0$, there is an $N$ such that for all 
        $n \geq N$, we have $|x_n-L| \leq \epsilon$.

        By (2a), $|x_n - L| < \epsilon$ if and only if $L-\epsilon < x_n < L + \epsilon$.

        Hence, for all $\epsilon > 0$, $(x_n)_{n \geq 1}$ converges to $L$ if 
        and only if there is an $N$ such that $L-\epsilon < x_n < L + \epsilon$ 
        for all $n \geq N$.
    \end{proof}

    % 2c) ----------------------------------------------------------------------
    \item If $(x_n)_{n \geq 1}$ converges to a number $L>0$, then there is an 
    $N$ such that $x_n > 0$ for every $n \geq N$.

    \begin{proof} ~\\
        If $(x_n)_{n \geq 1}$ converges to $L>0$, then by (2b), for all 
        $\epsilon > 0$, there is an $N$ such that for all $n \geq N$: 
        \begin{equation*}
            L-\epsilon < x_n < L+\epsilon
        \end{equation*}

        We have $L>0$, so take $\epsilon = L$. Then
        \begin{align*}
            L-\epsilon < x_n &\implies L-L < x_n \\
                             &\implies 0 < x_n
        \end{align*}

        And so we have $0<x_n$ for every $n \geq N$.
    \end{proof}

    % 2d) ----------------------------------------------------------------------
    \item
    \begin{proof} ~\\
    \end{proof}
    % 2e) ----------------------------------------------------------------------
    \item
    \begin{proof} ~\\
    \end{proof}
\end{enumerate}

% ------------------------------------------------------------------------------
% 3
% ------------------------------------------------------------------------------
\item If $(x_n)_{n \geq 1}$ converges to $L$, then for every natural
number $k$, the sequence $(x_n^k)_{n \geq 1}$ converges to $L^k$.

\begin{proof}
    We will induct on $n$.

    By hypothesis, we have $(x_n)_{n \geq 1}$ converges to $L$.

    Now suppose, for inductive hypothesis, that $(x_n^k)_{n \geq 1}$ converges 
    to $L^k$.

    Then, as proved in class, the product of two convergent sequences converges 
    to the product of their limits. And so 
    \begin{equation*}
        (x_n^k \cdot x)_{n \geq 1} \text{ converges to } L^k \cdot L \implies (x_n^{k+1})_{n \geq 1} \text{ converges to } L^{k+1}.
    \end{equation*}

    Hence, by induction, $(x_n^k)_{n \geq 1}$ converges to $L^k$.
\end{proof}

% ------------------------------------------------------------------------------
% 4
% ------------------------------------------------------------------------------
\item If $Y \subset \R$ is non-empty and bounded below, then the set of all 
lower bounds for $Y$ has a greatest element. 

\begin{proof} ~\\
    Let $X$ be the set of all lower bounds for $Y$. 

    We have $Y \neq \emptyset$, and so $X \neq \emptyset$. Since $X$ is the set of 
    lower bounds for $Y$, we have that $x \leq y$ for all $x \in X$ and for all 
    $y \in Y$ 

    Now we can apply Dedekind's Axiom: there is a $c \in \R$. such that 
    $c \geq x$ and $c \leq y$, for all $x \in X$ and for all $y \in Y$. 

    Since $c \leq y$ for all $y \in Y$, $c$ is a lower bounds for $y$, and so
    we have $c \in X$.

    Additionally, we have $c \geq x$ for all $x \in X$, and so $c$ is the 
    greatest element in $X$. 

    Hence, the set of all lower bounds for $Y$ has a greatest element. 
\end{proof}

% ------------------------------------------------------------------------------
% 5
% ------------------------------------------------------------------------------
\item
\begin{enumerate}[label=\bfseries(\alph*)]
    
    % 5a) ----------------------------------------------------------------------
    \item For every natural number $n$, $2^n \geq n+1$.
    \begin{proof} ~\\
        We will induct on $n$.

        For the base case, we have $2^1 = 1+1$.

        Now suppose that $2^n \geq n+1$. Then:
        \begin{align*}
            2^n \geq n+1 &\implies 2 \cdot 2^n \geq 2(n+1) \\
                         &\implies 2^{n+1} \geq 2n + 2
        \end{align*}

        Because $n \geq 1$ ($n$ is a natural number), we have
        \begin{equation*}
            n < 2n \implies n+2 < 2n+2
        \end{equation*}

        By transitivity, we have $n+2 < 2^n+1$.

        Hence, by induction, $2^n \geq n+1$ for all natural numbers $n$.
    \end{proof}
\end{enumerate}

% ------------------------------------------------------------------------------
% Don't edit below this!
% ------------------------------------------------------------------------------
\end{enumerate}
\end{document}
