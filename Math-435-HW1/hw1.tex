\RequirePackage[l2tabu, orthodox]{nag}

\documentclass[letterpaper, 12pt]{article}

\usepackage[margin=1in]{geometry}
\usepackage{amsmath,amsthm,amssymb,mathtools}
\usepackage{array}              % required for tabular environment
\usepackage[parfill]{parskip}   % paragraph formatting
\usepackage{paralist}           % compactitem

% ------------------------------------------------------------------------------
% Theorem Commands
% ------------------------------------------------------------------------------
\newtheoremstyle{mystyle}% name
    {2ex}       % Space above
    {4ex}       % Space below
    {}          % Body font
    {}          % Indent amount (empty = no indent, \parindent = para indent)
    {\bfseries} % Thm head font
    {.}         % Punctuation after thm head
    {\newline}  % Space after thm head: \newline = linebreak
    {}          % Thm head spec

\theoremstyle{mystyle}
\newtheorem*{prop}{Proposition}
\newtheorem*{lemma}{Lemma}

% ------------------------------------------------------------------------------
% Begin Document
% ------------------------------------------------------------------------------
\title{Math 435 - Homework 1}
\author{George Lamperis}
\date{}

\begin{document}
\maketitle

% ------------------------------------------------------------------------------
% 2.1(a), page 18
% ------------------------------------------------------------------------------
\section*{2.1.(a)}

\begin{lemma}
    If $a$ is an integer, then either $a^2 \equiv 0 \pmod 3$ or 
    $a^2 \equiv 1 \pmod 3$.
\end{lemma}
\begin{proof}
    We have two cases: the case where 3 divides $a$, and the case where 3 does
    not divide $a$.

    If 3 divides $a$, then $a \equiv 0 \pmod 3$. It follows that 
    $a^2 \equiv 0 \pmod 3$ and so we are done.
    
    If 3 does not divide $a$, then either $a \equiv 1 \pmod 3$ or 
    $a \equiv 2 \pmod 3$. Since $2^2 \equiv 1 \pmod 3$, in either case we have
    that $a^2 \equiv 1 \pmod 3$.
\end{proof}


\begin{prop}
    If $(a,b,c)$ is a primitive Pythagorean triple, then either $a$ or $b$ must
    be a multiple of 3.
\end{prop}

\begin{proof}
    If 3 divides $a$ and $b$, then 3 divides $c$. But $a,b,c$ are coprime and
    so we have a contradiction. Hence, it cannot be the case that 3 divides $a$
    and $b$.
    
    Now suppose that $3$ does not divide $a$ or $b$. Then
    $a^2 \equiv 1 \pmod 3$ and $b^2 \equiv 1 \pmod 3$. Since $a^2+b^2 = c^2$, 
    we have that $c^2 \equiv 2 \pmod 3$. But the previous lemma says that this 
    is impossible, and so we have a contradiction.
    
    It follows that exactly one of $a,b$ is divisible by 3.
\end{proof}


% ------------------------------------------------------------------------------
% 2.2, page 19
% ------------------------------------------------------------------------------
\section*{2.2}

\begin{prop}
    Let $m,n, d$ be integers. If an $d$ divides both $m$ and $n$, then $d$ also
    divides $m-n$ and $m+n$.
\end{prop}

\begin{proof}
    Since $d \mid m$, there is an integer $k$ such that $m = dk$. Similarly,
    since $d \mid n$, there is an integer $l$ such that $n = dl$.
    
    Then $m+n = dk+dl = d(k+l)$. Hence $d \mid (m+n)$.
    
    Similarly, $m-n = dk-dl = d(k-l)$. Hence $d \mid (m-n)$.
\end{proof}


% ------------------------------------------------------------------------------
% 3.1, page 23
% ------------------------------------------------------------------------------
\section*{3.1}

\subsection*{3.1.(a)}
Suppose $\gcd(u,v) = d$ such that $d > 1$. Then we can write $u = dx$ and 
$v = dy$.

So:
\begin{align*}
    a &= u^2 - v^2          & b &= 2uv          & c &= u^2 + v^2\\
      &= d^2 x^2 - d^2 y^2  &   &= 2(dx)(dy)    &   &= d^2 x^2 + d^2 y^2\\
      &= d^2(x^2 - y^2)     &   &= d^2 (2xy)    &   &= d^2 (x^2 + y^2) \\
\end{align*}

Since $d^2$ divides $a$, $b$, and $c$, then $(a,b,c)$ is not a primitive
Pythagorean triple.

\subsection*{3.1.(b)}
If $u = 5$ and $v = 1$, $u$ and $v$ are coprime but $(24, 10, 26)$ is not 
primitive because all three values are even.

\subsection*{3.1.(c)}
Below is a table with all values of $u$ and $v$ when we substitute all values
of ${1 \leq v < u \leq 10}$.

\begin{tabular}{l l l l l}
    u   & v   & a   & b   & c   \\
    \hline
    2 & 1 & 3 & 4 & 5 \\
    3 & 1 & 8 & 6 & 10 \\
    3 & 2 & 5 & 12 & 13 \\
    4 & 1 & 15 & 8 & 17 \\
    4 & 2 & 12 & 16 & 20 \\
    4 & 3 & 7 & 24 & 25 \\
    5 & 1 & 24 & 10 & 26 \\
    5 & 2 & 21 & 20 & 29 \\
    5 & 3 & 16 & 30 & 34 \\
    5 & 4 & 9 & 40 & 41 \\
    6 & 1 & 35 & 12 & 37 \\
    6 & 2 & 32 & 24 & 40 \\
    6 & 3 & 27 & 36 & 45 \\
    6 & 4 & 20 & 48 & 52 \\
    6 & 5 & 11 & 60 & 61 \\
    7 & 1 & 48 & 14 & 50 \\
    7 & 2 & 45 & 28 & 53 \\
    7 & 3 & 40 & 42 & 58 \\
    7 & 4 & 33 & 56 & 65 \\
    7 & 5 & 24 & 70 & 74 \\
    7 & 6 & 13 & 84 & 85 \\
    8 & 1 & 63 & 16 & 65 \\
    8 & 2 & 60 & 32 & 68 \\
\end{tabular}
\quad
\begin{tabular}{l l l l l}
    u   & v   & a   & b   & c   \\
    \hline
    8 & 3 & 55 & 48 & 73 \\
    8 & 4 & 48 & 64 & 80 \\
    8 & 5 & 39 & 80 & 89 \\
    8 & 6 & 28 & 96 & 100 \\
    8 & 7 & 15 & 112 & 113 \\
    9 & 1 & 80 & 18 & 82 \\
    9 & 2 & 77 & 36 & 85 \\
    9 & 3 & 72 & 54 & 90 \\
    9 & 4 & 65 & 72 & 97 \\
    9 & 5 & 56 & 90 & 106 \\
    9 & 6 & 45 & 108 & 117 \\
    9 & 7 & 32 & 126 & 130 \\
    9 & 8 & 17 & 144 & 145 \\
    10 & 1 & 99 & 20 & 101 \\
    10 & 2 & 96 & 40 & 104 \\
    10 & 3 & 91 & 60 & 109 \\
    10 & 4 & 84 & 80 & 116 \\
    10 & 5 & 75 & 100 & 125 \\
    10 & 6 & 64 & 120 & 136 \\
    10 & 7 & 51 & 140 & 149 \\
    10 & 8 & 36 & 160 & 164 \\
    10 & 9 & 19 & 180 & 181 \\
\end{tabular}


\end{document}
