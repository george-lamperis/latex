\RequirePackage[l2tabu, orthodox]{nag}

\documentclass[letterpaper, 12pt]{article}

\usepackage[margin=1in]{geometry}
\usepackage{amsmath,amsthm,amssymb,mathtools}
\usepackage{array}              % required for tabular environment
\usepackage[parfill]{parskip}   % paragraph formatting
\usepackage{paralist}           % compactitem

% ------------------------------------------------------------------------------
% Theorem Commands
% ------------------------------------------------------------------------------
\newtheoremstyle{mystyle}% name
    {2ex}       % Space above
    {4ex}       % Space below
    {}          % Body font
    {}          % Indent amount (empty = no indent, \parindent = para indent)
    {\bfseries} % Thm head font
    {.}         % Punctuation after thm head
    {\newline}  % Space after thm head: \newline = linebreak
    {}          % Thm head spec

\theoremstyle{mystyle}
\newtheorem*{prop}{Proposition}
\newtheorem*{lemma}{Lemma}

% ------------------------------------------------------------------------------
% Begin Document
% ------------------------------------------------------------------------------
\title{Math 435 - Homework 1}
\author{George Lamperis}
\date{}

\begin{document}
\maketitle

% ------------------------------------------------------------------------------
% 2.1(a), page 18
% ------------------------------------------------------------------------------
\section*{2.1.(a)}

\begin{lemma}
    If $a$ is an integer, then either $a^2 \equiv 0 \pmod 3$ or 
    $a^2 \equiv 1 \pmod 3$.
\end{lemma}
\begin{proof}
    We have two cases: the case where 3 divides $a$, and the case where 3 does
    not divide $a$.

    If 3 divides $a$, then $a \equiv 0 \pmod 3$. It follows that 
    $a^2 \equiv 0 \pmod 3$ and so we are done.
    
    If 3 does not divide $a$, then either $a \equiv 1 \pmod 3$ or 
    $a \equiv 2 \pmod 3$. Since $2^2 \equiv 1 \pmod 3$, in either case we have
    that $a^2 \equiv 1 \pmod 3$.
\end{proof}


\begin{prop}
    If $(a,b,c)$ is a primitive Pythagorean triple, then either $a$ or $b$ must
    be a multiple of 3.
\end{prop}

\begin{proof}
    If 3 divides $a$ and $b$, then 3 divides $c$. But $a,b,c$ are coprime and
    so we have a contradiction. Hence, it cannot be the case that 3 divides $a$
    and $b$.
    
    Now suppose that $3$ does not divide $a$ or $b$. Then
    $a^2 \equiv 1 \pmod 3$ and $b^2 \equiv 1 \pmod 3$. Since $a^2+b^2 = c^2$, 
    we have that $c^2 \equiv 2 \pmod 3$. But the previous lemma says that this 
    is impossible, and so we have a contradiction.
    
    It follows that exactly one of $a,b$ is divisible by 3.
\end{proof}


% ------------------------------------------------------------------------------
% 2.2, page 19
% ------------------------------------------------------------------------------
\section*{2.2}

\begin{prop}
    Let $m,n, d$ be integers. If an $d$ divides both $m$ and $n$, then $d$ also
    divides $m-n$ and $m+n$.
\end{prop}

\begin{proof}
    Since $d \mid m$, there is an integer $k$ such that $m = dk$. Similarly,
    since $d \mid n$, there is an integer $l$ such that $n = dl$.
    
    Then $m+n = dk+dl = d(k+l)$. Hence $d \mid (m+n)$.
    
    Similarly, $m-n = dk-dl = d(k-l)$. Hence $d \mid (m-n)$.
\end{proof}


\end{document}