\documentclass[12pt]{article}

\usepackage{amsmath, amsthm, amssymb,amscd}
\usepackage{setspace}
\usepackage{enumerate}
\usepackage[margin=1in]{geometry}

\pagestyle{empty}
\doublespace

% -----------------------------------------------------------------------------
% Begin Document
% -----------------------------------------------------------------------------
\title{Math 300 - Essay 1}
\author{George Lamperis}
\date{}

\begin{document}
\maketitle

Suppose you want to find the slope of the function $f$ at a certain
point $(a, f(a))$. We could approximate the slope at this point by onsidering
the slope of the line passing through $(a, f(a))$ and $(a+1, f(a+1))$. If we 
want to be more accurate, we could consider the slope of the line passing 
through $(a, f(a))$ and $(a+h, f(a+h))$, with $0 < h < 1$. The slope of this
line is given by 
$$
\frac{f(a+h)-f(a)}{(a+h) - a} = \frac{f(a+h)-f(a)}{h}
$$
In general, the smaller $c$ is, the more accurate the approximation becomes.

The derivative $f'(x)$ tells you the instaneous rate of change at the point 
$(a, f(a))$. We calculate the derivative $f'(x)$ by considering the points 
$(a+h, f(a+h))$ by choosing an infinitesimally small positive value for $h$. In 
other words, we are finding the line which passes through two points which are 
incredibly close to each other. Thus to find $f'(x)$ at any point:
$$
f'(x) = \lim_{h \rightarrow 0} \frac{f(x+h) - f(x)}{h}
$$

\end{document}
