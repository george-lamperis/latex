\documentclass[12pt]{article}

\usepackage{amsmath, amsthm, amssymb,amscd}
\usepackage{setspace}
\usepackage{enumerate}
\usepackage[margin=1in]{geometry}

\pagestyle{empty}
%\doublespace

% -----------------------------------------------------------------------------
% Begin Document
% -----------------------------------------------------------------------------
\title{Math 300 - Essay 1}
\author{George Lamperis}
\date{}

\begin{document}
\maketitle

Suppose you want to find the slope of the function $f(x) = x^2$ at a given
point. This is precisely what the derivative is for. The derivative tells you
the slope of a line passing through a given point.

A naive approach would be to calculate the average slope between two points on a
curve. As we reduce the distance of these points, we get a better approximation.
In fact, the derivative is the average slope between two points that are
infintesimally close.

\end{document}
