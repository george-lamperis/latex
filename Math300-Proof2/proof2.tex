\documentclass[12pt]{article}

\usepackage{amsmath, amsthm, amssymb,amscd}
\usepackage{setspace}
\usepackage{enumerate}
\usepackage[margin=1in]{geometry}


\newtheoremstyle{mystyle}% name
    {2ex}       % Space above
    {2ex}       % Space below
    {}          % Body font
    {}          % Indent amount (empty = no indent, \parindent = para indent)
    {\bfseries} % Thm head font
    {.}         % Punctuation after thm head
    {\newline}  % Space after thm head: \newline = linebreak
    {}          % Thm head spec

\theoremstyle{mystyle}

\newtheorem{thm}{Theorem}
\newtheorem{defn}[thm]{Definition}

\pagestyle{empty}
\doublespace
\raggedright

% ------------------------------------------------------------------------------
% Begin Document
% ------------------------------------------------------------------------------
\title{Math 300 - Proof 2}
\author{George Lamperis}
\date{}

\begin{document}
\maketitle

\begin{defn}[Continuity]
  Let $f$ be defined on some interval $I$. We say that $f$ is continuous at a
  point $c \in I$ if $\lim_{x \to c} f(x) = f(c)$.
\end{defn}


\begin{defn}[Differentiability]
  Let $f$ be a function defined on some interval $I$. Let $c \in I$ and
  suppose that $c$ is not an endpoint. We say that $f$ is differentiable at $c$
  if 
  $$ \lim_{x \to c} \frac{f(x)-f(c)}{x-c}$$
  exists. We call this limit the derivative of $f$, denoted by $f'$.
\end{defn}


\begin{thm}
  Let $f$ be a function defined on an interval $I$. Let $c \in I$ and suppose
  that $f$ is differentiable at $c$. Then $f$ is continuous at $c$. 
\end{thm}
\begin{proof}
  Begin by cleverly multiplying by 1. Then use the product rule for limits.
  \begin{align*}
    \lim_{x \to c} f(x)-f(c) &= \lim_{x \to c} \frac{f(x)-f(c)}{x-c} \cdot (x-c) \\
                             &= \lim_{x \to c} f'(x) \cdot (x-c) \\   
                             &= \lim_{x \to c} f'(x) \cdot \lim_{x \to c} x-c \\   
                             &= \lim_{x \to c} f'(x) \cdot 0 \\   
                             &= 0
  \end{align*}
  
  Clearly, $\lim_{x \to c} f(x) = f(c)$. Therefore, $f$ is continuous at $c$.
\end{proof}


\end{document}