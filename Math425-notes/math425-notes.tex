\documentclass[12pt]{article}

\usepackage{amsmath, amsthm, amssymb,amscd}
\usepackage{enumerate}

\raggedright

% ------------------------------------------------------------------------------
% Theorem Commands
% ------------------------------------------------------------------------------
\newtheoremstyle{mystyle}% name
    {2ex}       % Space above
    {2ex}       % Space below
    {}          % Body font
    {}          % Indent amount (empty = no indent, \parindent = para indent)
    {\bfseries} % Thm head font
    {.}         % Punctuation after thm head
    {\newline}  % Space after thm head: \newline = linebreak
    {}          % Thm head spec

\theoremstyle{mystyle}

\newtheorem{thm}{Theorem}[section]
\newtheorem{prop}[thm]{Proposition}
\newtheorem{cor}[thm]{Corollary}
\newtheorem{defn}[thm]{Definition}
\newtheorem{lemma}[thm]{Lemma}
\newtheorem{example}[thm]{Example}
\newtheorem{remark}[thm]{Remark}

% Number sets
\newcommand{\N}{\mathbb{N}}
\newcommand{\Z}{\mathbb{Z}}
\newcommand{\Q}{\mathbb{Q}}
\newcommand{\R}{\mathbb{R}}
\newcommand{\C}{\mathbb{C}}

% Matrix sets
\newcommand{\F}{\mathbb{F}}
\newcommand{\Mat}[2]{\text{Mat}_{#1 x #2}}


% ------------------------------------------------------------------------------
% Begin Document
% ------------------------------------------------------------------------------
\title{Math 425 - Notes}
\author{George Lamperis}
\date{}

\begin{document}
\maketitle

\section{Basic Theory}

\begin{remark}
  Multiplication of two matrices is defined only if the number of columns of the
  left matrix is the same as the number of rows of the right matrix. If 
  $A \in \Mat{m}{n}$ and $B \in \Mat{n}{p}$ then their matrix product AB is the
  m-by-p matrix whose entries are given by dot product of the corresponding row
  of A and the corresponding column of B:
  \begin{align*}
  (AB)_{i,j} &= A_{i,1}B_{1,j} + A_{i,2}B_{2,j} + \ldots +  A_{1,n}B_{n,j} \\
             &= \sum_{r=1}^n A_{i,r}B_{r,j} \\
  \end{align*}
  
\end{remark}


\begin{defn}[Vector Space]
\end{defn}

\begin{defn}[Linear Map]
\end{defn}

\begin{defn}[Linear Operator]
\end{defn}

\begin{defn}[L-Invarient]
\end{defn}


\end{document}