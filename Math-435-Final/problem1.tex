% ------------------------------------------------------------------------------
% Problem 1
% ------------------------------------------------------------------------------
\section{}

For parts (a-i) express your answer as congruence conditions.
\begin{enumerate}[(a)]
\item Find all primes $p$ such that $(-1/p) = +1$. 

lets see. what happens
\item Find all primes $p$ such that $(-1/p) = -1$. 
\item Find all primes $p$ such that $(2/p) = +1$.
\item Find all primes $p$ such that $(2/p) = -1$.
\item Find all primes $p$ such that $(3/p) = +1$.
\item Find all primes $p$ such that $(3/p) = -1$.
\item Find all primes $p$ such that $(-6/p) = +1$. 
\item Find all primes $p$ such that $(-6/p) = -1$. 
\item Find all primes $p$ such that the equation
$3x^2 + 6x + 5 \equiv 0 \pmod{p}$ is solvable.
\end{enumerate}

Hints.  First assume your prime p is larger that 5; check primes 2, 3, and 5 
separately.  For (a) and (b) use Euler's formula for the quadratic residue 
symbol. There is a formula for (2/p). The p's you find will be given by
congruence conditions. For (e) and (f) use Quadratic Reciprocity. For (g) and 
(h) use multiplicativity of the quadratic symbol, and the Chinese Remainder 
Theorem.


\subsection{1(a,b)}
Theorem 22.1 states that if $p$ is an odd prime, then
\[
\left( \frac{-1}{p} \right) =
\begin{cases}
	1  & \text{if } p \equiv 1 \pmod{4} \\
	-1 & \text{if } p \equiv 3 \pmod{4}
\end{cases}
\]

There is one remaining case, $p=2$. Notice $-1 \equiv 1 \pmod 2$. Also notice
that $1^2 \equiv 1 \pmod 2$, and so $1$ is a quadratic residue modulo 2. Hence
$\left( \dfrac{-1}{2} \right) = 1$.

Hence, $\left( \dfrac{-1}{p} \right)  = 1$ when $p \equiv 0 \pmod 2$ or 
$p \equiv 1 \pmod 4$ and 
$\left( \dfrac{-1}{p} \right)  = -1$ when $p \equiv 3 \pmod 4$.


\subsection{1(c,d)}
This also follows from Theorem 22.1. The only remaining case is $p=2$. But 
$2 \equiv 0 \pmod{2}$, and 0 is neither quadratic residue nor a nonresidue, and
so this is undefined.

Hence, $\left( \dfrac{-1}{p} \right)  = 1$ if $p \equiv \pm 1 \pmod{8}$
and $\left( \dfrac{-1}{p} \right)  = -1$ if $p \equiv \pm 3 \pmod{8}$

\subsection{1(e,f)}
\begin{lemma}
Let $p$ be an odd prime, $p \neq 3$. Then
\[
\left( \frac{3}{p} \right) =
\begin{cases}
	1  & \text{if } p \equiv \pm 1 \pmod{12} \\
	-1 & \text{if } p \equiv \pm 5 \pmod{12}
\end{cases}
\]
\end{lemma}
\begin{proof}
First note that since $p$ is odd, $p \equiv 1 \pmod 4$ or $p \equiv 3 \pmod 4$.
Also, since $p$ is prime and $p > 3$, either $p \equiv 1 \pmod 3$ or 
$p \equiv 2 \pmod 3$.

% Case 1
\textbf{Case 1.}
Suppose $p \equiv 1 \pmod 4$ and $p \equiv 1 \pmod 3$. This implies that
$p \equiv 1 \pmod{12}$.

Then using the Law of Quadratic Reciprocity
\[ 
\left( \frac{3}{p} \right) = 
\left( \frac{p}{3} \right) = 
\left( \frac{1}{3} \right) = 1 
\]

Since $1^2 = 1$.

% Case 2
\textbf{Case 2.}
Suppose $p \equiv 1 \pmod 4$ and $p \equiv 2 \pmod 3$. This implies that
$p \equiv 5 \pmod{12}$.

Then using the Law of Quadratic Reciprocity
\[ 
\left( \frac{3}{p} \right) = 
\left( \frac{p}{3} \right) = 
\left( \frac{2}{3} \right) = -1 
\]

% Case 3
\textbf{Case 3.}
Suppose $p \equiv 3 \pmod 4$ and $p \equiv 1 \pmod 3$. This implies that
$p \equiv 7 \equiv -5 \pmod{12}$.

Then using the Law of Quadratic Reciprocity
\[ 
\left( \frac{3}{p} \right) = 
- \left( \frac{p}{3} \right) = 
- \left( \frac{1}{3} \right) = -1 
\]

% Case 4
\textbf{Case 4.}
Suppose $p \equiv 3 \pmod 4$ and $p \equiv 2 \pmod 3$. This implies that
$p \equiv 11 \equiv -1 \pmod{12}$.

Then using the Law of Quadratic Reciprocity
\[ 
\left( \frac{3}{p} \right) = 
- \left( \frac{p}{3} \right) = 
- \left( \frac{2}{3} \right) = 1 
\]
\end{proof}

We still need to check 2 and 3. For $p=3$ we have $3 \equiv 0 \pmod{3}$, and 0
is neither quadratic residue nor a nonresidue, and so this is undefined. For
$p=2$, we have 
\[ \left( \frac{3}{2} \right) = \left( \frac{1}{2} \right) = 1 \]

Hence, $\left( \dfrac{3}{p} \right)  = 1$ if $p \equiv 0 \pmod 2 $ or $p
\equiv \pm 1 \pmod{12}$ and $\left( \dfrac{3}{p} \right)  = -1$ if 
$p \equiv \pm 5 \pmod{12}$


\subsection{1(g,h)}
\begin{lemma}
	Let $p$ be prime, $p > 5$. Then
\end{lemma}
\begin{proof}
We have some congruence conditions modulo 4, 8 and 12. Notice $lcm(4, 8,
12) = 24 = 8 \cdot 3$. Because of this, our congruence conditions will be modulo
24. We enumerate cases by considering $p$ modulo 3 and modulo 8, then use the
Chinese Remainder Theorem to solve $p$ modulo 24.

% Case 1
\textbf{Case 1.}
Suppose $p \equiv 1 \pmod{3}$ and $p \equiv 1 \pmod{8}$. Then 
$p \equiv 1 \pmod{24}$
\begin{align*}
\left( \frac{-6}{p} \right) 
&= \left( \frac{-1}{p} \right) \left( \frac{2}{p} \right) \left( \frac{3}{p} \right) \\
&= 1 \cdot 1 \cdot 1 \\
&= 1
\end{align*}

% Case 2
\textbf{Case 2.}
Suppose $p \equiv 1 \pmod{3}$ and $p \equiv 3 \pmod{8}$. Then \\
$p \equiv 17 \pmod{24}$
\begin{align*}
\left( \frac{-6}{p} \right) 
&= \left( \frac{-1}{p} \right) \left( \frac{2}{p} \right) \left( \frac{3}{p} \right) \\
&= 1 \cdot 1 \cdot -1 \\
&= -1
\end{align*}

% Case 3
\textbf{Case 3.}
Suppose $p \equiv 1 \pmod{3}$ and $p \equiv 5 \pmod{8}$. Then \\
$p \equiv 13 \pmod{24}$
\begin{align*}
\left( \frac{-6}{p} \right) 
&= \left( \frac{-1}{p} \right) \left( \frac{2}{p} \right) \left( \frac{3}{p} \right) \\
&= 1 \cdot -1 \cdot 1 \\
&= -1
\end{align*}

% Case 4
\textbf{Case 4.}
Suppose $p \equiv 1 \pmod{3}$ and $p \equiv 7 \pmod{8}$. Then \\
$p \equiv 7 \pmod{24}$
\begin{align*}
\left( \frac{-6}{p} \right) 
&= \left( \frac{-1}{p} \right) \left( \frac{2}{p} \right) \left( \frac{3}{p} \right) \\
&= -1 \cdot 1 \cdot -1 \\
&= 1
\end{align*}
\end{proof}
