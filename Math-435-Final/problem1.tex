% ------------------------------------------------------------------------------
% Problem 1
% ------------------------------------------------------------------------------
\section{}

For parts (a-i) express your answer as congruence conditions.
\begin{enumerate}[(a)]
\item Find all primes $p$ such that $(-1/p) = +1$. 
\item Find all primes $p$ such that $(-1/p) = -1$. 
\item Find all primes $p$ such that $(2/p) = +1$.
\item Find all primes $p$ such that $(2/p) = -1$.
\item Find all primes $p$ such that $(3/p) = +1$.
\item Find all primes $p$ such that $(3/p) = -1$.
\item Find all primes $p$ such that $(-6/p) = +1$. 
\item Find all primes $p$ such that $(-6/p) = -1$. 
\item Find all primes $p$ such that the equation
$3x^2 + 6x + 5 \equiv 0 \pmod{p}$ is solvable.
\end{enumerate}

Hints.  First assume your prime p is larger that 5; check primes 2, 3, and 5 
separately.  For (a) and (b) use Euler's formula for the quadratic residue 
symbol. There is a formula for (2/p). The p's you find will be given by
congruence conditions. For (e) and (f) use Quadratic Reciprocity. For (g) and 
(h) use multiplicativity of the quadratic symbol, and the Chinese Remainder 
Theorem.


\subsection{2(a)}
Theorem 22.1 states that if $p \equiv 1 \pmod 4$, then
$\left( \dfrac{-1}{p} \right)  = 1$ 

There is one remaining case, $p=2$. Notice $-1 \equiv 1 \pmod 2$. Also notice
that $1^2 \equiv 1 \pmod 2$, and so $1$ is a quadratic residue modulo 2. Hence
$\left( \dfrac{-1}{2} \right) = 1$.

Hence, $\left( \dfrac{-1}{p} \right)  = 1$ when $p = 2$ or $p \equiv 1 \pmod 4$.


\subsection{2(b)}
Theorem 22.1 states that if $p \equiv 3 \pmod 4$, then
$\left( \dfrac{-1}{p} \right)  = -1$. Nothing further is required. 