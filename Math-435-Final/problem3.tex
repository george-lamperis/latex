% ------------------------------------------------------------------------------
% Problem 3
% ------------------------------------------------------------------------------
\section{}

If $p > 5$ is a prime expressible as $a^2 + 5 b^2$, then $p \equiv 1 \pmod{20}$,
or $p \equiv 9 \pmod{20}$.

For example $29 = 3^2 + 5 \cdot 2^2$, and clearly $29 \equiv 9 \pmod{20}$. 

\begin{proof}
First some data:
\begin{table}[H]
\centering
\begin{subtable}[b]{.24\linewidth}
	\centering
	\begin{tabular}{c|c}
		$a$ & $a^2$ \\ \hline
		0 & 0 \\
		1 & 1 \\
		2 & 0 \\
		3 & 1 \\
	\end{tabular}
	\caption*{modulo 4}
\end{subtable}
\begin{subtable}[b]{.24\linewidth}
	\centering
	\begin{tabular}{c|c}
		$a$ & $a^2$ \\ \hline
		0 & 0 \\
		1 & 1 \\
		2 & 4 \\
		3 & 4 \\
		4 & 1 \\
	\end{tabular}
	\caption*{modulo 5}
\end{subtable}
\begin{subtable}[b]{.24\linewidth}
	\centering
	\begin{tabular}{c|c}
		$a$ & $a^2$ \\ \hline
		0 & 0 \\
		1 & 1 \\
		2 & 4 \\
		3 & 9 \\
		4 & 16 \\
		5 & 5 \\
		6 & 16 \\
		7 & 9 \\
		8 & 4 \\
		9 & 1 \\
	\end{tabular}
	\caption*{modulo 20}
\end{subtable}
\begin{subtable}[b]{.24\linewidth}
	\centering
	\begin{tabular}{c|c}
		$a$ & $a^2$ \\ \hline
		10 & 0 \\
		11 & 1 \\
		12 & 4 \\
		13 & 9 \\
		14 & 16 \\
		15 & 5 \\
		16 & 16 \\
		17 & 9 \\
		18 & 4 \\
		19 & 1
	\end{tabular}
	\caption*{}
\end{subtable}
\end{table}

First notice
\[ p = a^2 + 5b^2 \equiv a^2 + b^2 \pmod{4} \]
We know $p$ is odd, therefore exactly one of $a^2$ and $b^2$ is odd and the
other is even. Suppose $a^2$ is even. Then $a^2 \equiv 0 \pmod 4$ and
$b^2 \equiv 1 \pmod 4$, therefore $p \equiv 1 \pmod 4$. By a symmetric argument,
if $a^2$ is odd and $b^2$ is even, then we know $p \equiv 1 \pmod 4$.


Also note that $p = a^2 + 5b^2 \equiv a^2 \pmod 5$ i.e. $p$ is a quadratic
residue modulo 5. However, 1 and 4 are the only quadratic residues modulo 5.
Therefore, $p \equiv 1 \pmod 5$ or $p \equiv 4 \pmod 5$.

% Case 1
\begin{case}
$p \equiv 1 \pmod 4$ and $a^2 \equiv 1 \pmod 5$

Then
\begin{align*}
		  p &\equiv a^2 \pmod 5 \\
	 4k + 1 &\equiv 1 \pmod 5 \\
	     4k &\equiv 0 \pmod 5 \\
	      k &\equiv 0 \pmod 5
\end{align*}
So $k$ has the form $k=5m$. Substitute this back into our equation.
\begin{align*}
	p &= 4k + 1 \\
	  &= 4(5m) + 1 \\
	  &= 20m + 1
\end{align*}
Clearly $p \equiv 1 \pmod{20}$.
\end{case}

% Case 2 
\begin{case}
$p \equiv 1 \pmod 4$ and $a^2 \equiv 4 \pmod 5$

Then
\begin{align*}
		  p &\equiv a^2 \pmod 5 \\
	 4k + 1 &\equiv 4 \pmod 5 \\
	     4k &\equiv 3 \pmod 5 \\
	      k &\equiv 2 \pmod 5
\end{align*}
So $k$ has the form $k=5m+2$. Substitute this back into our equation.
\begin{align*}
	p &= 4k + 1 \\
	  &= 4(5m+2) + 1 \\
	  &= 20m + 9
\end{align*}
Clearly $p \equiv 9 \pmod{20}$.
\end{case}

