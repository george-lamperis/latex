% ------------------------------------------------------------------------------
% Problem 6
% ------------------------------------------------------------------------------
\section{}

If $p = 4q+1$ and $q = 3r + 1$ are prime numbers, show that $3$ is a primitive
root modulo $p$. Hint: What are the prime divisors of $p-1$? You will have to
use quadratic reciprocity.

\begin{lemma}
If $a$ is a primitive root modulo $p$, then $a$ is not a quadratic residue. 
\end{lemma}
\begin{proof}
Let $a$ be a primitive root modulo $p$. Then $a$ has order $p-1$.

If $a$ were a quadratic residue, then there is a $b$ such that 
$b^2 \equiv a \pmod p$. This implies that $(b^2)^k \equiv a^k$ for all $k > 1$.

Then $b^2$ has order $2(p-1)$, which is a contradiction. Hence, $a$ is not a
quadratic residue.
\end{proof}


Notice that 
\begin{align*}
p = 4q+1 &\equiv q + 1 \pmod 3 \\
   	 	 &\equiv 3r + 2 \pmod 3 \\
   	 	 &\equiv 2	   \pmod 3
\end{align*}

Now calculate
\begin{align*}
	\left( \frac{3}{p} \right) &= \left( \frac{p}{3} \right) \\
				 			   &= \left( \frac{2}{3} \right) \\
	 						   &= -1
\end{align*}
So 3 is a nonresidue. This is necessary but not sufficient for 3 to be a
primitive root.

Also calculate $\phi(p-1)$
\begin{align*}
	\phi(p-1) &= \phi(4q) \\
			  &= \phi(4) \phi(3r+1)	\\
			  &= 2 \cdot 3r \\
			  &= 6r
\end{align*}

Notice that there are $\phi(p-1)=6r$ primitive roots and $(p-1)/2 = 6r+2$ 
nonresidues. If we can find two numbers such that 1) these numbers are not
primitive roots modulo $p$ and 2) these numbers are nonresidues, then it
must be the case that 3 is a primitive root modulo $p$.

Unfortunately I can't find these numbers.