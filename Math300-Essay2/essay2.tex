\documentclass[12pt]{article}

\usepackage{amsmath, amsthm, amssymb,amscd}
\usepackage{setspace}
\usepackage{enumerate}
\usepackage[margin=1in]{geometry}


\newtheoremstyle{mystyle}% name
    {2ex}       % Space above
    {2ex}       % Space below
    {}          % Body font
    {}          % Indent amount (empty = no indent, \parindent = para indent)
    {\bfseries} % Thm head font
    {.}         % Punctuation after thm head
    {\newline}  % Space after thm head: \newline = linebreak
    {}          % Thm head spec

\theoremstyle{mystyle}

\newtheorem{thm}{Theorem}
\newtheorem{defn}[thm]{Definition}
\newtheorem{prop}[thm]{Proposition}
\doublespace

% ------------------------------------------------------------------------------
% Begin Document
% ------------------------------------------------------------------------------
\title{Math 300 - Essay 2}
\author{George Lamperis}
\date{}

\begin{document}
\maketitle

Zeno's Dichotomoy Paradox is stated as follows: Suppose a man wants to move from
point A to point B. Before he can reach point B, he must reach half the
distance. But before this can happen, the man must reach a quarter of the
distance, and before this, and eighth of the distance, and so on. We might
represent this as an infinite sequence:
$$ 
\{ \ldots, \frac{1}{16}, \frac{1}{8}, \frac{1}{4}, \frac{1}{2}, 1 \} 
$$
The paradox is this: how can we complete an infinite number of steps in a finite
amount of time? There is one other problem with this infinite sequence: there is
no first term in this sequence. By that logic, the man cannot even begin to
travel from point A to point B, since there is no first task to complete.

Mathematicians and philosophers continue to struggle with Zeno's
Dichotomy Paradox. However, if we rephrase the scenario slightly, Calculus
offers a possible solution to this paradox. Consider the following: to travel
from point A to point B, you must first travel half that distance. After that,
you must travel half of the remaining distance, and then half the remaining
distance again, and so on. You might express the total distance traveled
after $n$ of these steps with the partial sum
$$
\frac{1}{2} + \frac{1}{4} + \frac{1}{8} + \ldots + \frac{1}{2^n} = \sum_{i=1}^n \frac{1}{2^i} 
$$
This is a geometric series, and so using the ratio test we see that as $n$ tends
to infinity, this sum converges
$$
\sum_{i=1}^\infty \frac{1}{2^i} = 1
$$
Zeno's paradox insists that is impossible to complete an infinite number of
tasks in a finite amount of time. However, we can use Calculus to show
that is is not the case, as the sum of infinitely many terms can be
equal to a finite number.

\noindent \textbf{References:} \\
http://en.wikipedia.org/wiki/Zeno's\_paradoxes \\
http://youtu.be/u7Z9UnWOJNY

\end{document}
