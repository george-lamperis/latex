\documentclass[12pt]{article}

\usepackage{amsmath, amsthm, amssymb,amscd}
\usepackage{setspace}
\usepackage{enumerate}
\usepackage{hyperref}
\usepackage[margin=1in]{geometry}



\newtheoremstyle{mystyle}% name
    {2ex}       % Space above
    {2ex}       % Space below
    {}          % Body font
    {}          % Indent amount (empty = no indent, \parindent = para indent)
    {\bfseries} % Thm head font
    {.}         % Punctuation after thm head
    {\newline}  % Space after thm head: \newline = linebreak
    {}          % Thm head spec

\theoremstyle{mystyle}

\newtheorem{thm}{Theorem}
\newtheorem{defn}[thm]{Definition}
\newtheorem{prop}[thm]{Proposition}
\newtheorem{remark}[thm]{Remark}
\newtheorem{example}[thm]{Example}
\doublespace

% Number sets
\newcommand{\N}{\mathbb{N}}
\newcommand{\Z}{\mathbb{Z}}
\newcommand{\Q}{\mathbb{Q}}
\newcommand{\R}{\mathbb{R}}
\newcommand{\C}{\mathbb{C}}

% ------------------------------------------------------------------------------
% Begin Document
% ------------------------------------------------------------------------------
\title{Math 300 - Final Paper}
\author{George Lamperis}
\date{}

\begin{document}
\maketitle

\section{Introduction}
The number $e$ is extremely important to mathematics. According to Howard
Whitley Eves, ``the number $e$ is of eminent importance in mathematics,
alongside 0, 1, $\pi$ and $i$. \cite{wikipedia} '' The number $e$ is often
called Euler's number.

The number $e$ was discovered through the study of logarithms. The motivation
for the development of logarithms was an attempt to make multiplication easier
by turning it into addition instead. Several mathematicians would inadvertantly
discover the number $e$. In 1618, John Napier release an essay which is
considered to be the first work on logarithms. In this essay appears a table of 
logarithms to the base $e$. However, Napier was unaware of the number $e$
and it's connection to logarithms.  Several mathematicians would inadvertantly
discover the number $e$. In 1683, Jacob Bernoulli was studying compound interest. 
He tried to find the limit of $(1 + 1/n)^n$ as $n$ tends to infinity. Although
this limit is equal to $e$, Bernoulli likely did not realize the connection
between $e$ and the logarithm. \cite{USAS}

There is some debate over who was the first mathematician to discover the
connection between logarithms and $e$. However, Leonard Euler discovered some
remarkable properties of $e$ as well as it's connection to the logarithm, which
he published in \textit{Introductio in Analysin infinitorum}. It is because of
this that the number $e$ is named after Euler.


\section{Results}

\begin{defn}
  Define the function $L: (0, \infty) \mapsto \R$ as follows:
  $$ L(x) = \int_1^x \frac{dt}{t} $$
  
  The function $x \mapsto 1/x$ is continuous. Therefore, by the Fundamental
  Theorem of Calculus, $L$ is differentiable and
  $$ L^\prime(x) = \frac{1}{x} $$.
  
  Notice also that
  $$L(1) = 0$$
  
  Since $L'(x) > 0$ for all $x \in (0, \R)$,  we can deduce that $L(x)$ is
  strictly monotone increasing on its entire domain \cite[Theorem 4.13]{howie}.
  
  Additionally, since $L(x)$ is monotone increasing, we can apply the Integral
  Test, \cite[Theorem 5.37]{howie}. Since the series $\sum_1^n \frac{1}{x}$
  diverges, it follows that $L(x)$ diverges also.
\end{defn}


\begin{thm}
  For all $x,y \in (0, \infty)$,
  $$ L(xy) = L(x) + L(y) \text{ and } L(1/x) = -L(x)$$
\end{thm}
\begin{proof}
  This proof borrows heavily from \cite[Theorem 6.1]{howie}
  \begin{align*}
    L(xy) &= \int_{1}^{xy} \frac{dt}{t} \\
          &= \int_{1}^{x} \frac{dt}{t} + \int_{x}^{xy} \frac{dt}{t} \\
          &= L(x) + \int_{x}^{xy} \frac{dt}{t}
  \end{align*}
  Now in the remaining integral, substitute $u = t/x$. 
  \begin{align*}
          &= L(x) + \int_{1}^{y} \frac{du}{u} \\
          &= L(x) + L(y)
  \end{align*}
  as desired.
  
  The next property follows immediately from this:
  \begin{align*}
    L(x) + L(1/x) &= L(x \cdot 1/x) \\
                  &= L(1) \\
                  &= 0
  \end{align*}  

\end{proof}


\begin{thm}
  There is a unique number $e$ such that 
  $$ L(e) = \int_1^e \frac{dt}{t} = 1 $$
\end{thm}
\begin{proof}
  We know that $L(1) = 0$. Additionally, because $L(x)$ diverges, there is a $c$
  such that $L(c) > 1$. Now, apply the Intermediate Value Theorem and we have
  that there exists a number $e$, such that $1 < e < c$, and $L(x) = 1$. Because
  $L(x)$ is strictly monotone increasing, and thus bijective, it follows that
  $e$ is unique.
\end{proof}


\begin{defn}
  Since $L$ is differentiable and monotone increasing, there is a monotone
  increasing and differentiable inverse fuction $L^{-1}: \R \mapsto (0, \infty)$,
  which we will denote by $E$. And so we have
  \begin{align*}
    E(L(X)) = x \\
    L(E(x)) = x
  \end{align*}
  
  Note also that
  $$ E(0) = 1 \text{ and } E(1) = e $$
\end{defn}


\begin{thm}
  The function $E$ is its own derivative, i.e., $E^\prime(x) = E(x)$
\end{thm}
\begin{proof}
  This proof is taken from \cite[p. 168]{howie}:
  \begin{align*}
    E^\prime(x) 
        &= \frac{1}{L^\prime(L^{-1}(x))}  & \text{(Howie, Theorem 4.15)} \\
        &= L^\prime(x) \\
        &= E(x)
  \end{align*}
\end{proof}


\begin{thm}
  For all $x,y \in \R$, we have
  $$ E(x+y) = E(x)E(y) $$
  and
  $$ E(-x) = \frac{1}{E(x)}$$
\end{thm}
\begin{proof}
  This proof is taken from \cite[Theorem 6.2]{howie}
  Let $x,y \in \R$. Then
  \begin{align*}
    E(x+y) &= E[L(E(x)) + L(E(y))] \\
           &= E[L(E(x)E(y))] \\
           &= E(x)E(y)
  \end{align*}
  
  The second property follows since
  $$ E(x)E(-x) = E(x +(-x)) = E(0) = 1 $$
\end{proof}


From the previous theorem, we can deduce that
  $$E(x)^2 = E(x)E(x) = E(x+x) = E(2x)$$
or more generally, for a real number $n$,
  $$ E(x)^n = E(nx)$$.

Now let us gather this information and introduce a new notation.

\begin{thm}
  Define $e^x$ as $E(x)$ and also define $\log_e (y)$ (or equivalently $\ln y$)
  as $L(y)$.
  
  Now let us express what we know about $L$ and $E$ using this new notation.
  Recall that $E$ and $L$ are inverses. Thus,
  $$ e^{\ln x} = x \text{ and } \ln(e^x) = x $$
  
  We also have the following properties:
  \begin{align*}
    \int_{1}^x \frac{dt}{t} = \ln x \\
    \frac{d}{dx}(\ln x) = \frac{1}{x} \\
    \ln xy = \ln x + \ln y \\
    \ln(1/x) = - \ln x \\
    \frac{d}{dx}(e^x) = e^x \\
  \end{align*}
\end{thm}


Before this result, the function $a^x$ is only defined for rational numbers. The
following result defines $a^x$ for any real number $x$.
\begin{defn}
  If $a$ is a positive real number, we define $a^x$ as
  $$ a^x = e^{x \ln a} $$
  and
  $$ \log_a x = \frac{\ln x}{\ln a} $$
\end{defn}


\begin{thm}
  If $b > 0$ and $\frac{d}{dx} b^x = b^x$, then $b = e$.
\end{thm}
\begin{proof}
  \begin{align*}
    (b^x)' &= (e^{x \ln b})^\prime \\
           &= (e^{x \ln b}) \cdot (x \ln b)^\prime  \\
           &= b^x \cdot \ln b
  \end{align*}
  
  Hence, it must the case that $\ln b = 1$. Since $e$ is the unique number such
  that $\ln e = 1$, it follows that $b=e$.
\end{proof}


\begin{remark}
  If we apply the Taylor-Maclaurin Theorem to $e^x$, we have
  \begin{align*}
    e^x &= e^0 + xe^0 + \frac{x^2}{2!}e^0 + \frac{x^3}{3!} +  \ldots \\
        &= \sum_1^\infty \frac{x^n}{n!}
  \end{align*}
\end{remark}

\begin{thm}
  The number $e$ is irrational. 
\end{thm}
\begin{proof}
  This proof is taken from \cite[Example 6.10]{howie}
  Suppose for contradiction that $e = p/q$ for integers $p,q$. Let $n$ be an
  integer such that $n > max(q,3)$. It follows that $n!e$ is an integer. Now
  apply Taylor's Theorem:
  $$ e = 2 + \frac{1}{2!} + \ldots + \frac{1}{n!} + \frac{1}{(n+1)!}e^\theta $$
  where $0 < \theta < 1$. Hence
  $$ \frac{e^\theta}{n+1} = n!e - 2n! - \frac{n!}{2!} - \ldots - \frac{n!}{n!} $$
  is an integer. On the other hand, $1 < e^\theta < 3$, and so 
  $$ 0< \frac{e^\theta}{n_1} < \frac{3}{n+1} < 1 $$
  and so we have a contradiction. Therefore, we deduce that $e$ is irrational.
  
\end{proof}



\section{Practical uses}

\begin{example}
  Recall that when we apply the Taylor-Maclaurin Theorem to $e^x$, we have
  $$ e^x = \sum_1^\infty \frac{x^n}{n!} $$
  
  Because this series converges extremely fast, it provides us with an accurate
  way to approximate the value of $e$. Leonard Euler used this fact to give
  the approximation of
  $$ e = 2.718281828459045235 $$
  by taking 20 terms of the series. \cite{USAS}
\end{example}


\begin{example}
  Recall that the original motivation for the logarithm was to simplify the
  multiplication of large numbers. The multiplication algorithm taught in grade
  school works quite well for smaller numbers, but quickly becomes tedious and
  time consuming when the numbers get larger. 
  
  Before computers and calculators were available, base-10 logarithms were used
  as a way to approximate multiplication. To calculate the value of a
  logarithm, one would simply look up the value in a logarithm table, which
  commonly contained the values for $\log_{10} x$ for $0 < x \leq 10$ up to
  three decimal places of accuracy \cite[Remark 6.3]{howie}.
  
  Suppose we want to approximate 
  $$325.7 \times 48.43$$. 
  First write these numbers in scientific notation, then apply properties of
  the logarithm:
  \begin{align*}
    3.257 \cdot 10^2 \times 4.843 \cdot 10 
      &= 10^{\log(3.257 \times 10 ^2)} \cdot  10^{\log(4.843 \cdot 10)} \\
      &= 10^{\log 3.257 + \log 10^2} \cdot  10^{\log 4.843 + \log 10} \\
      &= 10^{\log 3.257 + 2} \cdot  10^{\log 4.843 + 1} \\
      &= 10^{4.1979} \\
  \end{align*}
  
  Now perform a reverse lookup to evalauate $10^4.1979 = 15770$. The exact
  answer is $15773.651$, but this approximation is quite accurate and sufficient
  for practical applications.
  
  Using this method, we are able to get a quite accurate approximation of the
  product by performing three table lookups and one addition. This is quite an
  impovement in efficency over the standard multiplication algorithm.
\end{example}


% ------------------------------------------------------------------------------
% References
% ------------------------------------------------------------------------------
\begin{thebibliography}{10}

\bibitem{howie}
John M. Howie, Real Analysis

\bibitem{SWAU}
\url{http://turner.faculty.swau.edu/mathematics/math110de/materials/logtable/}

\bibitem{USAS}
\url{http://www-history.mcs.st-and.ac.uk/HistTopics/e.html}

\bibitem{sliderule}
\url{http://riker.ps.missouri.edu/RicksPage/SlideRule02/index.html}

\bibitem{wikipedia}
\url{http://en.wikipedia.org/wiki/E_(mathematical_constant)}

\end{thebibliography}


\end{document}