\documentclass[12pt]{article}

\usepackage{amsmath, amsthm, amssymb,amscd}
\usepackage{setspace}
\usepackage{enumerate}
\usepackage{hyperref}
\usepackage[margin=1in]{geometry}



\newtheoremstyle{mystyle}% name
    {2ex}       % Space above
    {2ex}       % Space below
    {}          % Body font
    {}          % Indent amount (empty = no indent, \parindent = para indent)
    {\bfseries} % Thm head font
    {.}         % Punctuation after thm head
    {\newline}  % Space after thm head: \newline = linebreak
    {}          % Thm head spec

\theoremstyle{mystyle}

\newtheorem{thm}{Theorem}
\newtheorem{defn}[thm]{Definition}
\newtheorem{prop}[thm]{Proposition}
\newtheorem{remark}[thm]{Remark}
\newtheorem{example}[thm]{Example}
%\doublespace

% Number sets
\newcommand{\N}{\mathbb{N}}
\newcommand{\Z}{\mathbb{Z}}
\newcommand{\Q}{\mathbb{Q}}
\newcommand{\R}{\mathbb{R}}
\newcommand{\C}{\mathbb{C}}

% ------------------------------------------------------------------------------
% Begin Document
% ------------------------------------------------------------------------------
\title{Math 300 - Final Paper}
\author{George Lamperis}
\date{}

\begin{document}
\maketitle

% \begin{defn}
%   Let $e = \lim_{n \to \infty} \left( 1 + \frac{1}{n}\right)^n $
% \end{defn}
% \begin{proof}
% 
% \end{proof}
% 
% 
% \begin{defn}
%   Let $\ln x = \log_e(x)$
% \end{defn}
% \begin{proof}
% 
% \end{proof}
% 
% 
% \begin{prop}
%   $ \frac{d}{dx} \ln x = \frac{1}{x} $
% \end{prop}
% \begin{proof}
% This proof is not original. \cite{proofwiki1}
% 
% \begin{align*}
%     \frac{d}{dx} \ln x 
%         &= \lim_{h \to 0} \frac{\ln(x+h) - \ln(x)}{h} \\
%         &= \lim_{h \to 0} \frac{1}{h} \cdot \ln \left(\frac{x+h}{x} \right) \\
%         &= \lim_{h \to 0} \frac{1}{h} \cdot \ln \left(1 + \frac{h}{x} \right)
%         \\
% \end{align*}
% 
% Next let $u = \lvert \frac{x}{h} \rvert$
% 
% 
% \end{proof}

Provide clarification for $E,L$ notation.

\begin{defn}
  Define the function $L(x)$ as follows:
  $$ L(x) = \int_1^x \frac{dt}{t} $$
  for $x > 0$.
  
  
  
  The function $x \mapsto 1/x$ is continuous. Therefore, by the Fundamental
  Theorem of Calculus, $L$ is differentiable and
  $$ L^\prime(x) = \frac{1}{x} $$.
\end{defn}



\begin{thm}
  There is a unique number $e$, which lies between 2 and 3, such that 
  $$ L(e) = \int_1^e \frac{dt}{t} = 1 $$
\end{thm}

\begin{defn}
  Since $L$ is differentiable and monotone increasing, there is a monotone
  increasing and differentiable inverse fuction $L^{-1}: \R \mapsto (0, \infty)$,
  which we will denote by $E$. And so we have
  \begin{align*}
    E(L(X)) = x \\
    L(E(x)) = x
  \end{align*} 
\end{defn}


\begin{thm}
  The function $E$ is its own derivative, i.e., $E^\prime(x) = E(x)$
\end{thm}
\begin{proof}
  \begin{align*}
    E^\prime(x) 
        &= \frac{1}{L^\prime(L^{-1}(x))}  & \text{(Howie, Theorem 4.15)} \\
        &= L^\prime(x) \\
        &= E(x)
  \end{align*}
\end{proof}


\begin{thm}
  For all $x,y \in \R$, we have
  $$ E(x+y) = E(x)E(y) $$
  and
  $$ E(-x) = \frac{1}{E(x)}$$
\end{thm}

Now let us gather this information and introduce a new notation.

\begin{thm}
  Define $e^x$ as $E(x)$ and also define $\log_e (y)$ as $L(y)$. 
  
  If $y = e^x$, then 
  
  Now let us express what we know about $L$ and $E$ using this new notation.
  Recall that $E$ and $L$ are inverses. Thus,
  $$ e^{\ln x} = x \text{ and } \ln(e^x) = x $$
  We also have the following properties:
  \begin{align*}
    \int_{1}^x \frac{dt}{t} = \ln x \\
    \frac{d}{dx}(\ln x) = \frac{1}{x} \\
    \frac{d}{dx}(e^x) = e^x \\
  \end{align*}
    
\end{thm}


\begin{defn}
  If $a$ is a positive real number, we define
  $a^x = e^{x \log a}$
\end{defn}


Interesting fact: Bernoulli used this to approximate the value of $e$.
\begin{prop}
  $$ \lim_{n \to \infty} \left( 1 + \frac{x}{n} \right)^n = e^x $$
\end{prop}


\begin{thm}
  The number $e$ is irrational. 
\end{thm}


Maybe include some old-fashioned tricks for evaluating numbers with logs.

% ------------------------------------------------------------------------------
% References
% ------------------------------------------------------------------------------
\begin{thebibliography}{10}

\bibitem{howie}
John M. Howie, Real Analysis

\bibitem{proofwiki1}
\url{http://www.proofwiki.org/wiki/Derivative_of_Natural_Logarithm_Function/Proof_2}

\end{thebibliography}


\end{document}