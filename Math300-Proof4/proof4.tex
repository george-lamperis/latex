\documentclass[12pt]{article}

\usepackage{amsmath, amsthm, amssymb,amscd}
\usepackage{setspace}
\usepackage{enumerate}
\usepackage[margin=1in]{geometry}


\newtheoremstyle{mystyle}% name
    {2ex}       % Space above
    {2ex}       % Space below
    {}          % Body font
    {}          % Indent amount (empty = no indent, \parindent = para indent)
    {\bfseries} % Thm head font
    {.}         % Punctuation after thm head
    {\newline}  % Space after thm head: \newline = linebreak
    {}          % Thm head spec

\theoremstyle{mystyle}

\newtheorem{thm}{Theorem}
\newtheorem{defn}[thm]{Definition}
\newtheorem{prop}[thm]{Proposition}
\doublespace

% ------------------------------------------------------------------------------
% Begin Document
% ------------------------------------------------------------------------------
\title{Math 300 - Proof 4}
\author{George Lamperis}
\date{}

\begin{document}
\maketitle

\begin{prop}
  If $n > 0$, then 
  $$
    \frac{(2n)!}{n!^2}
  $$
  is an even integer.
\end{prop}
\begin{proof}
We will induct on $n$. When $n=1$, we have
$$
  \frac{(2 \cdot 1)!}{1!^2} = 2
$$
Assume for inductive hypothesis that for some integer $k$ we have
$$ 
  \frac{(2n)!}{n!^2} = 2k
$$
Then
\begin{align*}
  \frac{(2(n+1))!}{(n+1)!^2}
    &= \frac{(2n+2)!}{(n+1)!^2} \\
    &= \frac{(2n)! \cdot (2n+1)(2n+2)}{(n)!^2 (n+1)^2} \\
    &= \frac{(2n)!}{n!^2} \cdot \frac{(2n+1)(2n+2)}{(n+1)^2} \\
    &= 2k \cdot \frac{(2n+1)(2n+2)}{(n+1)^2}
\end{align*}
is even. Hence, if $n > 0$
$$
  \frac{(2n)!}{n!^2}
$$
is even.
\end{proof}

\end{document}