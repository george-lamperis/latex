\documentclass[12pt]{article}

\usepackage{amsmath, amsthm, amssymb,amscd}
\usepackage{setspace}
\usepackage{enumerate}
\usepackage[margin=1in]{geometry}


\newtheoremstyle{mystyle}% name
    {2ex}       % Space above
    {2ex}       % Space below
    {}          % Body font
    {}          % Indent amount (empty = no indent, \parindent = para indent)
    {\bfseries} % Thm head font
    {.}         % Punctuation after thm head
    {\newline}  % Space after thm head: \newline = linebreak
    {}          % Thm head spec

\theoremstyle{mystyle}

\newtheorem{thm}{Theorem}
\newtheorem{defn}[thm]{Definition}
\newtheorem{prop}[thm]{Proposition}
\doublespace

% ------------------------------------------------------------------------------
% Begin Document
% ------------------------------------------------------------------------------
\title{Math 300 - Research Essay 1}
\author{George Lamperis}
\date{}

\begin{document}
\maketitle


Alan Turing is perhaps the founder of computer science. He invented the current
model which we use to understand a computer, called a Turing machine. 

\section{Computing}

Although computers were still in their infancy, Alan Turing had an enourmous
impact on computing and computer science. An extremely important problem which
he worked on was The Halting Problem. The Halting Problem is stated as follows:
Given a program and an input, decide whether the program eventually finishes or
if the program will run forever. This is important because a program which runs
forever may not be very useful.

A key part of solving this problem would be to develop precise definitions of a
computer and of a computer program. This would lead Turing to develop a
theoretical model of a computer which we now call a Turing machine. 

``an unlimited memory capacity obtained in the form of an infinite tape marked out
into squares, on each of which a symbol could be printed. At any moment there is
one symbol in the machine; it is called the scanned symbol. The machine can
alter the scanned symbol and its behavior is in part determined by that symbol,
but the symbols on the tape elsewhere do not affect the behavior of the machine.
However, the tape can be moved back and forth through the machine, this being 
one of the elementary operations of the machine.'' Computer scientists today
continue to use this model to study computers.

Primitive computing machinery was designed to perform just one task or
calculation. The idea of the Universal Turing Machine is that \emph{any}
possible Turing machine could be represented as a stored program. With stored
programs, one machine could run any possible program. Turing also thought of
Abbreviated Code Instructions, a system where a computer could expand a program
from an abbreviated form. This is essentially the plans for the first
programming language. ACI would allow a computer to be programmed faster and
more easily.

After the end of World War II, Turing set out to develop the Universal Turing
Machine that he had envisioned. However a radar engineer named F. C. Williams
would beat him, developing the first implementation of a Turing machine in June
1948.


\section{World War II}

Prior to World War II, German engineers had developed the Enigma machine, a
mechanical and electrical device used for enciphering written messages.
The Enigma machine had 3 to 5 rotors which allowed the user to select one of
an extremely large number of settings. Each setting would result in a different
code, and so two users who wish to send messages to each other must be
using the same settings. During World War II, the German military made heavy use
of the Enigma machine to securely pass messages between their troops.

Clearly, the Allies would have a
great advantage if these messages would be deciphered, and Alan Turing and
several other mathematicians worked to break the Enigma code. Polish
mathematicians had made some progress; however their work was incomplete, as it
could not break the German Navy's Enigma code. Alan Turing made several key
observations that led to the breaking of the enigma code. One of these
observations was that the Enigma code would never encipher a letter to itself.
So for example, an ``a'' in the original message would never appear as ``a'' in 
the enciphered message. Turing also discovered an algorithm which would greatly
reduced the number of possiblities to be checked. After he had developed the
algorithm, Turing built the Bombe machine, a machine which carried out his
algorithm much faster than could be performed by hand. The Bombe machine could
work out the correct settings for the Enigma in about 20 minutes.

With the Enigma code broken, the Allies now had a huge advantage. They could
read the Nazi's messages which often included their military tactics and plans. 
This would prove to be crucial in several operations, such as D-Day. Prior to
D-Day, the Allies set a trap; they distributed messages containing plans for an
attack several miles away from Normandy Beach. Because the Enigma had been
cracked, the Allies were able to confirm that the Nazi's had fallen for their
trap. The Nazi's positioned troops at the site of the dummy invasion, while the
Allied troops stormed the beaches of Normandy instead.

Another significant problem which Turing worked on during World War II was The
German Tank Problem. The Allies wished to know how many tanks were posessed by
the German military. This was a difficult problem because we could only know how
many tanks were seen on the battlefield. Using only the number of tanks spotted
by Allied forces, Turing was able to estimate the total number of tanks which
the German military owned.

\section{Development of the Computer}


http://www.turing.org.uk/

\begin{thebibliography}{9}

\bibitem{numberphile1}
http://youtu.be/BdrrJ7qd4HA,
Numberphile

\end{thebibliography}
\end{document}
