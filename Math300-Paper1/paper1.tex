\documentclass[12pt]{article}

\usepackage{amsmath, amsthm, amssymb,amscd}
\usepackage{setspace}
\usepackage{enumerate}
\usepackage[margin=1in]{geometry}


\newtheoremstyle{mystyle}% name
    {2ex}       % Space above
    {2ex}       % Space below
    {}          % Body font
    {}          % Indent amount (empty = no indent, \parindent = para indent)
    {\bfseries} % Thm head font
    {.}         % Punctuation after thm head
    {\newline}  % Space after thm head: \newline = linebreak
    {}          % Thm head spec

\theoremstyle{mystyle}

\newtheorem{thm}{Theorem}
\newtheorem{defn}[thm]{Definition}
\newtheorem{prop}[thm]{Proposition}
\doublespace

% ------------------------------------------------------------------------------
% Begin Document
% ------------------------------------------------------------------------------
\title{Math 300 - Research Essay 1}
\author{George Lamperis}
\date{}

\begin{document}
\maketitle

Alan Turing is a British mathematician, well known for his enormous influence on
the field of Computer Science. Turing was born born on June 23, 1912 in London,
England. At a young age, Alan Turing showed an interest in the sciences. He
studied the works of Albert Einstein in his free time, at the age of only 16.
Turing attended King's College in Cambridge from 1931 to 1934, where he studied
mathematics. 

\section*{The Halting Problem}
At this time, Kurt Godel had made a shocking development in
mathematics. In 1931, Godel's Incompleteness Theorem showed a shocking result:
that there are true statementments about numbers which can not be proven from
basic axioms. A further question was posed by David Hilbert, known as the
Entscheidungsproblem: does there exist some method of determining whether a 
given mathematical statement was provable? The Entscheidungsproblem would serve
as a motivation for the Halting Problem. The Halting Problem is stated as
follows: Given a program and an input, decide whether the program eventually 
finishes or if the program will run forever. This is important because a program
which runs forever may not be very useful. In 1936, Alan Turing proved that, in 
general, there is not an algorithm to determine whether or not a program-input 
pair will halt. At the same time, Turing had also shown that a general solution
to Hilbert's Entscheidungsproblem does not exist. It's worth noting that Alonzo
Church had independently proved the same results as Turing in 1936. However,
Turing's ideas were different and original enough to have a wider impact than
Church's work.

A key part of solving this problem was to develop precise definitions of a
computer and of a computer program. This would lead Turing to develop a
theoretical model of a computer program which we now call a Turing machine. A
Turing machine consists of:
\begin{itemize}
  \item A finite set of symbols.
  \item An infinitely long roll of tape which is divided into squares. Each
    square may be empty, or it may have a symbol printed on it. The tape may be
    moved forward and backwards.
  \item An active square on the, known as the ``head'' or the ``scanned
    symbol''. The head symbol maybe be read or a new symbol written to the tape.
  \item A state register which keeps track of the state of the Turing machine.
  \item A table of instructions which, given the state and the symbol written on
    the scanned symbol, specifies how to manipulate the state and/or active
    symbol.
\end{itemize}
This model differed from Church's because it resembles a mechanical device. This
idea proves to be extremely useful precisely because it could be implemented as
a physical machine. Additionally, Alan Turing created the concept of the
Universal Turing Machine. The Universal Turing Machine is a Turing machine which
can simulate any arbitrary Turing machine. We can think of the Universal Turing
Machine as a computer, while a Turing machine is a computer program. 

These models may seem quite primitive; however, they prove to be extremely
useful. The Universal Turing Machine inspired the concept of the stored program.
Early computers were built to perform just one task. This required building a
new computer for each task you would like to automate. To contrast this, a
computer with stored programs allows you to build one computer, and perform a
myriad of different tasks. Additionally, computer scientists today continue to
use Turing machines to understand the capabilities and limitations of computers.
A great deal of tasks and problems may be solved using Turing machines, and so
computer scientists find that studying Turing machines is quite useful. Most
programming languages are Turing complete, meaning that the language can
simulate the Universal Turing Machine.

After the end of World War II, Turing set out to actually implement the
Universal Turing Machine that he had envisioned. However a radar engineer named 
F. C. Williams would beat him, developing the first implementation of a 
Turing machine in June 1948. Nevertheless, Alan Turing's ideas made an enormous
impact on the field of computing.



\section{World War II}

% This is all numberphile
Besides his breakthroughs in Computer Science, Alan Turing also made a
remarkable impact on the Allied World War II effort. Prior to World War II,
German engineers had developed the Enigma machine, a mechanical and electrical 
device used for enciphering written messages. The Enigma machine had 3 to 5
rotors, each with 26 settings, allowing for an extremely large number of
settings. Additionally, the Enigma machine has a plugboard on the back which 
allows even more possible configurations. The plugboard swaps two letters, and 
up to 10 pairs could be used. Each setting creates a unique code. The security 
of the Enigma code relies on the fact that there are an astronomically large 
amount of possible settings, since the sender and recipient must use the same
settings.

During World War II, the German military made heavy use of the Enigma machine to
securely pass messages between their troops. If the Allies were able to decipher
the German military's communication, then clearly the Allies would have a great
advantage. Because of this, the British and Polish militaries made it a priority
to crack the Enigma code. Polish mathematicians had worked out the electrical
wiring of the Enigma, leading to a good understanding of how it worked. 
However, the challenge lies in the settings on the plugboard.

Alan Turing made several key observations that led to the breaking of the Enigma
code. One of these observations was that the Enigma code would never encipher a 
letter to itself. So for example, an ``a'' in the original message would never
appear as ``a'' in the enciphered message. Turing realized that this was a flaw
in the code. This fact would be a key part of breaking the code.

Turing eventually developed an algorithm which could crack the Enigma code. The 
algorithm starts by choosing one Enigma setting. Assume this is the correct
setting, and begin deciphering until a contradiction is found: either a letter
goes to itelf, or two letters go to the same letter. When a contradiction is
devived, we can deduce that this setting is wrong. Alan Turing also realized
that anything following from this contradiction is also false, which allowed
him to drastically reduce the number of settings which must be checked. However,
even with the discovery of this algorithm, an enormous number of settings still
had to be checked, which was extremely time consuming. Breaking the code quickly
was extremely important, since the Germans changed the settings of their Enigma
machines every day. Because of this, Turing and other British mathematicians
built the Bombe machine, a machine which performed all of these checks using 
switches and electrical circuits. The Bombe machine could calcuate the correct
settings for the enigma machine in just 20 minutes, much faster than it could be
calculated by hand.

With the Enigma code broken, the Allies now had a huge advantage. They could
read the Nazi's messages which often included their military tactics and plans.
This would prove to be crucial in several important Allied operations, such as
D-Day. Prior to D-Day, the Allies had set a trap; they distributed messages
containing false plans for an attack several miles away from Normandy Beach. Because the
Enigma had been cracked, the Allies were able to confirm that the German
military  had fallen for their trap. The Germans positioned troops at the site
of the dummy invasion, while the Allied troops stormed the beaches of Normandy instead.


Despite all of Alan Turing's accomplishments and all that he did for his
country, Turing's life has a tragic ending. At this time, homosexual acts were
considered a crime in the United Kingdom. In 1952, Alan Turing was arrested and
charged with gross indecency when police learned of his sexual relationship
with another man. The courts gave him the choice between prison and chemical
castration. Turing opted for chemical castration, and recieved injections of
hormones intended to limit his sexual thoughts. Additionally, Turing's security
clearance was revoked, bringing an end to his work in Crytography for the
British government. Because of this, Turing became resentful and depressed. In
1954, at age 41, Alan Turing was found dead in his home. The cause of death was
determined to be suicide by means of cyanide poisening. In 2009, Prime Minister
Gordon Brown issued a formal apology to Alan Turing, and in 2013, Turing was 
pardoned for his charges by Queen Elizabeth II.


\begin{thebibliography}{9}

\bibitem{numberphile1}
http://youtu.be/BdrrJ7qd4HA,
Numberphile

\bibitem{}
http://en.wikipedia.org/wiki/G\%C3\%B6del's\_incompleteness\_theorems

\bibitem{}
http://en.wikipedia.org/wiki/Entscheidungsproblem

\bibitem{}
http://en.wikipedia.org/wiki/Turing\_machine

\bibitem{}
http://mathworld.wolfram.com/TuringMachine.html

\bibitem{}
http://en.wikipedia.org/wiki/Universal\_Turing\_machine

\bibitem{}
http://en.wikipedia.org/wiki/Halting\_problem

\bibitem{}
http://www.turing.org.uk/bio/index.html

\end{thebibliography}
\end{document}
