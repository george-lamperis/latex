\documentclass[12pt]{article}

\usepackage{amsmath, amsthm, amssymb,amscd}
\usepackage{setspace}
\usepackage{enumerate}
\usepackage[margin=1in]{geometry}


\newtheoremstyle{mystyle}% name
    {2ex}       % Space above
    {2ex}       % Space below
    {}          % Body font
    {}          % Indent amount (empty = no indent, \parindent = para indent)
    {\bfseries} % Thm head font
    {.}         % Punctuation after thm head
    {\newline}  % Space after thm head: \newline = linebreak
    {}          % Thm head spec

\theoremstyle{mystyle}

\newtheorem{thm}{Theorem}
\newtheorem{defn}[thm]{Definition}
\newtheorem{prop}[thm]{Proposition}
\doublespace

% ------------------------------------------------------------------------------
% Begin Document
% ------------------------------------------------------------------------------
\title{Math 300 - Research Essay 1}
\author{George Lamperis}
\date{}

\begin{document}
\maketitle

Alan Turing is a British mathematician, best known for his enormous influence on
Computer Science. Turing was born born on June 23, 1912.

\section{The Halting Problem}

An extremely important problem which he worked on was The Halting Problem. 
The Halting Problem was inspired by recent (at the time) developments in
mathematics. In 1931, Godel's Incompleteness Theorem showed a shocking result:
that there are true statementments about numbers which can not be proven from
basic axioms. A further question was posed by David Hilbert, known as the
Entscheidungsproblem: does there exist some method of determining whether a 
given mathematical statement was provable?
The Halting Problem is stated as follows: Given a program and an input, decide
whether the program eventually finishes or if the program will run forever. This
is important because a program which runs forever may not be very useful. In
1936, Alan Turing proved that, in general, there is not an algorithm to
determine whether or not a program-input pair will halt. At the same time,
Turing had also shown that a general solution to Hilbert's Entscheidungsproblem
does not exist. It's worth noting that Alonzo Church had independently proved
the same results as Turing in 1936. However, Turing's ideas were different and
original enough to have a wider impact than Church's work.

A key part of solving this problem was to develop precise definitions of a
computer and of a computer program. This would lead Turing to develop a
theoretical model of a computer program which we now call a Turing machine. A
Turing machine consists of:
\begin{itemize}
  \item A finite set of symbols.
  \item An infinitely long roll of tape which is divided into squares. Each
    square may be empty, or it may have a symbol printed on it. The tape may be
    moved forward and backwards.
  \item An active square on the, known as the ``head'' or the ``scanned
    symbol''. The head symbol maybe be read or a new symbol written to the tape.
  \item A state register which keeps track of the state of the Turing machine.
  \item A table of instructions which, given the state and the symbol written on
    the scanned symbol, specifies how to manipulate the state and/or active
    symbol.
\end{itemize}
This model differed from Church's because it resembles a mechanical device. This
idea proves to be extremely useful precisely because it could be implemented as
a physical machine. Although this model may seem to be a primitive computer, a
Turing machine can in fact compute anything that is computable. Computer
Scientists continue to use the Turing Machine to understand the capabilities and
limitiations of computers. We can think of the Universal Turing Machine as the
computer, while a Turing machine is a computer program. Turing also thought of
Abbreviated Code Instructions, a system where a computer could expand a program
from an abbreviated form. This is essentially the plans for the first
programming language. ACI would allow a computer to be programmed faster and
more easily.

After the end of World War II, Turing set out to actually implement the
Universal Turing Machine that he had envisioned. However a radar engineer named 
F. C. Williams would beat him, developing the first implementation of a 
Turing machine in June 1948. Nevertheless, Alan Turing's ideas made an enormous
impact on the field of computing.


Many people believe that it's impossible to make a computer which cannot
simulate a Turing machine.

Most programming languages are Turing complete.

\section{World War II}

% This is all numberphile

Prior to World War II, German engineers had developed the Enigma machine, a
mechanical and electrical device used for enciphering written messages.
The Enigma machine had 3 to 5 rotors, each with 26 settings, allowing for an
extremely large number of settings. Additionally, the Enigma machine has a
plugboard on the back which allows even more possible configurations. The
plugboard swaps two letters, and up to 10 pairs could be used. Each setting
creates a unique code. The security of the Enigma code relies on the fact that
there are an astronomically large amount of possible settings.

To transmit a secret message with the Enigma machine, first pick a
configuration for the rotors and plugboard. Next, type your message into the
Enigma machine, and copy the encrypted message. Since the Enigma machine does
not trasmit messages itself, you will have to deliver the message yourself. This
was often done using Morse Code. Then, the recipient must configure his Enigma
machine using the same settings as the sender used. Finally, the recipient types
the encrypted message on the Enigma machine, which will output the original
message.

During World War II, the German military made heavy use of the Enigma machine to
securely pass messages between their troops. If the Allies were able to decipher
the German military's communication, then clearly the Allies would have a great
advantage. Because of this, the British and Polish militaries made it a priority
to crack the Enigma code. Polish mathematicians had worked out the electrical
wiring of the Enigma, leading to a good understanding of how it worked. 
However, the challenge lies in the settings on the plugboard.

Alan Turing made several key observations that led to the breaking of the Enigma
code. One of these observations was that the Enigma code would never encipher a 
letter to itself. So for example, an ``a'' in the original message would never
appear as ``a'' in the enciphered message. Turing realized that this was a flaw
in the code.
    For example, it was extremely common for German messages to end with ``Heil
Hitler''.

Turing eventually developed an algorithm which could crack the Enigma code. The 
algorithm starts by choosing one Enigma setting. Assume this is the correct
setting, and begin deciphering until a contradiction is found: either a letter
goes to itelf, or two letters go to the same letter. When a contradiction is
devived, we can deduce that this setting is wrong. Furthermore, anything
following from this contradiction is also false, which allowed Turing to
drastically reduce the number of settings which must be checked.

Although the number of settings to check has been reduced, there still remains a
large number of settings. Because of this, Turing and other British
mathematicians built the Bombe machine, a machine which carried out his
algorithm much faster than could be performed by hand. The Bombe machine could 
work out the correct settings for the Enigma in about 20 minutes. Speed was
quite important because the German military would change the setting of their
Enigma machines every day.

 With the Enigma code broken, the Allies now had a huge advantage.
They could read the Nazi's messages which often included their military tactics and plans. 
This would prove to be crucial in several operations, such as D-Day. Prior to
D-Day, the Allies set a trap; they distributed messages containing plans for an
attack several miles away from Normandy Beach. Because the Enigma had been
cracked, the Allies were able to confirm that the Nazi's had fallen for their
trap. The Nazi's positioned troops at the site of the dummy invasion, while the
Allied troops stormed the beaches of Normandy instead.


\begin{thebibliography}{9}

\bibitem{numberphile1}
http://youtu.be/BdrrJ7qd4HA,
Numberphile

\bibitem{}
http://en.wikipedia.org/wiki/G\%C3\%B6del's\_incompleteness\_theorems

\bibitem{}
http://en.wikipedia.org/wiki/Entscheidungsproblem

\bibitem{}
http://en.wikipedia.org/wiki/Turing\_machine

\bibitem{}
http://mathworld.wolfram.com/TuringMachine.html

\bibitem{}
http://en.wikipedia.org/wiki/Universal\_Turing\_machine

\bibitem{}
http://en.wikipedia.org/wiki/Halting\_problem

\bibitem{}
http://www.turing.org.uk/bio/index.html

\end{thebibliography}
\end{document}
