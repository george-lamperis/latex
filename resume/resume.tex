\RequirePackage[l2tabu, orthodox]{nag}

\documentclass[letterpaper]{article}
\usepackage[margin=1in]{geometry} 
\usepackage{paralist}
\usepackage[parfill]{parskip}
\usepackage[compact]{titlesec}
\usepackage{url}

\titleformat{\section}{\normalfont\bfseries}{\thesection}{1em}{\uppercase}
\titlespacing\section{0pt}{4pt}{-4pt}

\pagestyle{empty}   % no page numbers

% TODO clickable links
% TODO specify font
\begin{document}

% ------------------------------------------------------------------------------
% Heading
\begin{center}
    \textbf{\LARGE{George Lamperis}} \\
    4525 N. Mozart      \hfill  (773) 332-7963  \\
    Chicago, IL 60625   \hfill  george.lamperis@gmail.com \\
    \line(1,0){450}
\end{center}

% ------------------------------------------------------------------------------
% Objective --------------------------------------------------------------------
\section*{Objective}
Seeking a internship or position that will allow me to apply my programming 
and/or math background.

% Education --------------------------------------------------------------------
\section*{Education}
\textbf{Northside College Prep High School} \hfill \emph{Class of 2010} \\
GPA: 3.61/5.0

\textbf{University of Illinois at Chicago (UIC)} \hfill \emph{expected December 2014} \\
Bachelor of Arts in Mathematics \\
GPA: 3.11/4.0 \\
Relevant Coursework:
\vspace{-1ex}
\begin{compactitem}
    \item Data Structures I and II
    \item Programming Practicum (C programming)
\end{compactitem}


% Projects ---------------------------------------------------------------------
\section*{Projects}
\begin{compactitem}
    \item \url{https://github.com/Rusty-Shackleford/Rubiks-Cube} \\
    This project renders a Rubik's Cube using OpenGL. The user can rotate the 
    camera and turn the faces of the Rubik's Cube using the keyboard and mouse. 
    A solver is currently in progress. I find this project to be an excellent 
    combination of my interests and skills. My math background is particularly 
    useful for this project.
    \vspace{1ex}

    \item \url{https://github.com/Rusty-Shackleford/CTA-Tracker} \\
    An Android app which displays arrival predictions for CTA trains and buses 
    using the CTA Train Tracker API and CTA Bustime API, respectively. Besides 
    Java, I also used Python to generate a database of bus and train data.
\end{compactitem}


% Skills -----------------------------------------------------------------------
\section*{Skills}
Languages:
\vspace{-1ex}
\begin{compactitem}
    \item Java
    \item C/C++ 
    \item Python
    \item SQLite3.
\end{compactitem}

Linux - installation, configuration and usage.

Familiar with common data structures, sorting algorithms, Object Oriented Programming, Git, and unit testing.

Strong Math background: Calculus I-III, Linear Algebra I, Abstract Algebra I, 
Analysis I  

% Work Experience --------------------------------------------------------------
\section*{Work Experience}
\textbf{Whitetail Carpentry (Family Owned Business)} \hfill \emph{06/2009 - Present} \\
Laboror \\
Responsibilities:
\vspace{-1ex}
\begin{compactitem}
    \item Use of various tools to accomplish a task
    \item Communicating with tenants and building staff
    \item Purchasing and transporting building materials
\end{compactitem}

References available on request. 

\end{document}
