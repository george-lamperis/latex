\RequirePackage[l2tabu, orthodox]{nag}
\documentclass[letterpaper, 11pt, oneside]{memoir}
\usepackage[margin=1in]{geometry} 
\usepackage{paralist}
\usepackage[compact]{titlesec}
\usepackage{url}

% no page numbers
\pagestyle{empty}   

% block paragraphs
\setlength{\parindent}{0cm}
\setlength{\parskip}{1ex}

\begin{document}

% ------------------------------------------------------------------------------
% Heading
% ------------------------------------------------------------------------------
\begin{center}
    \textbf{\LARGE{George Lamperis}} \\
    4525 N. Mozart      \hfill  (773) 332-7963  \\
    Chicago, IL 60625   \hfill  george.lamperis@gmail.com \\
    \line(1,0){450}
\end{center}

% ------------------------------------------------------------------------------
% Objective
% ------------------------------------------------------------------------------
\vspace{-3ex}
\section*{Objective}
Seeking a internship or co-op that will allow me to apply my programming and/or
math background.

% ------------------------------------------------------------------------------
% Education
% ------------------------------------------------------------------------------
\section*{Education}
\textbf{Northside College Prep High School} \hfill \emph{Class of 2010} \\
GPA: 3.61/5.0

\textbf{University of Illinois at Chicago (UIC)} \hfill \emph{expected December 2014} \\
Bachelor of Arts in Mathematics \\
GPA: 3.11/4.0

\textbf{Relevant Coursework}

Computer Science:
\begin{compactitem}
    \item Discrete Structures I and II
    \item Programming Practicum (C programming)
    \item Algorithms I
    \item Machine Organization
\end{compactitem}

Math:
\begin{compactitem}
    \item Calculus I-III 
    \item Linear Algebra I
    \item Abstract Algebra I
    \item Analysis I
    \item Foundations of Number Theory
\end{compactitem}


% Projects ---------------------------------------------------------------------
\section*{Projects}
\begin{compactitem}
    \item \url{https://github.com/Rusty-Shackleford/Rubiks-Python} \\
    This is my attempt at a Rubik's Cube solver. Unfortunately it is far from
    complete. However, I'm rather proud of the Permutation class I wrote.
    
    \item \url{https://github.com/Rusty-Shackleford/Rubiks-Cube} \\
    This project renders a Rubik's Cube using OpenGL. The user can rotate the 
    camera and turn the faces of the Rubik's Cube using the keyboard and mouse. 

    \item \url{https://github.com/Rusty-Shackleford/CTA-Tracker} \\
    An Android app which displays arrival predictions for CTA trains and buses 
    using the CTA Train Tracker API and CTA Bustime API, respectively. 
\end{compactitem}


% Skills -----------------------------------------------------------------------
\section*{Skills}
\begin{compactitem}
    \item Languages - Java, C, C++, Python
    \item Linux - installation, configuration and usage.
    \item Familiar with common data structures and sorting algorithms
\end{compactitem}

References available on request. 

\end{document}
