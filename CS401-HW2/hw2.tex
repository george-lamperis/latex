\RequirePackage[l2tabu, orthodox]{nag}
\documentclass[letterpaper, 12pt, oneside]{memoir}
\usepackage{amsmath,amsthm,amssymb,mathtools}

% for fancy figures
\usepackage{tikz}
\usetikzlibrary{trees}
\usepackage[labelformat=empty]{caption}

% block paragraphs
\pagestyle{plain}
\setlength{\parindent}{0cm}
\setlength{\parskip}{1ex}


\newcommand{\treediagram}[9] {
  \begin{tikzpicture}[level distance=1cm,
  level 1/.style={sibling distance=2cm},
  level 2/.style={sibling distance=1cm}]
  \node {#1}
    child {
      node {#2}
      child {
        node {#4}
        child {node{#8}}
        child {node{#9}}
      }
        child {node {#5}
      }
    }
    child {
      node {#3}
      child {
        node {#6}
      }
      child {
        node {#7}
      }
    };
  \end{tikzpicture}
}


% ------------------------------------------------------------------------------
% Begin Document
% ------------------------------------------------------------------------------
\title{CS 401 - Homework 2}
\author{George Lamperis}
\date{}

\begin{document}
\maketitle

% ------------------------------------------------------------------------------
% Exercise 5-2-1, page 122
% ------------------------------------------------------------------------------
\section*{Exercise 5-2-1, page 122}
In HIRE-ASSISTANT, assuming that the candidates are presented in a random order,
what is the probability that you hire exactly one time? What is the probability
that you hire exactly n times?


A note about notation: when I say candidate 2, I mean the second candidate
interviewed. When I say the 2nd ranked candidate, I mean the candidate which is
second-most qualified. In this case, the 1st ranked candidate is the least 
qualified and the $n$-th ranked candidate is the most qualified. 


In order to make exactly one hire, it must be the case that the $n$-th ranked
candidate is interviewed first. Since the candidates are presented in a random
order, the probability of the best candidate being interviewed first is $1/n$.

In order to make $n$ hires, it must be the case that candidate 1 is ranked 1st,
candidate 2 is ranked 2nd, \ldots, candidate $n$ is ranked $n$-th. That is,
each subsequent candidate is ranked better than the previous. There is only one 
permutation of candidates such that this holds. Hence, the probability of making
$n$ hires is $1/(n!)$.


% ------------------------------------------------------------------------------
% Exercise 5-2-2, page 122
% ------------------------------------------------------------------------------
\section*{Exercise 5-2-2, page 122}
In HIRE-ASSISTANT, assuming that the candidates are presented in a random order,
what is the probability that you hire exactly twice?

In order to make exactly two hires, it must be the case that the second best
candidate is interviewed first. This is the only condition required.

We always hire the first candidate interviewed. If we interview the $(n-1)$-th
ranked candidate first, there is only one remaining candidate that is better.
In fact, it doesn't matter when we interview the $n$-th ranked candidate,
since we will make no more hires until then (or after then). 

Hence, the probability of making exactly two hires is $1/n$.


% ------------------------------------------------------------------------------
% Exercise 6-3-1, page 159
% ------------------------------------------------------------------------------
\section*{Exercise 6-3-1, page 159}
Using Figure 6.3 as a model, illustrate the operation of BUILD-MAX-HEAP on the
array $A = \{5, 3, 17, 10, 84, 19, 6, 22, 9 \}$.


\begin{figure}[h]
    \centering
    \treediagram{5}{3}{17}{*10}{84}{19}{6}{22}{9}
    $\rightarrow$
    \treediagram{5}{3}{17}{22}{84}{19}{6}{10}{9}

    \caption{$i = 4$}
\end{figure}

\begin{figure}[h]
    \centering
    \treediagram{5}{3}{*17}{22}{84}{19}{6}{10}{9}
    $\rightarrow$
    \treediagram{5}{3}{19}{22}{84}{17}{6}{10}{9}
    
    \caption{$i = 3$}
\end{figure}

\begin{figure}[h]
    \centering
    \treediagram{5}{*3}{19}{22}{84}{17}{6}{10}{9}
    $\rightarrow$
    \treediagram{5}{84}{19}{22}{3}{17}{6}{10}{9}

    \caption{$i = 2$}
\end{figure}

\begin{figure}[h]
    \centering
    \treediagram{*5}{84}{19}{22}{3}{17}{6}{10}{9}
    $\rightarrow$
    \treediagram{84}{5}{19}{22}{3}{17}{6}{10}{9}

    \caption{$i = 1$}
\end{figure}


% ------------------------------------------------------------------------------
% Exercise 6-4-1, page 160
% ------------------------------------------------------------------------------
\section*{Exercise 6-4-1, page 160}
Using Figure 6.4 as a model, illustrate the operation of HEAPSORT on the array
$A = \{ 5, 13, 2, 25, 7, 17, 20, 8, 4 \} $.

BUILD-MAX-HEAP

\treediagram{5}{13}{2}{25}{7}{17}{20}{8}{4}
$\rightarrow$
\treediagram{25}{13}{20}{8}{7}{17}{2}{5}{4}

$i=9$

\treediagram{*25}{13}{20}{8}{7}{17}{2}{5}{*4}
$\rightarrow$
\treediagram{*4}{13}{20}{8}{7}{17}{2}{5}{*25}
$\rightarrow$
\treediagram{20}{13}{17}{8}{7}{4}{2}{5}{25}


$i=8$

\treediagram{*20}{13}{17}{8}{7}{4}{2}{*5}{25}
$\rightarrow$
\treediagram{*5}{13}{17}{8}{7}{4}{2}{*20}{25}
$\rightarrow$
\treediagram{17}{13}{5}{8}{7}{4}{2}{20}{25}

$i=7$

\treediagram{*17}{13}{5}{8}{7}{4}{*2}{20}{25}
$\rightarrow$
\treediagram{*2}{13}{5}{8}{7}{4}{*17}{20}{25}
$\rightarrow$
\treediagram{13}{8}{5}{2}{7}{4}{17}{20}{25}

$i=6$

\treediagram{*13}{8}{5}{2}{7}{*4}{17}{20}{25}
$\rightarrow$
\treediagram{*4}{8}{5}{2}{7}{*13}{17}{20}{25}
$\rightarrow$
\treediagram{8}{7}{5}{2}{4}{13}{17}{20}{25}

$i=5$

\treediagram{*8}{7}{5}{2}{*4}{13}{17}{20}{25}
$\rightarrow$
\treediagram{*4}{7}{5}{2}{*8}{13}{17}{20}{25}
$\rightarrow$
\treediagram{7}{4}{5}{2}{8}{13}{17}{20}{25}

$i=4$

\treediagram{*7}{4}{5}{*2}{8}{13}{17}{20}{25}
$\rightarrow$
\treediagram{*2}{4}{5}{*7}{8}{13}{17}{20}{25}
$\rightarrow$
\treediagram{5}{4}{2}{7}{8}{13}{17}{20}{25}

$i=3$

\treediagram{*5}{4}{*2}{7}{8}{13}{17}{20}{25}
$\rightarrow$
\treediagram{*2}{4}{*5}{7}{8}{13}{17}{20}{25}
$\rightarrow$
\treediagram{4}{2}{5}{7}{8}{13}{17}{20}{25}

$i=2$

\treediagram{*4}{*2}{5}{7}{8}{13}{17}{20}{25}
$\rightarrow$
\treediagram{*2}{*4}{5}{7}{8}{13}{17}{20}{25}


% ------------------------------------------------------------------------------
% Exercise 6-4-3, page 160, for decreasing order only.
% ------------------------------------------------------------------------------
\section*{Exercise 6-4-3, page 160}


\end{document}
