\RequirePackage[l2tabu, orthodox]{nag}
\documentclass[letterpaper, 12pt, oneside]{memoir}
\usepackage{amsmath,amsthm,amssymb,mathtools}

% for fancy figures
\usepackage{tikz}
\usetikzlibrary{trees}

% block paragraphs
\pagestyle{plain}
\setlength{\parindent}{0cm}
\setlength{\parskip}{1ex}


\newcommand{\treediagram}[9] {
  \begin{tikzpicture}[level distance=1cm,
  level 1/.style={sibling distance=2cm},
  level 2/.style={sibling distance=1cm}]
  \node {#1}
    child {
      node {#2}
      child {
        node {#4}
        child {node{#8}}
        child {node{#9}}
      }
        child {node {#5}
      }
    }
    child {
      node {#3}
      child {
        node {#6}
      }
      child {
        node {#7}
      }
    };
  \end{tikzpicture}
}


% ------------------------------------------------------------------------------
% Begin Document
% ------------------------------------------------------------------------------
\title{CS 401 - Homework 2}
\author{George Lamperis}
\date{}

\begin{document}
\maketitle

% ------------------------------------------------------------------------------
% Exercise 5-2-1, page 122
% ------------------------------------------------------------------------------
\section*{Exercise 5-2-1, page 122}
In HIRE-ASSISTANT, assuming that the candidates are presented in a random order,
what is the probability that you hire exactly one time? What is the probability
that you hire exactly n times?


Since all candidates are equally likely to be hired, the probability that the
first candidate is the best is $1/n$.

Given $n$ candidates, in order to make $n$ hires, it must be the case that
each subsequent candidate is ranked better than the previous. For example, 
candidate 2 is better than candidate 1, candidate 3 is better than candidate 2,
and so on. There is only one permutation of candidates such that this holds. 
Hence, the probability of making $n$ hires is $1/n!$.

% ------------------------------------------------------------------------------
% Exercise 5-2-2, page 122
% ------------------------------------------------------------------------------
\section*{Exercise 5-2-2, page 122}
In HIRE-ASSISTANT, assuming that the candidates are presented in a random order,
what is the probability that you hire exactly twice?

Some temporary filler.

Some temporary filler.


% ------------------------------------------------------------------------------
% Exercise 6-3-1, page 159
% ------------------------------------------------------------------------------
\section*{Exercise 6-3-1, page 159}
Using Figure 6.3 as a model, illustrate the operation of BUILD-MAX-HEAP on the
array $A = \{5, 3, 17, 10, 84, 19, 6, 22, 9 \}$.

$i = 4$

\treediagram{5}{3}{17}{*10}{84}{19}{6}{22}{9}
$\rightarrow$
\treediagram{5}{3}{17}{22}{84}{19}{6}{10}{9}

$i = 3$

\treediagram{5}{3}{*17}{22}{84}{19}{6}{10}{9}
$\rightarrow$
\treediagram{5}{3}{19}{22}{84}{17}{6}{10}{9}

$i = 2$

\treediagram{5}{*3}{19}{22}{84}{17}{6}{10}{9}
$\rightarrow$
\treediagram{5}{84}{19}{22}{3}{17}{6}{10}{9}

$i = 1$

\treediagram{*5}{84}{19}{22}{3}{17}{6}{10}{9}
$\rightarrow$
\treediagram{84}{5}{19}{22}{3}{17}{6}{10}{9}

% ------------------------------------------------------------------------------
% Exercise 6-4-1, page 160
% ------------------------------------------------------------------------------
\section*{Exercise 6-4-1, page 160}
Using Figure 6.4 as a model, illustrate the operation of HEAPSORT on the array
$A = \{ 5, 13, 2, 25, 7, 17, 20, 8, 4 \} $.

BUILD-MAX-HEAP

\treediagram{5}{13}{2}{25}{7}{17}{20}{8}{4}
$\rightarrow$
\treediagram{25}{13}{20}{8}{7}{17}{2}{5}{4}

$i=9$

\treediagram{*25}{13}{20}{8}{7}{17}{2}{5}{*4}
$\rightarrow$
\treediagram{*4}{13}{20}{8}{7}{17}{2}{5}{*25}
$\rightarrow$
\treediagram{20}{13}{17}{8}{7}{4}{2}{5}{25}


$i=8$

\treediagram{*20}{13}{17}{8}{7}{4}{2}{*5}{25}
$\rightarrow$
\treediagram{*5}{13}{17}{8}{7}{4}{2}{*20}{25}
$\rightarrow$
\treediagram{17}{13}{5}{8}{7}{4}{2}{20}{25}

$i=7$

\treediagram{*17}{13}{5}{8}{7}{4}{*2}{20}{25}
$\rightarrow$
\treediagram{*2}{13}{5}{8}{7}{4}{*17}{20}{25}
$\rightarrow$
\treediagram{13}{8}{5}{2}{7}{4}{17}{20}{25}

$i=6$

\treediagram{*13}{8}{5}{2}{7}{*4}{17}{20}{25}
$\rightarrow$
\treediagram{*4}{8}{5}{2}{7}{*13}{17}{20}{25}
$\rightarrow$
\treediagram{8}{7}{5}{2}{4}{13}{17}{20}{25}


% ------------------------------------------------------------------------------
% Exercise 6-4-3, page 160, for decreasing order only.
% ------------------------------------------------------------------------------
\section*{Exercise 6-4-3, page 160}


\end{document}
