\RequirePackage[l2tabu, orthodox]{nag}
\documentclass[letterpaper, 12pt, oneside]{memoir}
\usepackage{amsmath,amsthm,amssymb,mathtools}

% for fancy figures
\usepackage{tikz}
\usetikzlibrary{trees}
\usepackage[labelformat=empty]{caption}
\usepackage[section]{placeins}  % for Floatbarriers

% block paragraphs
\pagestyle{plain}
\setlength{\parskip}{1ex}


\newcommand{\treediagram}[9] {
  \begin{tikzpicture}[level distance=1cm,
  level 1/.style={sibling distance=1.5cm},
  level 2/.style={sibling distance=0.75cm}]
  \node {#1}
    child {
      node {#2}
      child {
        node {#4}
        child {node{#8}}
        child {node{#9}}
      }
        child {node {#5}
      }
    }
    child {
      node {#3}
      child {
        node {#6}
      }
      child {
        node {#7}
      }
    };
  \end{tikzpicture}
}


% ------------------------------------------------------------------------------
% Begin Document
% ------------------------------------------------------------------------------
\title{CS 401 - Homework 2}
\author{George Lamperis}
\date{}

\begin{document}
\maketitle

% ------------------------------------------------------------------------------
% Exercise 5-2-1, page 122
% ------------------------------------------------------------------------------
\section*{Exercise 5-2-1, page 122}
A note about notation: when I say candidate 2, I mean the second candidate
interviewed. When I say the 2nd ranked candidate, I mean the candidate which is
second-most qualified. In this case, the 1st ranked candidate is the least 
qualified and the $n$-th ranked candidate is the most qualified. 


In order to make exactly one hire, it must be the case that the $n$-th ranked
candidate is interviewed first. Since the candidates are presented in a random
order, the probability of the best candidate being interviewed first is $1/n$.

In order to make $n$ hires, it must be the case that candidate 1 is ranked 1st,
candidate 2 is ranked 2nd, \ldots, candidate $n$ is ranked $n$-th. That is,
each subsequent candidate is ranked better than the previous. There is only one 
permutation of candidates such that this holds. Hence, the probability of making
$n$ hires is $1/(n!)$.


% ------------------------------------------------------------------------------
% Exercise 5-2-2, page 122
% ------------------------------------------------------------------------------
\section*{Exercise 5-2-2, page 122}
We always hire the first candidate interviewed. So to make two hires, it cannot
be the case that the $n$-th ranked candidate is interviewed first.

If we want to make exactly two hires, then first we interview and hire the
$k$-th ranked candidate. Then, we interview but do not hire up to $k-1$ 
candidates which rank lower than the $k$-th candidate. Next, we interview and
hire the $n$-th ranked candidate. Finally, we interview but do not hire any 
remaining candidates which rank lower than the $n$-th candidate.

More formally, fix $k \in [1, \ n-1]$ and then choose $m \in [0, \ k-1]$.

We want a permutation of candidates:
\[ k, \ a_1, \ a_2, \ \ldots, \ a_m, \ n, \ b_1, \ b_2, \ \ldots, \
b_{n-m-2} \]
such that each $a_i$ ranks lower than $k$, and each $b_i$ ranks lower than $n$.
Notice that there are $m$ many $a_i$ terms and $n-m-2$ many $b_i$ terms.

There are $n-1$ choices for $k$, and for each $k$ there are $k$ choices for $m$.
For each $k$ and for each $m$, there are $m! * (n-m-2)!$ possible arrangments
of candidates.

Unfortunately, the probablility of this is really difficult to find :(
% ------------------------------------------------------------------------------
% Exercise 6-3-1, page 159
% ------------------------------------------------------------------------------
\section*{Exercise 6-3-1, page 159}
Apply BUILD-MAX-HEAP to $A = \{5, 3, 17, 10, 84, 19, 6, 22, 9 \}$:

\begin{figure}[h!]
    \centering
    \treediagram{5}{3}{17}{*10}{84}{19}{6}{22}{9}
    $\rightarrow$
    \treediagram{5}{3}{17}{22}{84}{19}{6}{10}{9}

    \caption{$i = 4$}
\end{figure}

\begin{figure}[h!]
    \centering
    \treediagram{5}{3}{*17}{22}{84}{19}{6}{10}{9}
    $\rightarrow$
    \treediagram{5}{3}{19}{22}{84}{17}{6}{10}{9}
    
    \caption{$i = 3$}
\end{figure}

\begin{figure}[h!]
    \centering
    \treediagram{5}{*3}{19}{22}{84}{17}{6}{10}{9}
    $\rightarrow$
    \treediagram{5}{84}{19}{22}{3}{17}{6}{10}{9}

    \caption{$i = 2$}
\end{figure}

\begin{figure}[h!]
    \centering
    \treediagram{*5}{84}{19}{22}{3}{17}{6}{10}{9}
    $\rightarrow$
    \treediagram{84}{22}{19}{10}{3}{17}{6}{5}{9}

    \caption{$i = 1$}
\end{figure}

% ------------------------------------------------------------------------------
% Exercise 6-4-1, page 160
% ------------------------------------------------------------------------------
\section*{Exercise 6-4-1, page 160}
Apply HEAPSORT on the array $A = \{ 5, 13, 2, 25, 7, 17, 20, 8, 4 \} $.

\begin{figure}[h!]
    \centering
    \treediagram{5}{13}{2}{25}{7}{17}{20}{8}{4}
    $\rightarrow$
    \treediagram{25}{13}{20}{8}{7}{17}{2}{5}{4}
    \caption{BUILD-MAX-HEAP}
\end{figure}

\begin{figure}[h!]
    \centering
    \treediagram{*25}{13}{20}{8}{7}{17}{2}{5}{*4}
    $\rightarrow$
    \treediagram{*4}{13}{20}{8}{7}{17}{2}{5}{*25}
    $\rightarrow$
    \treediagram{20}{13}{17}{8}{7}{4}{2}{5}{25}

    \caption{$i=9$}
\end{figure}

\begin{figure}[h!]
    \centering
    \treediagram{*20}{13}{17}{8}{7}{4}{2}{*5}{25}
    $\rightarrow$
    \treediagram{*5}{13}{17}{8}{7}{4}{2}{*20}{25}
    $\rightarrow$
    \treediagram{17}{13}{5}{8}{7}{4}{2}{20}{25}
    
    \caption{$i=8$}
\end{figure}

\begin{figure}[h!]
    \centering
    \treediagram{*17}{13}{5}{8}{7}{4}{*2}{20}{25}
    $\rightarrow$
    \treediagram{*2}{13}{5}{8}{7}{4}{*17}{20}{25}
    $\rightarrow$
    \treediagram{13}{8}{5}{2}{7}{4}{17}{20}{25}

    \caption{$i=7$}
\end{figure}

\begin{figure}[h!]
    \centering
    \treediagram{*13}{8}{5}{2}{7}{*4}{17}{20}{25}
    $\rightarrow$
    \treediagram{*4}{8}{5}{2}{7}{*13}{17}{20}{25}
    $\rightarrow$
    \treediagram{8}{7}{5}{2}{4}{13}{17}{20}{25}

    \caption{$i=6$}
\end{figure}

\begin{figure}[h!]
    \centering
    \treediagram{*8}{7}{5}{2}{*4}{13}{17}{20}{25}
    $\rightarrow$
    \treediagram{*4}{7}{5}{2}{*8}{13}{17}{20}{25}
    $\rightarrow$
    \treediagram{7}{4}{5}{2}{8}{13}{17}{20}{25}

    \caption{$i=5$}
\end{figure}

\begin{figure}[h!]
    \centering
    \treediagram{*7}{4}{5}{*2}{8}{13}{17}{20}{25}
    $\rightarrow$
    \treediagram{*2}{4}{5}{*7}{8}{13}{17}{20}{25}
    $\rightarrow$
    \treediagram{5}{4}{2}{7}{8}{13}{17}{20}{25}

    \caption{$i=4$}
\end{figure}

\begin{figure}[h!]
    \centering
    \treediagram{*5}{4}{*2}{7}{8}{13}{17}{20}{25}
    $\rightarrow$
    \treediagram{*2}{4}{*5}{7}{8}{13}{17}{20}{25}
    $\rightarrow$
    \treediagram{4}{2}{5}{7}{8}{13}{17}{20}{25}

    \caption{$i=3$}
\end{figure}

\begin{figure}[h!]
    \centering
    \treediagram{*4}{*2}{5}{7}{8}{13}{17}{20}{25}
    $\rightarrow$
    \treediagram{*2}{*4}{5}{7}{8}{13}{17}{20}{25}
    
    \caption{$i=2$}
\end{figure}


% ------------------------------------------------------------------------------
% Exercise 6-4-3, page 160, for decreasing order only.
% ------------------------------------------------------------------------------
\section*{Exercise 6-4-3, page 160}
Given an array in decreasing order, we already have a max-heap. For example,
the array $\{ 9, 8, 7, 6, 5, 4, 3, 2, 1\}$ corresponds to a tree which is
already a max heap.
\begin{figure}[h!]
    \centering
    \treediagram{9}{8}{7}{6}{5}{4}{3}{2}{1}
\end{figure}





\end{document}
