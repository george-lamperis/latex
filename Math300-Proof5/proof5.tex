\documentclass[12pt]{article}

\usepackage{amsmath, amsthm, amssymb,amscd}
\usepackage{setspace}
\usepackage{enumerate}
\usepackage[margin=1in]{geometry}


\newtheoremstyle{mystyle}% name
    {2ex}       % Space above
    {2ex}       % Space below
    {}          % Body font
    {}          % Indent amount (empty = no indent, \parindent = para indent)
    {\bfseries} % Thm head font
    {.}         % Punctuation after thm head
    {\newline}  % Space after thm head: \newline = linebreak
    {}          % Thm head spec

\theoremstyle{mystyle}

\newtheorem{thm}{Theorem}
\newtheorem{defn}[thm]{Definition}
\newtheorem{prop}[thm]{Proposition}
\newtheorem{remark}[thm]{Remark}
\newtheorem{example}[thm]{Example}
%\doublespace

% ------------------------------------------------------------------------------
% Begin Document
% ------------------------------------------------------------------------------
\title{Math 300 - Proof 5}
\author{George Lamperis}
\date{}

\begin{document}
\maketitle

\begin{remark}
  Notice that any natural number $n$ can be written as a power of 2 times an odd
  natural number, i.e. we can write $n = 2^k * s$.
\end{remark}

\begin{defn}
  Let $a,b$ be natural numbers such that $a = 2^k \cdot m$ and $b = 2^l \cdot n$,
  where $m,n$ are odd. We say $a \sim b$ if $m = n$. Following from the
  properites of natural numbers $\sim$ is an equivalence relation.
  
  Suppose $a = 2^k \cdot x$ and $b = 2^l \cdot x$. Then $a,b$ belong to the
  equivalence class $[x]$. Also notice that if $k < l$, we have $a \mid b$ and
  if $k \geq l$ we have $b \mid a$.
\end{defn}
    
\begin{prop}
  Let $n$ be a natural number, and let $S$ be a subset of ${1, 2,..., 2n}$ with
  $|S| = n + 1$. Then there are distinct $a,b$ in S such that $a$ divides $b$.
\end{prop}
\begin{proof}
  Notice that $\{1, 2, \ldots, 2n\}$ is partitioned into $n$ equivalence classes
  $[1], [3], \ldots, [2n-1]$. Recall that any two members $a,b \in [x]$
  we have $a \mid b$ or $b \mid a$. Hence, by the pidgeonhole principle, 
  any subset $S$ such that $|S| = n+1$ contains distinct elements $a,b$ such
  that $a \mid b$ or $b \mid a$.
\end{proof}

\begin{example}
  Here is $n=4$ as an example. Write each number from 1 to 8 as a power of 2
  times an odd number.
  \begin{align*}
    1 &= 2^0 \cdot 1 \\
    2 &= 2^1 \cdot 1 \\
    3 &= 2^0 \cdot 3 \\
    4 &= 2^2 \cdot 1 \\
    5 &= 2^0 \cdot 5 \\
    6 &= 2^1 \cdot 3 \\
    7 &= 2^0 \cdot 7 \\
    8 &= 2^3 \cdot 1
  \end{align*}
  
  From this, construct our four equivalence classes:
  \begin{align*}
    [1] &= \{1, 2, 4, 8\} \\
    [3] &= \{3, 6\} \\
    [5] &= \{5\} \\
    [7] &= \{7\}
  \end{align*}
  
  Notice that any two elements $a,b$ in the same equivalence class have the
  property that $a \mid b$ or $b \mid a$.
\end{example}

\end{document}