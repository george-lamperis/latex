\RequirePackage[l2tabu, orthodox]{nag}

\documentclass[letterpaper, 11pt]{article}

\usepackage[margin=1in]{geometry}
\usepackage{amsmath,amsthm,amssymb,mathtools}
\usepackage[parfill]{parskip}   % paragraph formatting
\usepackage{paralist}           % compactitem
\usepackage{array}              % required for tabular environment

% ------------------------------------------------------------------------------
% Theorem Commands
% ------------------------------------------------------------------------------
\newtheoremstyle{mystyle}% name
    {2ex}       % Space above
    {4ex}       % Space below
    {}          % Body font
    {}          % Indent amount (empty = no indent, \parindent = para indent)
    {\bfseries} % Thm head font
    {.}         % Punctuation after thm head
    {\newline}  % Space after thm head: \newline = linebreak
    {}          % Thm head spec

\theoremstyle{mystyle}
% \theoremstyle{plain}
\newtheorem*{prop}{Proposition}
\newtheorem*{sln}{Solution}




% ------------------------------------------------------------------------------
% Begin Document
% ------------------------------------------------------------------------------
\title{CS 401 - Homework 1}
\author{George Lamperis}
\date{UIN:664714097}

\begin{document}
\maketitle

% ------------------------------------------------------------------------------
% Exercise 3-1-1, page 52

% ------------------------------------------------------------------------------
\section{Exercise 3-1-1, page 52}

\begin{prop}
    Let $f(n)$ and $g(n)$ be asymptotically nonnegative functions. Then:
    \[max(f(n) + g(n)) = \Theta (f(n) + g(n))\]
\end{prop}

\begin{proof}
    Since $f(n)$ and $g(n)$ are asymptotically nonnegative, there exists positive
    constants $n_f, n_g$ such that:
    \begin{equation} 0 \leq f(n), \text{for all } n \geq n_f \end{equation}
    and
    \begin{equation} 0 \leq g(n), \text{for all } n \geq n_g \end{equation}

    Set $n_0 = max(n_f, n_g)$. Then $max(f(n), g(n))$ and $f(n) + g(n)$ are asymptotically
    nonnegative, i.e. for all $n \geq n_0$ we have:
    \begin{equation} 0 \leq max(f(n), g(n)) \end{equation}
    and
    \begin{equation} 0 \leq f(n)+g(n) \end{equation}

    We also have (for all $n \geq n_0$)
    \begin{equation} max(f(n), g(n)) \leq f(n)+g(n) \end{equation}

    Also note that $ \dfrac{f(n)+g(n)}{2} \leq max(f(n), g(n))$

    Now let $c_1 = \dfrac{1}{f(n) + g(n)}$ and let $c_2 = 1$.

    Hence: \[max(f(n) + g(n)) = \Theta (f(n) + g(n))\].
\end{proof}


% ------------------------------------------------------------------------------
% Exercise 3-1-4, page 53
% ------------------------------------------------------------------------------
\section{Exercise 3-1-4, page 53}
Is $2^{n+1} = O(2^n)$? 

\begin{sln}
    Pick $c=2$. Then $2 * 2^n = 2^{n+1}$. 

    Hence $2^{n+1} = O(2^n)$
\end{sln}

Is $2^{2n} = O(2^n)$?


% ------------------------------------------------------------------------------
% Problem 3-3-a, page 61
% ------------------------------------------------------------------------------
\section{Problem 3-3-a, page 61}

\begin{compactitem} 
    \item $n! = \Omega(2^{2^n})$ : $2^{2^n} < n!$ for all positive $n$. 

\end{compactitem}


% ------------------------------------------------------------------------------
% Exercise 4-4-5, page 93
% ------------------------------------------------------------------------------
\section{Exercise 4-4-5, page 93}


% ------------------------------------------------------------------------------
% Exercise 4-5-1, page 96
% ------------------------------------------------------------------------------
\section{Exercise 4-5-1, page 96}
Use the master method to give tight asymptotic bounds for the following recurrences.

In all cases, we have $a=2$ and $b=4$. Also, we have $n^{log_4 (2)} = O(\sqrt{n})$

\begin{sln}[a]
    \[T(n) = 2T(n/4) + 1\]

    Since $f(n) = 1 = O(n^{1/2 - \epsilon})$, where $\epsilon = 1/2$, we apply case 1.

    Hence $T(n) = \Theta (\sqrt{n})$.
\end{sln}

\begin{sln}[b]
    \[T(n) = 2T(n/4) + \sqrt{n}\]

    Since $f(n) = \sqrt{n} = \Theta (\sqrt{n})$, we apply case 2.

    Hence, $T(n) = \Theta (\sqrt{n} \log n)$
\end{sln}

\begin{sln}[c]
    \[T(n) = 2T(n/4) + n\]

    We have $f(n) = n = \Omega (n^{1/2+\epsilon})$, where $0 <\epsilon < 1/2$. 

    For all $n \geq 1$, pick $c = 1/2$. Then we have:
    \begin{align*}
        af(n/b) & = 2(n/4) \\
                & = n/2 \\
                & \leq n/2 \\
    \end{align*}

    Hence, by case 3, $T(n) = \Theta(n)$.
\end{sln}

\begin{sln}[d]
    \[T(n) = 2T(n/4) + n^2 \]

    We have $f(n) = n = \Omega (n^{1/2+\epsilon})$, where $0 <\epsilon < 3/2$.

    For all $n \geq 1$, pick $c = 1/2$. Then we have:
    \begin{align*}
        af(n^2/b)   & = 2(n^2/4) \\
                    & = n^2/2 \\
                    & \leq n^2/2 \\
    \end{align*}

    Hence, by case 3, $T(n) = \Theta(n)$. 
\end{sln}

\end{document}
