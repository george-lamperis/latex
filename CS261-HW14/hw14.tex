\RequirePackage[l2tabu, orthodox]{nag}

\documentclass[letterpaper, 12pt, oneside]{memoir}
\usepackage{amsmath,amsthm,amssymb,mathtools}
\usepackage{float} % H option to place tables exactly

% Memoir configuration
\pagestyle{plain}
%\setlength{\parindent}{0cm}
\setlength{\parskip}{1ex}

% Omit chapter numbering
\counterwithout{section}{chapter}

% ------------------------------------------------------------------------------
% Begin Document
% ------------------------------------------------------------------------------
\title{CS 261 - Homework w14}
\author{George Lamperis}
\date{}

\begin{document}
\maketitle

\section{Data}

\subsection{blocking.cc}
For every test with blocking.cc, we have 4,200,000 accesses. This data 
was taken using a direct mapped cache.

Blocking Factor: 20
Matrix size: 100
% ------------------------------------------------------------------------------
% blocking.cc
% ------------------------------------------------------------------------------
\begin{table}[H]
\centering
\begin{tabular}{c|c|c|r}
    Cache Size (KB)  & Block Size & Miss Rate & Total time \\ \hline 
    4  & 1 & 4.5    & 62865020 \\
    4  & 2 & 3.0    & 56889120 \\
    4  & 4 & 2.3    & 55645940 \\
    4  & 8 & 2.8    & 63088080 \\ \hline
    8  & 1 & 3.4    & 57504170 \\
    8  & 2 & 2.0    & 51951240 \\
    8  & 4 & 1.4    & 49987280 \\
    8  & 8 & 1.4    & 52446660 \\ \hline
    16 & 1 & 2.9    & 55605900 \\
    16 & 2 & 1.6    & 50142720 \\
    16 & 4 & 1.0    & 47808600 \\
    16 & 8 & 0.9    & 48560640 \\ \hline
    32 & 1 & 2.0    & 51315570 \\
    32 & 2 & 1.1    & 47424360 \\
    32 & 4 & 0.6    & 45652320 \\
    32 & 8 & 0.5    & 45765420 \\
\end{tabular}
\caption{blocking.cc}
\end{table}

\subsection{obvious.cc}
For every test with obvious.cc, we have 4,040,000 accesses.

Matrix size: 100
% ------------------------------------------------------------------------------
% obvious.cc
% ------------------------------------------------------------------------------
\begin{table}[H]
\centering
\begin{tabular}{c|c|c|r}
    Cache Size (KB)  & Block Size & Miss Rate & Total time \\ \hline 
    4  & 1 & 27.7   & 163548630 \\
    4  & 2 & 16.5   & 120387320 \\
    4  & 4 & 10.9   & 102084280 \\
    4  & 8 & 20.6   & 190459340 \\ \hline
    8  & 1 & 26.5   & 158077340 \\
    8  & 2 & 14.5   & 110739680 \\
    8  & 4 & 8.5    & 88656040 \\
    8  & 8 & 18.8   & 177242300 \\ \hline
    16 & 1 & 25.8   & 155259360 \\
    16 & 2 & 13.5   & 106037600 \\
    16 & 4 & 7.4    & 82203580 \\
    16 & 8 & 18.0   & 171396080 \\ \hline
    32 & 1 & 10.5   & 87068490 \\
    32 & 2 & 5.6    & 67332680 \\
    32 & 4 & 3.1    & 57918480 \\
    32 & 8 & 1.9    & 54344780 \\
\end{tabular}
\caption{obvious.cc}
\end{table}

\subsection{Hardware}

Matrix size: 1000
Blocking factor: 50
\begin{table}[H]
\centering
\begin{tabular}{c|c}
    Program & Execution Time (seconds) \\ \hline
    blocking.cc        & 8.26 \\
    obvious.cc         & 15.04 \\
    blocking.cc (-O2)  & 1.20 \\
    obvious.cc  (-O2)  & 16.39
\end{tabular}
\caption{hardware data}
\end{table}

\subsection{Associativity}
We'll use a blcok size of 1

\begin{table}[H]
\centering
\begin{tabular}{c|c|r}
    Associativity & Miss Rate (blocking.cc) & Total Time \\ \hline
    1           & 4.5   & 62865020 \\
    2           & 4.2   & 70003210 \\
    4           & 2.6   & 79303410 \\
    8           & 2.6   & 112900110 \\
    16          & 2.6   & 180100110 \\
    1024 (fully) & 2.6   & 8647300110 \\ \hline
    1           & 2.0   & 51315570 \\
    2           & 2.0   & 59717880 \\
    4           & 1.8   & 75466170 \\
    8           & 1.7   & 108500110 \\
    16          & 1.7   & 175700110 \\
    8192 (fully) & 1.7  & 68854100110 \\
\end{tabular}
\caption{blocking.cc with 4 kb cache (top) and 32 kb cache (bottom)}
\end{table}

\begin{table}[H]
\centering
\begin{tabular}{c|c|r}
    Associativity & Miss Rate (blocking.cc) & Total Time \\ \hline
    1           & 27.7  & 163548630 \\
    2           & 25.3  & 160724110 \\
    4           & 25.2  & 176840660 \\
    8           & 25.2  & 209160110 \\
    16          & 25.2  & 273800110 \\
    1024 (fully) & 25.2  & 8418440110 \\ \hline
    1           & 10.5  & 87068490 \\
    2           & 14.9  & 114713090 \\
    4           & 23.2  & 167921640 \\
    8           & 25.2  & 209160110 \\
    16          & 25.2  & 273800110 \\
    8192 (fully) & 25.2 & 66335880110 \\
\end{tabular}
\caption{obvious.cc with 4 kb cache (top) and 32 kb cache (bottom)}
\end{table}

\subsection{Blocking factor}
block size 1

\begin{table}[H]
\centering
\begin{tabular}{c|c|c}
    Blocking factor & Miss Rate (4 kb) & Miss Rate (32 kb) \\ \hline
    1   & 14.2  & 5.3 \\
    10  & 5.5   & 2.3 \\
    20  & 4.5   & 2.0 \\
    30  & 10.4  & 2.0 \\
    40  & 21.0  & 2.0 \\
    50  & 26.7  & 1.8 \\
    60  & 26.9  & 1.9 \\
    70  & 27.1  & 1.9 \\
    80  & 27.3  & 1.9 \\
    90  & 27.9  & 5.7 \\
\end{tabular}
\caption{different blocking factors with matrix size 100, direct mapped cache, and a 1 word block size}
\end{table}

% ------------------------------------------------------------------------------
% Questions
% ------------------------------------------------------------------------------
\section{Discussion Questions}


Impact of Associativity

The issue here is the impact of the choice of the parameter E, the number of
cache lines per set. The advantage of higher associativity (i.e., larger values of E)
is that it decreases the vulnerability of the cache to thrashing due to conflict misses.
However, higher associativity comes at a significant cost. Higher associativity is
expensive to implement and hard to make fast. It requires more tag bits per
line, additional LRU state bits per line, and additional control logic. Higher
associativity can increase hit time, because of the increased complexity, and it can
also increase the miss penalty because of the increased complexity of choosing a
victim line.


\subsection{1}
\textbf{A comparison of blocking.cc and obvious.cc in terms of their algorithmic 
complexity and cache-friendliness. Why did increasing the block size sometimes 
increase the miss rate?}

The obvious algorithm appears less complex, with only three nested loops, as
opposed to five nested using the blocking algorithm. One might think that
fewer nested loops would imply a faster execution time. However, this was
not the case.

The matrix $b$ is especially cache freindly, allowing us to exploit spacial
locality. On the first iteration of the inner-most loop, we must fetch
$b[i][k]$. As a result $b[i][k+1]$, $b[i][k+2]$, and so on will be stored in 
the cache as well. Therefore, we have quick access to this memory on successive
iterations of the loop. The line $ a[i][j] = a[i][j] + r$  exibits spacial
locality as well. As with the matrix $b$ above, other values in the same row 
as $a[i][j]$ will also be cached. 

Increasing the block size decreases the miss rate by increasing the ability to
exploit spacial locality. When you request one byte of memory, the
entire block in which that byte is stored gets loaded into the cache. Because 
of this, when we request one entry of the matrix, other nearby matrix entries
will be cached for later use. On successive iterations, the data we need to use
is already in the cache.

However, a larger block size means that a cache miss is more expensive. When
there is a cache miss, there is a larger about of data to be evicted and a
larger amount of new data to store in the cache. It takes more time to transfer
a larger amount of data. From our experimental data, it seems that a block size
of 4 seems to be optimal for smaller sized caches.




\subsection{2}
\textbf{What would be the optimal blockingFactor to use for matrix
multiplication?}

As blocking factor increases, we can fit more of our matrices in the cache, 
and exploit spacial locality. The innermost loop can run quickly because
all the data it needs is stored in the cache. 

However, as blocking factor increases too much, we lose the ability to exploit
spacial locality because our entire block doesn't fit in the cache. Thus, 
in the innner most loop, we must change the data stored in the cache.

The optimal blocking factor is 20 for a matrix of size 100.


\subsection{3}
\textbf{A discussion of the significance of the results, given that all of the
simulated caches were considerably smaller than current hardware caches.}

One thing to consider is that we tested a relatively simple program. Although
the cache is small, so is our problem size. 

\subsection{4}
\textbf{Does increased associativity always reduce the miss rate?}

No, increased associativity seems to keep miss rate constant. However, execution
time increases because high associativity becomes slower to implement at 
the hardware level.

Clearly, fully associative is only feasible with really small memory. Even then, 
associativity has diminishing returns. As you increase associativity, the miss
rate remains constant, but the total running time increases.

\subsection{5}
\textbf{How are caches implemented in hardware? Why does the simulator insist
that the block size, and the number of sets, must be a whole power of two? Is it always
necessary to store the entire block address in the cache for identification
purposes?}


\subsection{6}
\textbf{Why does a main memory access involve a significant, fixed overhead? How
can main memory systems be designed to support caches?}


\subsection{7}
\textbf{What are the attractions of a multilevel caching strategy?}

Although L2 cache is slower than L1 cache, it is orders of magnitude faster
than main memory. Still an improvement.



\end{document}
