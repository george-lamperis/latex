\RequirePackage[l2tabu, orthodox]{nag}

\documentclass[letterpaper, 12pt, oneside]{memoir}
\usepackage{amsmath,amsthm,amssymb,mathtools}
\usepackage{float} % H option to place tables exactly

% Memoir configuration
\pagestyle{plain}
\setlength{\parindent}{0cm}
\setlength{\parskip}{1ex}

% Omit chapter numbering
\counterwithout{section}{chapter}

% ------------------------------------------------------------------------------
% Begin Document
% ------------------------------------------------------------------------------
\title{CS 261 - Homework w14}
\author{George Lamperis}
\date{}

\begin{document}
\maketitle

record the miss rates for direct mapped caches of size 4, 8, 16 and 32 KBytes,
with block sizes of 1, 2, 4 and 8 words (32 measurements in total). 

For every test, we have 4,200,000 accesses.
% ------------------------------------------------------------------------------
% blocking.cc
% ------------------------------------------------------------------------------
\begin{table}[H]
\centering
\begin{tabular}{c|c|c|c}
    Cache Size (KB)  & Block Size & Miss Rate & Total time \\ \hline 
    4  & 1 & 4.5    & 62865020 \\
    4  & 2 & 3.0    & 56889120 \\
    4  & 4 & 2.3    & 55645940 \\
    4  & 8 & 2.8    & 63088080 \\ \hline
    8  & 1 & 3.4    & 57504170 \\
    8  & 2 & 2.0    & 51951240 \\
    8  & 4 & 1.4    & 49987280 \\
    8  & 8 & 1.4    & 52446660 \\ \hline
    16 & 1 & 2.9    & 55605900 \\
    16 & 2 & 1.6    & 50142720 \\
    16 & 4 & 1.0    & 47808600 \\
    16 & 8 & 0.9    & 48560640 \\ \hline
    32 & 1 & mr & tt \\
    32 & 2 & mr & tt \\
    32 & 4 & mr & tt \\
    32 & 8 & mr & tt \\
\end{tabular}
\caption{blocking.cc}
\end{table}

% ------------------------------------------------------------------------------
% obvious.cc
% ------------------------------------------------------------------------------
\begin{table}[H]
\centering
\begin{tabular}{c|c|c|c}
    Cache Size (KB)  & Block Size & Miss Rate & Total time \\ \hline 
    4  & 1 & mr & tt \\
    4  & 2 & mr & tt \\
    4  & 4 & mr & tt \\
    4  & 8 & mr & tt \\ \hline
    8  & 1 & mr & tt \\
    8  & 2 & mr & tt \\
    8  & 4 & mr & tt \\
    8  & 8 & mr & tt \\ \hline
    16 & 1 & mr & tt \\
    16 & 2 & mr & tt \\
    16 & 4 & mr & tt \\
    16 & 8 & mr & tt \\ \hline
    32 & 1 & mr & tt \\
    32 & 2 & mr & tt \\
    32 & 4 & mr & tt \\
    32 & 8 & mr & tt \\
\end{tabular}
\caption{obvious.cc}
\end{table}

\end{document}
