\documentclass[12pt]{article}

\usepackage{amsmath, amsthm, amssymb,amscd}
\usepackage{setspace}
\usepackage{enumerate}
\usepackage[margin=1in]{geometry}


\newtheoremstyle{mystyle}% name
    {2ex}       % Space above
    {2ex}       % Space below
    {}          % Body font
    {}          % Indent amount (empty = no indent, \parindent = para indent)
    {\bfseries} % Thm head font
    {.}         % Punctuation after thm head
    {\newline}  % Space after thm head: \newline = linebreak
    {}          % Thm head spec

\theoremstyle{mystyle}

\newtheorem{thm}{Theorem}
\newtheorem{defn}[thm]{Definition}
\newtheorem{prop}[thm]{Proposition}
\doublespace

% ------------------------------------------------------------------------------
% Begin Document
% ------------------------------------------------------------------------------
\title{Math 300 - Essay 3}
\author{George Lamperis}
\date{}

\begin{document}
\maketitle

Although mathematics seems to have an answer for everything, there are in fact
several unsolved problems in mathematics. An important example of a currently
unsolved problem is $P$ versus $NP$.


Before we can discuss $P$ versus $NP$, we need to first talk about problem
complexity.
When we talk about the complexity of a problem, we do not necessarily care
about the exact number of steps to solve the problem. Instead, we study the
running time for a problem, the upper bounds for the number of steps than an
efficent algorithm requires to solve the problem. As the problem size $n$ grows,
we describe the running time with a function of $n$.

A decision problem is a question which can be answered with ``yes'' or ``no''.
Two important classifications of decision problems are $P$ and $NP$.  A decision
problem is in $P$ if we can find a solution in polynomial time, while a problem
is in $NP$ if we can check a solution in polynomial time. Thus, to find a
solution, we must try every possibility, and so problems in $NP$ run in exponential time. 
Because exponential functions grow faster than polynomial functions, we can
consider problems in $P$ to be easier than problems in $NP$. Clearly $P \subset
NP$ because an exponential function also serves as an upper bounds for problems in $P$.
The P versus NP problem asks if $P = NP$. That is, if a problem can be verified
in polynomial time, can a solution also be found in polynomial time?


The solution of $P$ versus $NP$ has several important consequences. A proof that
$P = NP$ would mean that many problems which we thought were difficult in fact
have an efficient solution. As a result, many of our cyptographic systems,
such as RSA and AES, would be vulnerable, as an efficient way to break them
exists. However, most computer scientists have the opinion that $P \neq NP$. The
consequences of proving this are far less exciting. However, a proof that $P
\neq NP$ would still be considered quite useful, as it solves a problem which
has puzzled computer scientists since 1971.


\end{document}
